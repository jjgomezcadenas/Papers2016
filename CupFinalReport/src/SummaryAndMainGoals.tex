CUP (Canfranc Underground Physics) was granted with the main goal of developing the scientific program of the
 \emph{Laboratorio Subterr\'aneo de Canfranc} (LSC), one of the ICTS (instalaci\'on cient\'ifico t\'ecnica singular) of the ministry of science (currently integrated in the MINECO). The two main activities of CUP were called
 CAFE (Canfranc Future Experiment) and NEXT (Neutrino Experiment with a Xenon TPC).

CAFE has spawn a wide range of phenomenological studies, coordinated by professor 
M.C. Gonz\'alez-Garc\'ia. These studies have involved several institutions of the CUP consortium, including the Universidad de Barcelona (UB), Instituto de F\'isica Corpuscular, (IFIC), and Instituto de F\'isica Te\'orica (IFT). 

The goals of the subproject CAFE were:
\begin{enumerate}
\item The characterisation of LSC to demonstrate its viability 
as a large underground European laboratory in the context of
the European project LAGUNA (Large Apparatus for Studying Grand
Unification and Particle Astrophysics) as well as in the context
of studies of viability for future neutrino factories which were
performed as part of the European project EURO-$\nu$.
\item Perform phenomenological studies to explore and exploit 
the physics potential LSC  as well more broad studies related
to astroparticle physics in general (neutrino physics, cosmology, etc.)
\end{enumerate}

%The CAFE activities were amply described in our previous reports, and here we will simply update the lists of publications related with the activity. 

%Two major spinoffs of the activity are two European projects directly related with CAFE goals: 
%
%\begin{enumerate}
%\item 
%A european training network, called INVISIBLES, coordinates by professor B. Gavela, from IFT. 
%%
%\item A european project of large infrastructures called LAGNA-LBNO. The spanish IP of the project is  professor J.J. Gómez-Cadenas, for IFIC, who acts also as spokesperson of NEXT.
%\end{enumerate}

CUP main goal was the construction, commissioning and operation of the NEXT detector\footnote{\url{http://next.ific.uv.es/next/}}, a high-pressure, xenon (HPXe) Time Projection Chamber (TPC), whose goal was to search for neutrinoless double beta decay  (\bbonu) events in xenon enriched at 90\% in the isotope \XE. The first phase of the experiment, called NEW, deploying 10 kg of xenon, is currently being commissioned at the LSC.The second phase of the experiment, NEXT-100, will deploy 100 kg of xenon. A third phase, deploying up to one ton of xenon, is actively being discussed, with strong international interest.
 
%
%The discovery potential of NEXT is very large. It combines four desirable features that make it an almost-ideal experiment for \bbonu\ searches, namely: 
%\begin{enumerate}
%\item Excellent energy resolution (better than 1\% FWHM in the region of interest).
%\item A topological signature (the observation of the tracks of the two electrons).
%\item A fully active, very radiopure apparatus of large mass.  
%\item The capability of extending the technology to much larger masses.
%\end{enumerate}
%
%The project has evolved very satisfactorily, from the initial Letter of Intent (LOI) in 2009 to the Technical Design Report (TDR) in 2012. A substantial number of papers, proceeding reports and conferences, documenting and demonstrating the physics case, the results of the prototypes and the technological choices have been published and are attached to this document. They can also be found in the NEXT web page: \url{http://next.ific.uv.es/next/talks.html}.
%

