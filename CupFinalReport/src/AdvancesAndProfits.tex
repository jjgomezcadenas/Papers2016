CAFE has result into several advances
\begin{enumerate}
\item {\bf It has enhanced the visibility of the LSC} by participating into
European studies devoted to the long term planning for large infrastructures
in Europe such as of such as LAGUNA-LBNO and Euro-$\nu$.
\item {\bf It has contributed to the  creation of a new generation of European physicists}
with expertness in phenomenology of relevance for underground physics
by our participation in two consecutive International Training Networks
and a Network for Research and Innovation Staff Exchange, all
funded by the European Commission, as reported below. 
\item  {\bf It has  produced a large number of scientific articles, which have
advanced quantitatively the state-of-the-art of the field.} The total
number of references of published articles by the CAFE senior members
in international peer reviewed journals 
since 2008 is 70  (listed below) that  have  received a total of 
more than 3000 citations in the SPIRES data basis which is the one commonly used
in our field for reference.
\end{enumerate}

NEXT has result in several advances:
\begin{enumerate}
\item {\bf It has advanced the field of underground physics instrumentation}, with the construction of the NEXT detector series. The large prototype which has operated at IFIC since 2011 (NEXT-DEMO) was the largest HPXe electroluminescent TPC in the World, until the construction of NEW, currently being commissioned at the LSC. The detectors have shown the feasibility of the electroluminescent technology, have demonstrated the excellent energy resolution that can be achieved in xenon gas, and have demonstrated  the availability of a unique topological signature in HPXe, e.g, the capability to fully reconstruct the two electrons emitted in \bb\ decay, providing a powerful tool to discriminate against backgrounds. 
\item {\bf It has resulted in a cutting-edge experiment}, which will be searching for \bbonu\ decays during the next few years, with chances of making or contributing to a major discovery. 
\item {\bf It has bought to the field a new technology}, that of the HPXe-EL, which is currently considered one of the best candidates for the next generation of \bbonu\ experiments.
 \item {\bf It has resulted in an international collaboration}, involving groups from Portugal, United States, Russia and Colombia. Other groups (Poland, Australia), are currently considering joining the experiment.
 \item {\bf It has attracted external funding}, with contributions to the experiment from all the members of the collaboration, and in particular from the US.
  \item {\bf It has resulted in an Advanced Grant of the ERC}, granted to the spokesperson of NEXT, prof. J.J. G\'omez Cadenas, who has acted as co-PI of CUP. 
   \item {\bf It has boosted the scientific interest of the LSC}, as intended in our proposal. Currently NEXT is the most important experiment in the Canfranc Underground Laboratory and the only one with truly international projection. 
  \item {\bf It has produced a major scientific spinoff, called PETALO}, a new concept for a full-body, TOF-capable PET using xenon. Currently, a patent has been filled and two scientific papers have been published on the new concept. A new collaboration is being formed to explore the scientific potential and commercial applications of PETALO. 
   \item {\bf It has contributed in an important way to science outreach}. See for example:
   \url{http://www.jotdown.es/2012/09/david-nygren-y-alessandro-bettini-the-physics-as-fountain-of-eternal-youth/}
   http://www.jotdown.es/2013/11/coversaciones-de-fisica-en-el-santa-cristina-ariella-cattai-y-concha-gonzalez-garcia/
\end{enumerate}
