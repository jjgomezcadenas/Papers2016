
\subsection*{NEXT: Reports from the LSC Scientific Committee}

In the CUP proposal, we stated:
\begin{quotation}
We propose a very simple and efficient evaluation scheme for CUP. The  activities are focused in the LSC, which has an international scientific committee, composed by high-reputation specialists in underground physics. This committee is perfectly suited to evaluate our progress in all the areas proposed.
\end{quotation}

The reports of the LSC Scientific Committee (LSCSC) have been consistently encouraging and enthusiastic towards NEXT. They represent a continuous and methodic follow up of a very difficult and complex project, and they
also keep a detailed historical track of the steady progress of the experiment. At the same time, the reports reflect the concerns of the LSCSC regarding a perceived lack of domestic support to the project, in particular during 2011 and 2012.
It is important to remark that the LSCSC is an independent body, formed almost exclusively by experts of international reputation. This adds tremendous weight to the opinions, recommendations and commendations expressed in their reports and permits an objective evaluation of the degree of success of the project. 

{\em
In appendix A, we offer a detailed discussion of the LSC reports. 
}

\subsection*{Other key indicators for NEXT. Publications, international projection and grants}

\begin{itemize}
\item {\bf Publications}: the project has generated a large number of publications in refereed journals and conferences, as well as posters and communications. A full list of publications can be found online\footnote{\url{http://next.ific.uv.es/next/talks.html}}.
\item{\bf International projection}: NEXT is a CERN recognised experiment and is included among the most important
\bbonu\ experiments in Europe. See for example the recent presentation of A. Giuliani in the APPEC meeting \footnote{\url{http://app2016.in2p3.fr/programme.html}}.
\item{\bf Grants:} The spokesperson of NEXT has obtained and Advanced Grant of the ERC for the NEXT project. 
\end{itemize}

\subsection*{CAFE: key indicators}

%\begin{quotation}
%As for the CAFE activity, there is an obvious evaluation criteria, namely the scientific papers to be produced along the five years of the project, including the proceedings of the workshop(s) and/or schools as well as other outreach material.
%\end{quotation}
%
The key indicators of the final results of CAFE are the following:

\begin{enumerate}
\item Involvement in the project LAGUNA-LBNO
(\url{http://laguna.ethz.ch:8080/Plone}). This project continued the
studied started by LAGUNA with the aim at developing large infrastructures
in Europe. Spain participated with four associates LAGUNA-LBO, three of
which are participants in CUP (UAM, IFIC y el propio LSC).
\item Involvement of our scientists in the european project EURO-$\nu$.
In particular de group at UAM made very relevant contributions to
this project as listed in the bibliography.
\item Participation in three funded projects of the European Commission
for the training of young researchers in fields directly related
with underground physics. They are: 
\begin{itemize}
\item 
Title: INVISIBLES\\
Initial year: 2012 Year final: 2016\\
Reference: PITN-GA-2011-289442\\
\item 
Title: 	Elusives\\
Initial year: 2016 Year final: 2020\\
Reference: H2020-MSCA-ITN-2015-674896 \\
\item  Title: 	InvisiblesPlus\\
Initial year: 2016 Year final: 2020\\
Reference: H2020-MSCA-RISE-2015-690575\\
\end{itemize}
The coordinator of the UB node of these networks is our IP, Prof.
M.C. Gonzalez-Garcia while the coordinator of the node at IFIC
is Prof. P. Hernandez, a member of CAFE/CUP.
\item Publication in peer review international journal 
of more than 70 articles in phenomenological studies of which 
we list a selection in the
bibliography. They have accumulated more than 3000 citation  in the
basis SPIRES. For example our updated determinations of the neutrino 
parameters, fundamental for any study of the expected signal at NEXT
\cite{Gonzalez-Garcia:2014bfa,GonzalezGarcia:2012sz,GonzalezGarcia:2010er} 
have accumulated more than 1000 citations,  
while our contributions to studies associated to future neutrino facilities,
such as Ref.~\cite{Bandyopadhyay:2007kx,Abazajian:2012ys,Choubey:2011zzq,Adey:2014rfv} have received more than 500 references (all data from SPIRES
Data basis).
\end{enumerate}

%
