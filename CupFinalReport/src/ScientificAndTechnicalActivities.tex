\subsection*{NEXT: Scientific and technical progress}

The NEXT experiment has evolved, largely thanks to the support of the Consolider CUP, from a visionary idea ({\em to setup an international experiment at the LSC able to be in the forefront of the Underground Physics field, with a high discovery potential}) to an established reality. After an initial period of R\&D (2010 and the first half of 2011), the project has evolved in a consistent way, first with the construction, commissioning and operation of NEXT-DEMO (2011--2013), which has produced a large scientific and technical output, then with NEW which started in earnest in 2014 and will carry its scientific program through 2018. The current plan is to present an update TDR in November of 2016, and start the construction of NEXT-100 in mid/late 2017 or 2018, depending on the available human resources and funding. 
 
 NEXT-DEMO and NEXT-DBDM were essential to fully demonstrate the performance of the HPXe-EL technology, including:
 \begin{enumerate}
\item An energy resolution near the Fano-Factor ($\sim$ 0.5 \% FWHM at \Qbb)\cite{Alvarez:2012yxw,Alvarez:2012zsz}.
\item First operation in the World of a tracking plane made of SiPMs \cite{Alvarez:2013gxa}.
\item Demonstration of the unique topological signature available in HPXe\cite{Ferrario:2015kta}.
\end{enumerate}

During the NEW phase of the project, the collaboration has addressed both scientific and technical challenges. In the scientific front:

\begin{enumerate}
\item Understanding of the radioactive budget \cite{Cebrian:2015jna, Alvarez:2014kvs, Dafni:2014yja}.
\item Definition of the detector sensitivity \cite{Martin-Albo:2015rhw}.
\item Demonstration of the unique topological signature available in HPXe\cite{Ferrario:2015kta}.
\end{enumerate}

And in the technical front:

\begin{enumerate}
\item Construction and certification of a large underground detector (NEW) for underground operation.
\item Construction and certification of large infrastructures (gas system, high-voltage system) for for underground operation.
\item Development of slow controls.
\item Risk analysis.
\item System integration.
\end{enumerate}

With the acquired experience, the collaboration is looking forward to an steady operation of NEW followed by the
timely construction, commissioning and operation of the NEXT-100 detector. The NEXT program is one of the major alternatives in Europe for a future, ton-scale, \bbonu\ experiment. 


\subsection*{CAFE: Scientific activity}

The scientific activity of CAFE has been described above. We list
here the main lines developed in their phenomenological studies
which are directly related to underground physics
\begin{itemize}
\item Precise and updated characterization of the leptonic flavour
parameters, whose values are crucial to the prediction of expected
event rates in neutrinoless double beta decay experiments
\cite{Song:2015xaa,Bergstrom:2015rba,Gonzalez-Garcia:2014bfa,Gonzalez-Garcia:2013usa,GonzalezGarcia:2012sz,GonzalezGarcia:2011my,GonzalezGarcia:2010un,GonzalezGarcia:2010er}
\item Studies of new phenomena expected from presence of additional
sterile neutrinos and their effect at future underground experiments
\cite{Vincent:2014rja,Bergstrom:2014fqa,Adey:2014rfv,GonzalezGarcia:2012yq,SolagurenBeascoa:2012cz,Donini:2011jh,Abazajian:2012ys}
\item Studies directly related to future underground facilities
in Europe 
\cite{Bross:2013oua,Edgecock:2013lga,Blennow:2013swa,Bayes:2012ex,Agarwalla:2012zu,Hernandez:2012zz,Coloma:2012wq,Coloma:2011rq,Donini:2010xk,Coloma:2010wa,Choubey:2009ks,Bandyopadhyay:2007kx}
\item Potential of underground experiments to detect and 
characterize astrophysical neutrino sources 
\cite{Bergstrom:2016cbh,Gonzalez-Garcia:2013iha,Ahlers:2011jj,Ahlers:2010fw,GonzalezGarcia:2009ya,GonzalezGarcia:2009jc}
\item Study of theoretical frameworks for the connection  between Majorana 
nature of the neutrino (as searched for
in neutrinoless doublebeta decay) and the production of the
observed matter-antimater in the Universe
\cite{Hernandez:2015wna,Fong:2013gaa,Fong:2011yx,Fong:2010bv,Fong:2010bh,Fong:2010qh,Fong:2010zu,GonzalezGarcia:2009qd,Fong:2009iu}
\item Construction of models which can account for all neutrino observations
and study their specific predictions in future experiments
(including neutrinoless doublebeta decay experiments)
\cite{Gago:2015vma,Hernandez:2014fha,Hernandez:2013lza,Donini:2012tt,Eboli:2011ia,Gavela:2009cd}
\item Studies of scenarios of electroweak symmetry breaking related
for mass generation
\cite{Corbett:2015lfa,Corbett:2014ora,Gavela:2014vra,Brivio:2014pfa,Brivio:2013pma,Corbett:2013pja,Corbett:2012ja,Corbett:2012dm,Eboli:2011ye,Eboli:2011bq,Eboli:2010qd,Alves:2009aa}
\end{itemize}
%
