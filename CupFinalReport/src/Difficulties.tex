The project went through a difficult period which extended from 2011 to late 2013. The difficulties started with the decision, in early 2011 to adopt the electroluminescence (EL) technology as the baseline for NEXT. Such a decision came after a long internal debate in the collaboration and was firmly rooted in scientific arguments. It has indeed proven to be correct, as discussed with more detail in Appendix B. The decision was supported by the LSC Scientific Committee, who wrote, in its May-2011 report.

\begin{quotation}
The committee congratulates the NEXT collaboration on very substantial progress made towards the definition of the conceptual design for the xenon double beta decay experiment. A major decision has been the choice of electroluminescence as the gain mechanism [...]

The committee concurs with the need to move expeditiously to build the detector if the results are to be competitive. The committee feels that the choice of electroluminescence is likely to lead to energy and tracking measurements adequate for the project.
\end{quotation}
 
The choice of EL was not only sound from the scientific point of view and supported by the LSCSC, but it was backed by the overwhelming majority of the collaboration. The alternative to EL was the proposal to use a gain-based TPC, with a Micromegas readout. 
%The main reason why this proposal was considered not competitive with EL was energy resolution, which had been demonstrated to be considerably worse than that of EL (3--4 \% FWHM to be compared with 0.5--0.7 \% FWHM, both at \Qbb, see Appendix B for details). 
%
The decision mechanism was based in the elaboration of Conceptual Design Report Proposals (CDRP). 
The Micromegas (MM) CDRP was signed by the group of Zaragoza and by the group of IFAE. The EL CDRP was signed by the rest of the collaboration (IFIC, UPV, US, IFT-UAM, UG, LBNL, TAMU and UAN) with the exception of CIEMAT which chose not to sign any of the two CDRPs. The two proposals were intensely debated inside the collaboration and the EL CDRP was endorsed by overwhelming majority of the board members. 

After the decision of adopting EL, the IFAE group opted out of the NEXT collaboration, sending a letter to the board which stated their lack of trust in the technology, the schedule and the management of the collaboration. For the soundness of the technology we refer to Appendix B. Regarding the aggressive schedule that the collaboration was attempting, we quote the report of the LSCSC (December 2011):

\begin{quotation}
The NEXT collaboration {\em has made tremendous progress} in defining the gaseous xenon detector for double beta decay. The conceptual design has been improved in many respects and prototype detectors of moderate scale have been successfully operated proving the main concepts of the proposed design. Excellent resolution has been demonstrated for 137Cs gamma rays at 662 keV \footnote{Relevant references: \cite{Alvarez:2012yxw, Alvarez:2012zsz,Alvarez:2012hu,Alvarez:2013gxa,Lorca:2014sra}} and good tracking has also been shown \footnote{Relevant references:  \cite{Ferrario:2015kta}}. {\em The committee strongly supports the need to expedite this project} as there are now 2 operating xenon detectors in the world and the scientific impact will be greatly reduced if the NEXT project cannot be completed on their proposed timescale.
\end{quotation}

Regarding managerial aspects, we quote again the LSCSC (May 2012, see Appendix A):

\begin{quotation}
{\em The NEXT Collaboration presented enormous progresses both on technical and on managerial domains} [...]

Managerial aspect: We congratulate the Collaboration for presenting a very convincing work-breakdown structure with convincing work packages (WP).
\end{quotation}

The withdrawal of the IFAE group was followed by the withdrawal of the CIEMAT group, after they decline to sign the NEXT Technical Design Report.
%\footnote{NB: with hindsight, it still appears that the withdrawal of IFAE was unavoidable, given their position, extremely critical of the NEXT management. Perhaps, however, the CIEMAT group could have been kept in NEXT by negotiating some temporary solutions that would have allowed them to delay the signature of the TDR for a given period of time.}.
%
Notice, on the other hand, that neither IFAE, nor CIEMAT withdrew formally from CUP. However, their activity in the consortium stopped. 

In 2011, the IFIC, UPV and UAM presented a join funding proposal to the open call for science projects, within the area of particle physics (FPA, for short). 
Notice that it was recognised since the beginning of CUP, that the CUP consolider grant alone could not supply  enough  funds for the construction of the NEXT- 100 detector and the associated infrastructure,
together with the salaries that were needed to create a hitherto non-existing group of experts at the leading institutions. Without additional funds from FPA, the construction of the needed infrastructures and the large underground detector was not realistic.

The FPA committee in 2011 included among the panel of experts, several leaders of IFAE and CIEMAT groups. It could be argue, therefore, that objectivity was not fully granted. 
%In fact, F. Sanchez presented the IFAE arguments to withdraw from NEXT during his public exposition {\em of a different research proposal}\footnote{The transparencies
%are available at \url{https://www.fpa.csic.es/docs/2011Exp/29823-C02-02\_Sanchez\_Nieto.pdf}}. Remarkably, as it can be followed in such presentation, Sanchez described the R\&D conducted at IFAE for NEXT, {\em which had focused in the EL option, not in the MM option}. Indeed, IFAE (and Sanchez) had focused their research in an EL detector, identical to the one finally approved, with the only difference that they proposed the use of Avalanche Photodiodes (APDs) for the tracking plane, while IFIC, LBNL, TAMU and others proposed the use of silicon fotomultipliers (SiPMs). The choice of one type of sensor or the other could be considered a relatively minor technological issue (and in fact, the evolution of technology has demonstrated without doubts that the choice of SiPMs, nowadays omnipresent in particle physics and medical imaging, while APDs have practically disappeared from the market was correct). Therefore it is hard to understand the decision to sign the MM proposal (the IFAE had worked since the beginning of the project in the EL option), or the criticism of a technical choice that was, in fact supported by their work, except for the minor detail of the choice of sensor. 
%
The join IFIC-UPV-UAM proposal submitted to FPA in 2011 obtained only a ``C'' (mediocre) and was rejected without funding. Regarding the possible bias of the FPA panel of experts and the referees designed by the ANEP coordinator, we offer the following remarks:

\begin{itemize}
\item There is no reference anywhere in the inform of the panel of experts to the very positive reports produced by the LSCSC.
\item The coordinator of the ANEP for FPA was the head of the experimental division at CIEMAT. The coordinator of ANEP has the role to request independent evaluators for a given project. However, at least one of the reports was extremely biased and included numerous subjective statements and attacks {\em at hominem}\footnote{We quote literally the opening statement of such report: {\em El IP tiene un historial lleno de luces y de sombras, y sus logros cient\'ificos no est\'an tan claros como \'el expresa. \'Ultimamente dedica m\'as tiempo a la escritura de novelas u otros libros que a ocuparse de los experimentos y de los consolider en los que tiene muchas responsabilidades.} The translation to English reads: {\em The PI has a trajectory full of lights and shadows and his accomplishments are not so obvious as he states. Lately he devotes more time to writing novels and other books than to care about the experiments en projects of which he is responsible}.}. But the report was not withdrawn by the coordinator and no additional reports were requested.
\item The panel of experts included also leading scientists from IFAE. Without discussing the obvious scientific merit of those scientists, one can raise reasonable doubts about a potential bias related with the situation. 
%In particular, Dr. Cavalli-Svorza has made clear {\em publicly} (the last time in a seminar at the ALBA installation, in 2015) his clear adversity towards the spokesperson of NEXT.
%\item The external expert was Dr. Giorgio Gratta, the spokesperson of EXO, which is a US-based experiment proposing to use liquid xenon, rather than an HPXe detector. Again, in spite of the prestige and clear scientific merits of Prof. Gratta, it is also a fact that EXO and NEXT are direct competitors, and one could have imagined other external referees with  less potential bias.  
\item The external expert was Prof. Giorgio Gratta from Stanford University,
an internationally recognised expert on Xe detectors. However, prof. Gratta
happens to be  the spokesperson  of EXO, a US-based experiment proposing to use liquid xenon to pursue a physics program very similar to that of NEXT.  So EXO and NEXT can be seen as direct competitors, and one could have imagined other external referees with  less potential bias.  
\end{itemize}
 
The project submitted in 2011 requested essential additional funding, not covered by CUP. The negative result implied a major logistic drawback and a serious blow to the morale of the collaboration. It also implied loosing international credibility, since from the point of view of international agencies (in particular DOE in the USA) it was unclear whether NEXT was being supported by the Spanish scientific authorities or not. It can be argued that the prospects of international funding and of expanding the NEXT collaboration suffered from the negative FPA decision in 2011. 

In 2012, a new research project was submitted, to a different committee (general physics, FIS), given the perceived bias of the FPA coordinator and his panel of experts. This time, the proposal  obtained a B (``Good'') and modest funding for 2 years. In 2013, the spokesperson of NEXT was granted the first (and to this day the only one in the area of FPA) advanced grant of the ERC. The process of selecting a project for an AdG grant implies the pair review of numerous international experts (in 2013 there were 8 independent referees evaluating NEXT) and requires marks above 90\% to be selected. In 2014, another project was submitted to FIS for 4 years. The project obtained an A (``excellent'') and obtained sizeable funding, in spite of the limited budget available. 

The problem may not be fully solved yet, since the NEXT project is still funded by an area other than FPA, a totally illogical situation, given the well established merits of the experiment and its relevance for the future of the LSC. NEXT leading groups will apply again for funding in 2018.  

%Clearly, there is an inconsistency which points towards a bias. In 2015, the gestor of FPA has changed. However, the perception of the collaboration is that the situation is not fully solved yet. While the ERC funds and the funds granted in 2014 has given the Spanish groups enough solvency to carry forward the project up to 2019, the uncertainties regarding   funding once the AdG and the current project expire remain. The collaboration faces the real possibility that insufficient funding in 2019 hinder our competitiveness precisely in a moment where Europe will be, very likely, considering what are the candidates for a ton-scale experiment. NEXT has the real chance of becoming one of the leading \bbonu\ experiments in Europe in the NEXT decade, projecting consequently the relevance of the LSC, {\em but only if there is firm, clear and sustained support to the project from the side of the Spanish scientific authorities.} 
 

%The rejection of the research proposal in 2011 and the very scarce funding in 2012, forced the collaboration to redefine its initial plan to start NEXT-100 construction in 2012. Instead, it was decided to introduce a first-stage experiment, NEW (a detector with half of the size of NEXT-100, a fourth of the sensors and 10 kg of xenon mass). It was perceived that there was not sufficient support in Spain to attempt the construction of NEXT-100 directly. After the ERC grant the situation has improved considerably, but the collaboration decided to continue with the NEW program. Currently, NEW is starting data taking at the LSC, and the lessons learned in its construction (and certainly by operating it) are very valuable to build a better NEXT-100 detector. 
%
