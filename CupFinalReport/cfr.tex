\documentclass[11pt,a4paper]{article}
\usepackage[utf8]{inputenc}
\usepackage[spanish]{babel}
\usepackage[a4paper,height=26cm,width=17cm]{geometry}
\usepackage{graphicx}
\usepackage{epsfig,rotating}
\usepackage{amssymb}
\usepackage{mathrsfs}
\usepackage{amsmath}
\usepackage{amsfonts}
\usepackage{multirow}
\usepackage{eurosym}
\usepackage{dcolumn}
\usepackage{cite}

\usepackage{multirow}
%\usepackage{enumitem}

\usepackage[hyphens]{url}
%\usepackage{hyperref}


%% Hack to make math formulas bold in section titles
\makeatletter
\DeclareRobustCommand*{\bfseries}{%
  \not@math@alphabet\bfseries\mathbf
  \fontseries\bfdefault\selectfont
  \boldmath
}
\makeatother



\begin{document}

\input{../Commands.tex}

\begin{center}
{\LARGE \bf \textsf{CONSOLIDER-INGENIO 2010 PROGRAMME }} \\ \vspace{0.5cm}
{\Large \bf \textsf{FINAL SCIENTIFIC ANNUAL REPORT}} \\ \vspace{0.5cm}
{\Large \bf \textsf{I01/01/2012 al 31/12/2012 }} \\ \vspace{2cm}
\end{center}


\begin{table}[htp!]
\begin{center}
\begin{tabular}{|c|c|}
\hline
PROJECT REFERENCE NUMBER: & CDS2008-00037\\
\hline
Coordinating Researcher:  & M.C. G\'onzalez-Garc\'ia \\
Project Title: & Canfranc Underground Physics (CUP) \\
Managing Institution:  & CSIC \\
Project Start Date:  & 15/12/2008\\
Project Final Date: &  14/12/2014)\\
\hline\hline
\end{tabular}
\end{center}
\caption{Project data}
\label{default}
\end{table}%

\newpage

%%%%%%%%%%%%%%%%%%%%%%%%%%%%%%%%%%%%%%%%%%%%%%%%%%%%%%%%%
%% SECTION 1
\section{\bf \textsf{CUP: summary and main goals }}

CUP (Canfranc Underground Physics) was granted with the main goal of developing the scientific program of the
 \emph{Laboratorio Subterr�neo de Canfranc} (LSC), one of the ICTS (instalaci�n cient\'ifico t�cnica singular) of the ministry of science (currently integrated in the MINECO). The two main activities of CUP were called
 CAFE (Canfranc Future Experiment) and NEXT (Neutrino Experiment with a Xenon TPC).

CAFE has spawn a wide range of phenomenological studies, coordinated by professor 
M.C. Gonz\'alez-Garc\'ia, who is also the coordinator of CUP (PI hereafter). These studies have involved several institutions of the CUP consortium, including the Universidad de Barcelona (UB), Instituto de F�sica Corpuscular, (IFIC), and Instituto de F�sica Te�rica (IFT). 
%The CAFE activities were amply described in our previous reports, and here we will simply update the lists of publications related with the activity. 

%Two major spinoffs of the activity are two European projects directly related with CAFE goals: 
%
%\begin{enumerate}
%\item 
%A european training network, called INVISIBLES, coordinates by professor B. Gavela, from IFT. 
%%
%\item A european project of large infrastructures called LAGNA-LBNO. The spanish IP of the project is  professor J.J. G�mez-Cadenas, for IFIC, who acts also as spokesperson of NEXT.
%\end{enumerate}

CUP main goal was the construction, commissioning and operation of the NEXT detector\footnote{\url{http://next.ific.uv.es/next/}}, a high-pressure, xenon (HPXe) Time Projection Chamber (TPC), whose goal was to search for neutrinoless double beta decay  (\bbonu) events in xenon enriched at 90\% in the isotope \XE. The first phase of the experiment, called NEW, deploying 10 kg of xenon, is currently being commissioned at the Canfranc Underground Laboratory (LSC).The second phase of the experiment, NEXT-100, will deploy 100 kg of xenon. A third phase, deploying up to one ton of xenon, is actively being discussed.
 
%
%The discovery potential of NEXT is very large. It combines four desirable features that make it an almost-ideal experiment for \bbonu\ searches, namely: 
%\begin{enumerate}
%\item Excellent energy resolution (better than 1\% FWHM in the region of interest).
%\item A topological signature (the observation of the tracks of the two electrons).
%\item A fully active, very radiopure apparatus of large mass.  
%\item The capability of extending the technology to much larger masses.
%\end{enumerate}
%
%The project has evolved very satisfactorily, from the initial Letter of Intent (LOI) in 2009 to the Technical Design Report (TDR) in 2012. A substantial number of papers, proceeding reports and conferences, documenting and demonstrating the physics case, the results of the prototypes and the technological choices have been published and are attached to this document. They can also be found in the NEXT web page: \url{http://next.ific.uv.es/next/talks.html}.
%


\section{\bf \textsf{Main advances and profits }}

CAFE has produced a large number of scientific articles, which have advanced quantitatively the state-of-the-art of the field. The total number of references is XXX. 

NEXT has result in several advances:
\begin{enumerate}
\item {\bf It has advanced the field of underground physics instrumentation}, with the construction of the NEXT detector series. The large prototype which has operated at IFIC since 2011 (NEXT-DEMO) was the largest HPXe electroluminescent TPC in the World, until the construction of NEW, currently being commissioned at the LSC. The detectors have shown the feasibility to transform all the signals in xenon to VUV light, have demonstrated the excellent energy resolution that can be achieved in xenon gas, and have shown the feasibility to reconstruct the two electrons emitted in \bb\ decay (this is called the topological signature). 
\item {\bf It has resulted in a cutting-edge experiment}, which will be searching for \bbonu\ decays during the next few years, with chances of making or contributing to a major discovery. 
\item {\bf It has bought to the field a new technology}, that of the HPXe-EL, which is currently considered one of the candidates for the next generation of \bbonu\ experiments.
 \item {\bf It has resulted in an international collaboration}, involving groups from Portugal, United States, Russia and Colombia. Other groups (Poland, Australia), are currently considering joining the experiment.
 \item {\bf It has attracted external funding}, with contributions to the experiment from all the members of the collaboration.
  \item {\bf It has resulted in an Advanced Grant of the ERC}, granted to the spokesperson of NEXT, prof. J.J. G\'omez Cadenas, who has acted as co-PI of CUP. 
   \item {\bf It has boosted the scientific interest of the LSC}, as intended in our proposal. Currently NEXT is the most important experiment in the Canfranc Underground Laboratory and the only one with truly international projection. 
  \item {\bf It has produced a major scientific spinoff, called PETALO}, a new concept for a full-body, TOF-capable PET using xenon. 
   \item {\bf It has contributed in an important way to science outreach}. See for example:
   \url{http://www.jotdown.es/2012/09/david-nygren-y-alessandro-bettini-the-physics-as-fountain-of-eternal-youth/}
   http://www.jotdown.es/2013/11/coversaciones-de-fisica-en-el-santa-cristina-ariella-cattai-y-concha-gonzalez-garcia/
\end{enumerate}


\section{\bf \textsf{ Key Indicators}}

In the proposal, we stated:
\begin{quotation}
We propose a very simple and efficient evaluation scheme for CUP. The  activities are focused in the LSC, which has an international scientific committee, composed by high-reputation specialists in underground physics. This committee is perfectly suited to evaluate our progress in all the areas proposed.
\end{quotation}

and
\begin{quotation}
As for the CAFE activity, there is an obvious evaluation criteria, namely the scientific papers to be produced along the five years of the project, including the proceedings of the workshop(s) and/or schools as well as other outreach material.
\end{quotation}



\section{\bf \textsf{ Difficulties}} 


4.-Describe the difficulties and/or problems that may have been encountered during the development of the project, as well as any change that have been occurred with respect to the objectives of the initial work plan

\section{\bf \textsf{ Scientific and technical activities }} 

5.-Describe the scientific and technical activities to reach the goals outlined in the project. Indicate, for each activity, the members of the equipment who have participated

\section{\bf \textsf{Collaborations between team members}}
6.-Relate the collaborations between all team members 

\section{\bf \textsf{Collaborations with other groups of research}}

7.-Relate the collaborations with other groups of research and the added value for the project. Describe, if necessary, the access to equipment or infrastructures of other groups or institutions

\section{\bf \textsf{Collaborations with companies}}

8.-Relate the collaborations with companies or other socioeconomic sectors and the added value for the project, the knowledge transference or other results

\section{\bf \textsf{UE FP7 calls}}

9.-Indicate if you have attended, and how effective has been, to any of UE FP7 calls (project, training, infrastructures) and/or to other international programs in topics related to this project. Specify the program, partners, countries and topics and, where appropriate, funding received

\section{\bf \textsf{National and international Program visibility}}

10.-National and international Program visibility

\section{\bf \textsf{Problems and suggestions}}

11.- Problems and suggestions
\end{document}

