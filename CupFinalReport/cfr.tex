\documentclass[11pt,a4paper]{article}
\usepackage[utf8]{inputenc}
\usepackage[spanish]{babel}
\usepackage[a4paper,height=26cm,width=17cm]{geometry}
\usepackage{graphicx}
\usepackage{epsfig,rotating}
\usepackage{amssymb}
\usepackage{mathrsfs}
\usepackage{amsmath}
\usepackage{amsfonts}
\usepackage{multirow}
\usepackage{eurosym}
\usepackage{dcolumn}
\usepackage{cite}

\usepackage{multirow}
%\usepackage{enumitem}

\usepackage[hyphens]{url}
%\usepackage{hyperref}


%% Hack to make math formulas bold in section titles
\makeatletter
\DeclareRobustCommand*{\bfseries}{%
  \not@math@alphabet\bfseries\mathbf
  \fontseries\bfdefault\selectfont
  \boldmath
}
\makeatother



\begin{document}

% BB
\newcommand{\bb}{\ensuremath{\beta\beta}}
% BB0NU
\newcommand{\bbonu}{\ensuremath{\beta\beta0\nu}}
% BB2NU
\newcommand{\bbtnu}{\ensuremath{\beta\beta2\nu}}
% NME
\newcommand{\Monu}{\ensuremath{\Big|M_{0\nu}\Big|}}
\newcommand{\Mtnu}{\ensuremath{\Big|M_{2\nu}\Big|}}
% PHASE-SPACE FACTOR
\newcommand{\Gonu}{\ensuremath{G_{0\nu}(\Qbb, Z)}}
\newcommand{\Gtnu}{\ensuremath{G_{2\nu}(\Qbb, Z)}}

% mbb
\newcommand{\mbb}{\ensuremath{m_{\beta\beta}}}
\newcommand{\kgy}{\ensuremath{\rm kg \cdot y}}
\newcommand{\ckky}{\ensuremath{\rm counts/(keV \cdot kg \cdot yr)}}
\newcommand{\mbba}{\ensuremath{m_{\beta\beta}^a}}
\newcommand{\mbbb}{\ensuremath{m_{\beta\beta}^b}}
\newcommand{\mbbt}{\ensuremath{m_{\beta\beta}^t}}
\newcommand{\nbb}{\ensuremath{N_{\beta\beta^{0\nu}}}}

% Qbb
\newcommand{\Qbb}{\ensuremath{Q_{\beta\beta}}}

% Tonu
\newcommand{\Tonu}{\ensuremath{T_{1/2}^{0\nu}}}

% Tonu
\newcommand{\Ttnu}{\ensuremath{T_{1/2}^{2\nu}}}

% Xe-136
\newcommand{\Xe}{\ensuremath{^{136}}Xe}
\newcommand{\COT}{\ensuremath{CO_2}}
\newcommand{\CHF}{\ensuremath{CH_4}}
\newcommand{\CFF}{\ensuremath{CF_4}}

% 2S
\newcommand{\TwoS}{\ensuremath{^{2}S_{1/2}}}

\newcommand{\TwoP}{\ensuremath{^{2}P_{1/2}}}

\newcommand{\TwoD}{\ensuremath{^{2}D_{3/2}}}


% Xe-136
\newcommand{\CS}{\ensuremath{^{137}}Cs}

% Xe-136
\newcommand{\NA}{\ensuremath{^{22}}Na}


% Bi-214
\newcommand{\Bi}{\ensuremath{^{214}}Bi}

% Tl-208
\newcommand{\Tl}{\ensuremath{^{208}}Tl}

% Pb-208
\newcommand{\Pb}{\ensuremath{^{208}}Pb}
% Pb-208
\newcommand{\PBD}{\ensuremath{^{210}}Pb}

% Po-214
\newcommand{\Po}{\ensuremath{^{214}}Po}
\newcommand{\Kr}{\ensuremath{^{83}}Kr}

% bru
\newcommand{\bru}{cts/(keV$\cdot$kg$\cdot$y)}
\newcommand{\dten}{10 mm/$\sqrt{\rm m}$}
\newcommand{\dtwo}{2 mm/$\sqrt{\rm m}$}
\newcommand{\BAPP}{\ensuremath{Ba^{++}}}
\newcommand{\BAP}{\ensuremath{Ba^{+}}}

\newcommand{\HPXE}{\sc{HPXe}\rm}
\newcommand{\BATA}{\sc{BaTa}\rm}

% Saltos de carro en tablas
\newcommand{\minitab}[2][l]{\begin{tabular}{#1}#2\end{tabular}}

\newcommand{\thedraft}{0.1.1}% version for referees

\newcommand{\MO}{\ensuremath{{}^{100}{\rm Mo}}}
\newcommand{\SE}{\ensuremath{{}^{82}{\rm Se}}}
\newcommand{\ZR}{\ensuremath{{}^{96}{\rm Zr}}}
\newcommand{\KR}{\ensuremath{{}^{82}{\rm Kr}}}
\newcommand{\ND}{\ensuremath{{}^{150}{\rm Nd}}}
\newcommand{\XE}{\ensuremath{{}^{136}\rm Xe}}
\newcommand{\GE}{\ensuremath{{}^{76}\rm Ge}}
\newcommand{\GES}{\ensuremath{{}^{68}\rm Ge}}
\newcommand{\TE}{\ensuremath{{}^{128}\rm Te}}
\newcommand{\TEX}{\ensuremath{{}^{130}\rm Te}}
\newcommand{\TL}{\ensuremath{{}^{208}\rm{Tl}}}
\newcommand{\CA}{\ensuremath{{}^{48}\rm Ca}}
\newcommand{\CO}{\ensuremath{{}^{60}\rm Co}}
\newcommand{\PO}{\ensuremath{{}^{214\rm Po}}}
\newcommand{\U}{\ensuremath{{}^{235}\rm U}}
\newcommand{\CT}{\ensuremath{{}^{10}\rm C}}
\newcommand{\BE}{\ensuremath{{}^{11}\rm Be}}
\newcommand{\BO}{\ensuremath{{}^{8}\rm Be}}
\newcommand{\UDTO}{\ensuremath{{}^{238}\rm U}}
\newcommand{\CD}{\ensuremath{^{116}{\rm Cd}}}
\newcommand{\THO}{\ensuremath{{}^{232}{\rm Th}}}
\newcommand{\BI}{\ensuremath{{}^{214}}Bi}
\newcommand{\FDG}{\ensuremath{^{18}}F}


\begin{center}
{\LARGE \bf \textsf{CONSOLIDER-INGENIO 2010 PROGRAMME }} \\ \vspace{0.5cm}
{\Large \bf \textsf{FINAL SCIENTIFIC ANNUAL REPORT}} \\ \vspace{0.5cm}
{\Large \bf \textsf{I01/01/2012 al 31/12/2012 }} \\ \vspace{2cm}
\end{center}


\begin{table}[htp!]
\begin{center}
\begin{tabular}{|c|c|}
\hline
PROJECT REFERENCE NUMBER: & CDS2008-00037\\
\hline
Coordinating Researcher:  & M.C. G\'onzalez-Garc\'ia \\
Project Title: & Canfranc Underground Physics (CUP) \\
Managing Institution:  & CSIC \\
Project Start Date:  & 15/12/2008\\
Project Final Date: &  14/12/2014)\\
\hline\hline
\end{tabular}
\end{center}
\caption{Project data}
\label{default}
\end{table}%

\newpage

%%%%%%%%%%%%%%%%%%%%%%%%%%%%%%%%%%%%%%%%%%%%%%%%%%%%%%%%%
%% SECTION 1
\section{\bf \textsf{CUP: summary and main goals }}

CUP (Canfranc Underground Physics) was granted with the main goal of developing the scientific program of the
 \emph{Laboratorio Subterr�neo de Canfranc} (LSC), one of the ICTS (instalaci�n cient\'ifico t�cnica singular) of the ministry of science (currently integrated in the MINECO). The two main activities of CUP were called
 CAFE (Canfranc Future Experiment) and NEXT (Neutrino Experiment with a Xenon TPC).

CAFE has spawn a wide range of phenomenological studies, coordinated by professor 
M.C. Gonz\'alez-Garc\'ia. These studies have involved several institutions of the CUP consortium, including the Universidad de Barcelona (UB), Instituto de F�sica Corpuscular, (IFIC), and Instituto de F�sica Te�rica (IFT). 

The goals of the subproject CAFE were:
\begin{enumerate}
\item The characterisation of LSC to demonstrate its viability 
as a large underground European laboratory in the context of
the European project LAGUNA (Large Apparatus for Studying Grand
Unification and Particle Astrophysics) as well as in the context
of studies of viability for future neutrino factories which were
performed as part of the European project EURO-$\nu$
\item Perform phenomenological studies to explore and exploit 
the physics potential LSC  as well more broad studies related
to astroparticle physics in general (neutrino physics, cosmology, \dots)
\end{enumerate}

%The CAFE activities were amply described in our previous reports, and here we will simply update the lists of publications related with the activity. 

%Two major spinoffs of the activity are two European projects directly related with CAFE goals: 
%
%\begin{enumerate}
%\item 
%A european training network, called INVISIBLES, coordinates by professor B. Gavela, from IFT. 
%%
%\item A european project of large infrastructures called LAGNA-LBNO. The spanish IP of the project is  professor J.J. G�mez-Cadenas, for IFIC, who acts also as spokesperson of NEXT.
%\end{enumerate}

CUP main goal was the construction, commissioning and operation of the NEXT detector\footnote{\url{http://next.ific.uv.es/next/}}, a high-pressure, xenon (HPXe) Time Projection Chamber (TPC), whose goal was to search for neutrinoless double beta decay  (\bbonu) events in xenon enriched at 90\% in the isotope \XE. The first phase of the experiment, called NEW, deploying 10 kg of xenon, is currently being commissioned at the Canfranc Underground Laboratory (LSC).The second phase of the experiment, NEXT-100, will deploy 100 kg of xenon. A third phase, deploying up to one ton of xenon, is actively being discussed.
 
%
%The discovery potential of NEXT is very large. It combines four desirable features that make it an almost-ideal experiment for \bbonu\ searches, namely: 
%\begin{enumerate}
%\item Excellent energy resolution (better than 1\% FWHM in the region of interest).
%\item A topological signature (the observation of the tracks of the two electrons).
%\item A fully active, very radiopure apparatus of large mass.  
%\item The capability of extending the technology to much larger masses.
%\end{enumerate}
%
%The project has evolved very satisfactorily, from the initial Letter of Intent (LOI) in 2009 to the Technical Design Report (TDR) in 2012. A substantial number of papers, proceeding reports and conferences, documenting and demonstrating the physics case, the results of the prototypes and the technological choices have been published and are attached to this document. They can also be found in the NEXT web page: \url{http://next.ific.uv.es/next/talks.html}.
%


\section{\bf \textsf{Main advances and profits }}

CAFE has result into several advances
\begin{enumerate}
\item {\bf It has enhanced the visibility of the LSC} by participating into
European studies devoted to the long term planning for large infrastructures
in Europe such as of such as LAGUNA-LBNO and Euro-$\nu$.
\item {\bf tt has contributed to the  creation of a new generation of European physicists}
with expertness in phenomenology of relevance for underground physics
by our participation in two consecutive International Training Networks
and a Network for Research and Innovation Staff Exchange, all
funded by the European Commission, as reported below. 
\item  {\bf It has  produced a large number of scientific articles, which have
advanced quantitatively the state-of-the-art of the field.} The total
number of references of published articles by the CAFE senior members
in international peer reviewed journals 
since 2008 is 70  (listed below) that  have  received a total of 
more than 3000 citations in the SPIRES data basis which is the one commonly used
in our field for reference.
\end{enumerate}

NEXT has result in several advances:
\begin{enumerate}
\item {\bf It has advanced the field of underground physics instrumentation}, with the construction of the NEXT detector series. The large prototype which has operated at IFIC since 2011 (NEXT-DEMO) was the largest HPXe electroluminescent TPC in the World, until the construction of NEW, currently being commissioned at the LSC. The detectors have shown the feasibility to transform all the signals in xenon to VUV light, have demonstrated the excellent energy resolution that can be achieved in xenon gas, and have shown the feasibility to reconstruct the two electrons emitted in \bb\ decay (this is called the topological signature). 
\item {\bf It has resulted in a cutting-edge experiment}, which will be searching for \bbonu\ decays during the next few years, with chances of making or contributing to a major discovery. 
\item {\bf It has bought to the field a new technology}, that of the HPXe-EL, which is currently considered one of the candidates for the next generation of \bbonu\ experiments.
 \item {\bf It has resulted in an international collaboration}, involving groups from Portugal, United States, Russia and Colombia. Other groups (Poland, Australia), are currently considering joining the experiment.
 \item {\bf It has attracted external funding}, with contributions to the experiment from all the members of the collaboration.
  \item {\bf It has resulted in an Advanced Grant of the ERC}, granted to the spokesperson of NEXT, prof. J.J. G\'omez Cadenas, who has acted as co-PI of CUP. 
   \item {\bf It has boosted the scientific interest of the LSC}, as intended in our proposal. Currently NEXT is the most important experiment in the Canfranc Underground Laboratory and the only one with truly international projection. 
  \item {\bf It has produced a major scientific spinoff, called PETALO}, a new concept for a full-body, TOF-capable PET using xenon. 
   \item {\bf It has contributed in an important way to science outreach}. See for example:
   \url{http://www.jotdown.es/2012/09/david-nygren-y-alessandro-bettini-the-physics-as-fountain-of-eternal-youth/}
   http://www.jotdown.es/2013/11/coversaciones-de-fisica-en-el-santa-cristina-ariella-cattai-y-concha-gonzalez-garcia/
\end{enumerate}


\section{\bf \textsf{ Key Indicators}}

In the proposal, we stated:
\begin{quotation}
We propose a very simple and efficient evaluation scheme for CUP. The  activities are focused in the LSC, which has an international scientific committee, composed by high-reputation specialists in underground physics. This committee is perfectly suited to evaluate our progress in all the areas proposed.
\end{quotation}

and
\begin{quotation}
As for the CAFE activity, there is an obvious evaluation criteria, namely the scientific papers to be produced along the five years of the project, including the proceedings of the workshop(s) and/or schools as well as other outreach material.
\end{quotation}

The key indicators of the final results of CAFE are the following:

\begin{enumerate}
\item Involvement in the project LAGUNA-LBNO
(\url{http://laguna.ethz.ch:8080/Plone}). This project continued the
studied started by LAGUNA with the aim at developing large infrastructures
in Europe. Spain participated with four associates LAGUNA-LBO, three of
which are participants in CUP (UAM, IFIC y el propio LSC).
\item Involvement of our scientists in the european project EURO-$\nu$.
In particular de group at UAM made very relevant contributions to
this project as listed in the bibliography.
\item Participation in three funded projects of the European Commission
for the training of young researchers in fields directly related
with underground physics. They are: 
\begin{itemize}
\item 
Title: INVISIBLES\\
Initial year: 2012 Year final: 2016\\
Reference: PITN-GA-2011-289442\\
\item 
Title: 	Elusives\\
Initial year: 2016 Year final: 2020\\
Reference: H2020-MSCA-ITN-2015-674896 \\
\item  Title: 	InvisiblesPlus\\
Initial year: 2016 Year final: 2020\\
Reference: H2020-MSCA-RISE-2015-690575\\
\end{itemize}
The coordinator of the UB node of these networks is our IP, Prof.
M.C. Gonzalez-Garcia while the coordinator of the node at IFIC
is Prof. P. Hernandez, a member of CAFE/CUP.
\item Publication in peer review international journal 
of more than 70 articles in phenomenological studies of which 
we list a selection in the
bibliography. They have accumulated more than 3000 citation  in the
basis SPIRES. For example our updated determinations of the neutrino 
parameters, fundamental for any study of the expected signal at NEXT
\cite{Gonzalez-Garcia:2014bfa,GonzalezGarcia:2012sz,GonzalezGarcia:2010er} 
have accumulated more than 1000 citations,  
while our contributions to studies associated to future neutrino facilities,
such as Ref.~\cite{Bandyopadhyay:2007kx,Abazajian:2012ys,Choubey:2011zzq,Adey:2014rfv} have received more than 500 references (all data from SPIRES
Data basis).
\end{enumerate}



\section{\bf \textsf{ Difficulties}} 

The main difficulty the project has faced was related with the withdrawal, in 2011, of the IFAE and CIEMAT groups from the NEXT collaboration (although they did not formally leave the CUP consortium). The reasons argued for the withdrawal were related with different point of views concerning the technical choices implemented in NEXT, as well as disagreement with the leadership of the NEXT spokesperson.

A few months after the withdrawal of the IFAE and CIEMAT groups, the regular research project submitted by IFIC-UPV was marked with at C (mediocre) and rejected by the funding committee of particle physics (FPA). The external expert of that particular committee was prof. Giorgio Gratta (the spokesperson of the EXO experiment, which is the main competitor of NEXT) and the committee members included several leaders of the IFAE and CIEMAT group. The committee disregarded the reports of the LSC scientific committee and the opinion of the director of the LSC. The reports from the ANEP were requested and at least one of them was found to be extremely biased, with ad-hominem attacks and absolute lack of rigor, as acknowledged by the ANEP itself. However, the committee of particle physics did not reject that particular report.

The project submitted in 2011, requested essential additional funding, not covered by CUP. The negative result implied a major logistic drawback and a serious blow for the collaboration. In 2012, a new research project was submitted, to a different committee (general physics, FIS), given the obvious bias of the FPA committee. The project obtained a B and modest funding. In 2013, the spokesperson of NEXT was granted the first (and the only one, to date) advanced grant of the ERC in experimental particle physics in Spain. In 2014, another project was submitted to FIS, which obtained a top mark. 

The perception of the consortium and the international collaboration is that, in spite of the demonstrated excellence of the project (availed by the reports of the LSC scientific committee as well as by the excellent reports produced by external evaluators, in particular in the ERC), its capability to attract external funding, and its importance for the scientific case of the LSC, a bias still exist in FPA, mostly related with the opposition to the project of the IFAE group. Indeed, the ex-director of IFAE, Dr. Mateo Cavalli-Svorza has recently (in 2015) expressed publicly (in a seminar at the ALBA seminar) his enmity towards the spokesperson of the experiment. 

The situation is not fully solved yet, and poses a serious threat to the future of the collaboration, since external collaborators from other countries may feel that the project is not sufficiently supported in Spain. It also poses a serious threat for the future of the Canfranc laboratory, whose reputation would suffer very much should the NEXT experiment be forced to move to a different underground laboratory. 

The rejection of the research proposal in 2011 and the very scarce funding in 2012, forced the collaboration to redefine its initial plan to start NEXT-100 construction in 2012. Instead, it was decided to introduce a first-stage experiment, NEW (a detector with half of the size of NEXT-100, a fourth of the sensors and 10 kg of xenon mass). It was perceived that there was not sufficient support in Spain to attempt the construction of NEXT-100 directly. After the ERC grant the situation has improved considerably, but the collaboration decided to continue with the NEW program. Currently, NEW is starting data taking at the LSC, and the lessons learned in its construction (and certainly by operating it) are very valuable to build a better NEXT-100 detector. 


\section{\bf \textsf{ Scientific and technical activities }} 

5.-Describe the scientific and technical activities to reach the goals outlined in the project. Indicate, for each activity, the members of the equipment who have participated

The scientific activity of CAFE has been described above. We list
here the main lines developed in their phenomenological studies
which are directly related to underground physics
\begin{itemize}
\item Precise and updated characterization of the leptonic flavour
parameters, whose values are crucial to the prediction of expected
event rates in neutrinoless double beta decay experiments
\cite{Song:2015xaa,Bergstrom:2015rba,Gonzalez-Garcia:2014bfa,Gonzalez-Garcia:2013usa,GonzalezGarcia:2012sz,GonzalezGarcia:2011my,GonzalezGarcia:2010un,GonzalezGarcia:2010er}
\item Studies of new phenomena expected from presence of additional
sterile neutrinos and their effect at future underground experiments
\cite{Vincent:2014rja,Bergstrom:2014fqa,Adey:2014rfv,GonzalezGarcia:2012yq,SolagurenBeascoa:2012cz,Donini:2011jh,Abazajian:2012ys}
\item Studies directly related to future underground facilities
in Europe 
\cite{Bross:2013oua,Edgecock:2013lga,Blennow:2013swa,Bayes:2012ex,Agarwalla:2012zu,Hernandez:2012zz,Coloma:2012wq,Coloma:2011rq,Donini:2010xk,Coloma:2010wa,Choubey:2009ks,Bandyopadhyay:2007kx}
\item Potential of underground experiments to detect and 
characterize astrophysical neutrino sources 
\cite{Bergstrom:2016cbh,Gonzalez-Garcia:2013iha,Ahlers:2011jj,Ahlers:2010fw,GonzalezGarcia:2009ya,GonzalezGarcia:2009jc}
\item Study of theoretical frameworks for the connection  between Majorana 
nature of the neutrino (as searched for
in neutrinoless doublebeta decay) and the production of the
observed matter-antimater in the Universe
\cite{Hernandez:2015wna,Fong:2013gaa,Fong:2011yx,Fong:2010bv,Fong:2010bh,Fong:2010qh,Fong:2010zu,GonzalezGarcia:2009qd,Fong:2009iu}
\item Construction of models which can account for all neutrino observations
and study their specific predictions in future experiments
(including neutrinoless doublebeta decay experiments)
\cite{Gago:2015vma,Hernandez:2014fha,Hernandez:2013lza,Donini:2012tt,Eboli:2011ia,Gavela:2009cd}
\item Studies of scenarios of electroweak symmetry breaking related
for mass generation
\cite{Corbett:2015lfa,Corbett:2014ora,Gavela:2014vra,Brivio:2014pfa,Brivio:2013pma,Corbett:2013pja,Corbett:2012ja,Corbett:2012dm,Eboli:2011ye,Eboli:2011bq,Eboli:2010qd,Alves:2009aa}
\end{itemize}



\section{\bf \textsf{Collaborations between team members}}
6.-Relate the collaborations between all team members 


CAFE: As can be seen from our publication list there are regular
collaborations between the scientists in UB, IFIC and UAM. 
The members in the three groups in CAFE are all involved in the European
networks mentioned listed above. 


\section{\bf \textsf{Collaborations with other groups of research}}

7.-Relate the collaborations with other groups of research and the added value for the project. Describe, if necessary, the access to equipment or infrastructures of other groups or institutions

CAFE: We have ongoing collaborations with scientist in major institutions
and laboratories worldwide as can be deduced from our list
of publications. Just to mention a few: CERN, Fermilab (USA), 
INFN-Frascati (ITALY), Gran Sasso (Italy), Max Planck Institute (Germany),
Stony Brook Univ (USA), U. Wisconsin (USA), and, U. Sao Paulo (Brazil). 

\section{\bf \textsf{Collaborations with companies}}

8.-Relate the collaborations with companies or other socioeconomic sectors and the added value for the project, the knowledge transference or other results

\section{\bf \textsf{UE FP7 calls}}

9.-Indicate if you have attended, and how effective has been, to any of UE FP7 calls (project, training, infrastructures) and/or to other international programs in topics related to this project. Specify the program, partners, countries and topics and, where appropriate, funding received

\section{\bf \textsf{National and international Program visibility}}

10.-National and international Program visibility

CAFE: The members of CAFE been regularly reporting the results of their 
studies in  conferences and workshops. In particular they have been invited to
be review talks on neutrino physics in the most important conference
in the area. For example:
\begin{itemize}
\item Prof. M.C. Gonzalez-Garcia gave the plenary talk on Theory of Neutrino
Physics in the ICHEP Conference in Melbourne in July 2012
\item Prof. P. Hernandez gave the plenary talk on Theory of Neutrino Physics
at the HEP-EPS Conference in Vienna in July 2015 
\item Prof. M.C. Gonzalez-Garcia gave the plenary talk on Neutrino Physics
at the  Astroparticle Physics Joint  TeVPA/IDM Conference, Amsterdam, 
Netherlands in June 2014
\end{itemize}


\section{\bf \textsf{Problems and suggestions}}

11.- Problems and suggestions
\bibliographystyle{JHEP}
\bibliography{biblio}
\end{document}

