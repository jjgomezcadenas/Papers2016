\subsubsection*{2014-May}
\begin{quotation}
LSC Scientific Committee \\
14th Meeting\\
May 29-30, 2014\\

EXP-05 (NEXT)\\

{\em The committee wants to congratulate the NEXT Collaboration for the continuous and steady progresses that strengthen the soundness of the NEXT concep}t. Among the excellent work produced, in many cases with support from the LSC staff, the points listed below were particularly appreciated:

{\bf Infrastructure}:
\begin{itemize}
\item The completion of the castle structure.
\item The procurement of all the lead needed to construct the shield (it comes from OPERA as a loan for the NEXT lifetime)\footnote{This loan has to be read as an additional external support to NEXT and the LSC, and its credit rests mainly in the personal prestige of the then-director, Alessandro Bettini.}. 
\item The plan to clean the 60 t of lead in an industrial way on the platform.
\item The progresses in construction and understanding of the mechanics of the NEW detector.
\item The detailed design and construction of the pressure vessel and the cleaning protocol adopted.
\item The new solutions adopted for the mechanics of the Energy Plane.
\item The use of an appropriate optical gel that couples effectively the sapphire window to the quartz window of the PMT; the response has increased by a factor 2.
\item The high vacuum quality tested, using RGA.
\item Operation with argon up to 20 bar pressure with a leak rate as low as 0.01 g/y (expected to diminish with Xe).
\item The new solutions adopted for Tracking Plane.
\item The successful New Kapton pig-tail Design.
\item The extensive investigations pursued to optimize the feed-through.
\item The completion of the simulation of the electric field cage (COMSOL Multi physics) with which it is possible to study in details different regions of the cage.
\item The large campaign of radio-purity measurements; all components are carefully measured to assure that any contaminated element is introduced in the detector. Actually some components were eliminated from the production chain.
\item The progress in understanding the functioning and the performance of the detector.
\item The careful analysis of the background induced by different radio-elements\footnote{see 
\cite{Martin-Albo:2015rhw}}.
\item The study performed with point-deposited energy from X-ray: drift velocity studies, calibration of PMT, study of energy resolution \footnote{See \cite{Lorca:2014sra}}.
\item The successful detection of the electron tracks at the double escape peak generated by a Thorium source  \footnote{See \cite{Ferrario:2015kta}}. 
\item The extensive studies of TMA as quenching gas – More investigations are needed at this point; in particular to prove the compatibility of TMA with the materials that constitute the detector\footnote{See \cite{Alvarez:2013oha,Alvarez:2013kqa}}.
\end{itemize}
	

Together with many achievements, some concerns emerged during the review.
	 
The collaboration has identified the need to control radon in the detector environment. It should present a study of the sensitivity of the neutrino-less decay measurement in NEXT-100, to the level of external radon as a metric of the actual reduction required. They should then consider technical implementations such as a radon reduced air or nitrogen environment for the detector to achieve the required radon reduction\footnote{Currently a radon reduction abatement system has been purchased by the LSC and will be used to provide radon-free air to the NEXT experimental area}. 
The radon emanation from the pressure vessel and detector materials has not been presented. The committee recommends that this be investigated prior to the next meeting as this can be a significant source of background\footnote{Radon emanation measurements are under way in Cracow, Poland.}.
	 
The committee welcomes the aggressive schedule leading to installation of the NEW detector at LSC at the beginning of next year. However, the group needs to ensure that adequate time is allowed for testing of the new technologies being employed. The committee was particularly concerned about the high voltage structure and the sensitivity of the new anode structure to possible breakdown. The new anode structure in particular is expected to be very sensitive to electrical discharge and steps are likely required to ensure that breakdown does not occur, even at a low rate.

The committee was very pleased to learn that the collaboration has all funds required to complete the NEW phase of the project. This will provide an excellent proof of concept for the xenon gas detector. However, the committee, the project, and the laboratory all concur that the full potential of the project will only come from the construction of the full NEXT-100 detector. The committee is pleased to learn that there are opportunities to apply for further support to move ahead with the 100 kg detector. The committee strongly supports such a development and re-iterates its endorsement of the physics potential of the 100 kg gas detector as being a major contribution to the global search for neutrino-less double beta decay.


\end{quotation}

\paragraph{Summary-May 2014:} In May 2014, progress was steady solving technical issues and producing scientific results. The remarks of the committee were very relevant, and led to major improvements in the detector planning and procedures, in particular concerning the EL structure and the campaign to understand the effect of radon. In 2014, the leading groups of CUP/NEXT presented a join proposal for funding, which was evaluated with top ranks and granted. However the funding level was about half of the requested amount (due as limited budget, as explicitly stated in the resolution).  See section \ref{sec.diff}.

\subsubsection*{2014-November}
\begin{quotation}
LSC Scientific Committee \\
15th Meeting \\
November 20,21, 2014\\

EXP-05 (NEXT)\\

{\em The committee congratulates the NEXT Collaboration for the continuous achievements and the impressive progresses on the construction of the first detector demonstrator and the understanding of its basic performances.} Among the excellent work produced, the points listed below were particularly appreciated:

{\bf Infrastructure}:
 \begin{itemize}
\item The completion and installation of the platform and castle structures
\item The completion of the gas system.
\end{itemize}
	 
 
{\bf The progresses of the NEW detector }
  \begin{itemize}
\item The completion of the pressure vessel at the factory 
\item The production of the Energy Plane and the assembly test of the sapphire window (including the use of brass screws)
\item The strategy adopted for cleaning, assembly and test the vessel and the energy plane.
\item The performances of the new SiPMs produced by SensL (better radio-purity, gain, temperature dependence and dark current wrt the Hamamatsu one).
\item The solutions adopted for Tracking Plane
\item The potting solution adopted for the feed-through 
\item The decision of constructing a full mock-up of the Tracking Plane.
\item The quartz anode plane (featuring the Dark Side approach) is ordered; a series of groves is foreseen on the mechanical structure in order to limit the effect of the discharges.
\item The clear experimental reconstruction in the NEXT-DEMO prototype of two-electron tracks corresponding to pair production with a 232Th source, requested by the Committee one year ago and now successfully achieved\footnote{ See \cite{Ferrario:2015kta}}.
\item The progress in understanding the functioning and the performances of the detector and the excellent results obtain on radio-purity measurements and background suppression. In particular, the Committee has appreciated the detailed evaluation of the contributions to the final NEXT-100 background in the Region of Interest (ROI) from the various detector elements, based on an intensive campaign of specific radioactivity measurements. 
\item The very convincing studies of Radon contamination that conclude that the rate can be suppressed by 3 orders of magnitude provided that a decontaminated atmosphere surrounds the detector. The Committee would like to stress that for the first time the Radon issue has been approached with a fully quantitative analysis, establishing precise correlations (and therefore well-defined targets) between the 222Rn activity level outside and inside the detector and the background in the ROI for NEXT-100.
\end{itemize}

{\em The scientific committee acknowledges the large progresses made by the NEXT experiment: the new design and hardware solutions presented, the tests and the simulation performed.  The reviewers did not identify any substantial problem and therefore encourage the NEXT group to proceed at full speed to the construction of the NEW detector}.  The team should work with the laboratory to ensure that any safety requirements are understood and planned for. The team should also understand that laboratory approval will be required prior to use of the enriched isotope. This approval will require a clear scientific justification as well as a demonstration of technical readiness.

\end{quotation}

\paragraph{Summary-November 2014:} The November 2014 meeting was characterized by steady progress, and a sense of optimism related with better funding prospects. 