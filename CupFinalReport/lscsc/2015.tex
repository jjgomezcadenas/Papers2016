\subsubsection*{2015-April}
\begin{quotation}

LSC Scientific Committee \\
16th Meeting\\
April 20-22, 2015\\
EXP-05  NEXT  \\

{\em The committee wants to congratulate the NEXT Collaboration for the continuous progresses and the impressive achievements on the construction of the first detector demonstrator}. Among the excellent work produced, the points listed below were particularly appreciated: the pressure vessel and inner copper shield have been installed and cleaned; the gas purification and recovery system is well advanced; the read out electronics for the energy plane is installed and the complete assembly of this system is progressing well; and progress is being made towards completion of the tracking plane.

Despite this excellent technical progress, there are two areas of concern. 

A first risk analysis has recently been provided to the laboratory. A dialogue between the LSC and the collaboration has started with the aim of bringing this to a level acceptable to the laboratory but the experiment might expedite this process by getting input from a consultant with experience in producing such analyses for underground physics experiments.

A second concern is that it has been realized that the pressure vessel for the experiment has not been certified as required under EU law. The experiment will investigate how this certification might be obtained\footnote{The PV has been successfully certified.}.

In order to help prevent this type of problem in the future, it is recommended that the LSC consider producing a document listing all approvals required for equipment to be located in the laboratory. While production of such a document is labour intensive, the committee suggests that it may be possible to obtain such a document from another European underground laboratory at least to use as a starting point.

\end{quotation}
\paragraph{Summary-May 2015:} The May 2015 witnessed steady progress. Several issues with safety certification were identified, highlighting that the safety and risk control procedures of the LSC were not yet fully spelled out. During 2015 and the first half of 2016, much work has been invested in defining such procedures and carrying out the needed risks analysis. 

subsubsection*{2016-November}
\begin{quotation}
LSC Scientific Committee\\
17th Meeting\\
November 5-4, 2015\\
EXP-05 NEXT
{\em 
The committee congratulates the NEXT Collaboration for the continuous progresses and the impressive achievements on the detector, electronics and Infrastructures.}

Among the excellent work produced, the points listed below were particularly appreciated:
\begin{itemize}
\item The aggressive New schedule and installation campaigns that will allow  to have, by the end of 2015
\begin{itemize}
\item Integration and cabling of the front-end electronics, commissioning of DAQ, and consequent tests of the tracking plane sensors\footnote{Achieved.}.
\item An operative gas system\footnote{Achieved.}.
\end{itemize}
\item The aggressive road-map to winter 2016 that will permit:
\begin{itemize}
\item The installation of the gas system and commissioning of the slow control\footnote{Achieved.}.
\item The installation of the field cage\footnote{Scheduled for April, 2016.}.
\item Commissioning of the detector by March 2016 and initial results by the end of the year \footnote{The commissioning of the detector started in March, as schedule and initial results are expected by May, 2016.}
\end{itemize}
\item The effort of presenting an updated TDR of NEXT-100 by November-2016 meeting. If so, the construction of major parts of NEXT-100 could start in 2017.  In this framework it was appreciated that no changes are foreseen for the FEE and DAQ.
\item The effort invested in improving the infrastructures that will be fully commissioned by end of the year (in order to reduce the background, the collaboration plan to construct a Cu shielding hub).
\item Progresses of the NEW detector: 
\begin{itemize}
\item NEW vessel: it has been certified for operation with pressure and it is installed on its platform enclosed by a fully-functional lead castle.
\item Energy Plane (EP): is completed and was successfully tested up to 15 bar!   Only one out of the 12 PMT broke and will be investigated in the incoming future\footnote{PMT replaced, energy plane fully operational.}.
\item Commissioning of the detector by March 2016 and initial results by the end of the year \footnote{The commissioning of the detector started in March, as schedule and initial results are expected by May, 2016.}
\item Tracking plane (TP):  all 33 Dice Boards are tested and ready to go with all the necessary services (just 1 SiPM out of 2000 failed). Noticeable that all feed-through were tested and are qualified\footnote{Tracking plane already installed in detector.}
\end{itemize}
\item FE electronic: all ready and tested.
\item Read-out and DAQ electronic: all ready and tested.
\item Gas system: all parts are in hand, will be assembled, will undergo internal and external refereeing and will be certified by Cryvoc by February 2016. A risk analysis will be performed afterwards.
\item  Gas system safety and control protocol: A very detailed and complete  protocol was presented; it will  allow monitoring  all relevant parameters  that can jeopardize the good functioning of the system and will take immediate safety  actions.
\end{itemize}


Together with many achievements, some concerns emerged during the review. The reviewers solicit clarifications, or follow-up, on the items listed below:
\begin{itemize}
\item NEW requires the construction of a sealed containment around the NEW detector shield fed with radon reduced air. Details of this system need to be worked out with the Lab management\footnote{The problem is solved with the new radon abatement system, that allows pumping radon-clean air into the NEW experimental area.}.
\item The gas Slow Control system needs to be finalized and commissioned\footnote{Slow Control finalized, commissioning in progress. }
\item The HVFT just arrived from Cryvac; it is not up to specifications and has to be returned to the company. A second piece will arrive next week but the break-down tests have still to be performed and the connector has still to be qualified\footnote{Two new HVFT have been manufactured. Testing under way. }
\item Computing and infrastructure: the radio link of 32Mb/sec is insufficient to send data to IFIC and should be improved. This item might take time since it depends on the Jaca region and not only on LSC. A back-up solution needs to be put in place.
\end{itemize}

{\bf
The scientific committee acknowledges the large efforts and huge progresses made by the NEXT Collaboration in the assembly of NEXT-NEW a flagship project of the laboratory.}

The schedule for bringing this project into operation is aggressive and to be successful, it needs to be carefully coordinated with the laboratory. The committee urges the experiment to provide a detailed plan to the laboratory, including milestones at which the safety analyses will be available for the lab’s review. Adequate time for such reviews must be included in the plan\footnote{Two such reviews have been carried out during 2016.}.

\end{quotation}
\paragraph{Summary-November 2015:} The November 2015 was also characterized by steady progress. The recommendations of the committee have been followed, in particular in the establishment of a safety protocol, risk analysis and slow control, of which have been major tasks whose importance was no fully perceived in the initial stages of the project. In that sense, the NEW stage is proving extremely successful, since it allows us to work out the safety, controls and infrastructures for the full experiment. 


