\subsubsection*{2012-May}
\begin{quotation}
LSC Scientific Committee \\
10th Meeting\\
May 9th and 10th , 2012\\
EXP-05 (NEXT)\\

{\em The NEXT Collaboration presented enormous progresses both on technical and on managerial domains}. We appreciated, in particular, the points listed below:
\begin{itemize}
\item The experiment presented an excellent quality assurance plan for qualifying the Sapphire windows, a very delicate part of the experiment. The effort to guarantee the soundness of the windows is very convincing. In addition NEXT made large and constructive effort to work together with manufacturers to assure the quality of the windows when they were at the level of raw material before purchasing them.
\item The NEXT spokesman presented a very convincing and systematic study of the homogeneity of the TPB coating. The quality of the coating depends on the exposure time and consequently on the thickness, and the spinning velocity. Homogeneity was studied using glass plates with the same dimensions (30??30??3 mm3) of the final one. They were coated with different TPB thicknesses (0.6, 0.2, 0.1 and 0.05 mg/cm2). Since the study was to reproduce the final set-up, we are confident that the positive results obtained by the Collaboration reflect those that will be obtained in final production \footnote{See  \cite{Alvarez:2012ub}}.
\item The long-term stability of the TPB depositions is an important issue for NEXT. They realized that poor storage of TBP could result in very poor quality of the deposition layer and therefore jeopardize the detector. They have a good and safe plan to avoid this potential problem.
\item NEXT conducted very systematic aging studies on 14 SiPM after a storage time of 9 months in a controlled environment. They successfully demonstrated that the relative current variation of the 14 SiPMs is less than 1\%, which is well within the experimental uncertainties. 
\item The Collaboration developed a very successful protocol for coating the SiPMs with TPB used as WLS. In particular the precautions adopted for obtaining clean and uniform coatings allowed them to obtain optimal fluorescence 
efficiency \footnote{See  \cite{Alvarez:2012ub}}. 
\item We acknowledge the impressive work devoted to estimating the radioactivity of all materials that constitute the energy plane system \footnote{See, in particular \cite{Cebrian:2015jna}}.
\item The Next-DEMO prototype was successfully and systematically studies with a Na22 source for many months. The results obtained with the prototype reinforce the soundness of the experiments \footnote{ See  \cite{Alvarez:2012yxw, Alvarez:2012zsz,Alvarez:2012hu,Alvarez:2013gxa,Lorca:2014sra,Renner:2014mha,Serra:2014zda, Ferrario:2015kta}}.
\item The NEXT Collaboration demonstrated that Teflon coating of all surfaces enormously augmented the light collection. They can therefore trigger on only S1 signal! Unfortunately, the life-time of electrons is very small, which may be due to some water contamination that occurred during the coating process. We encourage the Collaboration to follow this point and report on it at the next LSC meeting.
\end{itemize}

We acknowledge the efforts and progress that the Collaboration did progressing toward NEXT-100, namely:
\begin{itemize}
\item The pressure vessel is steel. NEXT measured the activation of the material and it is well below the expected limit. We are very pleased to hear that the Collaboration has already procured this high quality material. 
\item We acknowledge that the order for the procurement of all the PMTs for the energy plane has been completed.
\item We appreciate the enormous effort made by the Collaboration to understand the activation of each component in the tracking plane and the radioactive screening that they employ for material selection.
\item The Collaboration has created a very efficient QA test to check and screen the dice boards that carry the SiPMs.
\item Amplification region - Work is steadily and surely going on! NEXT will measure the deflection of the meshes for the large extraction grids. This is quite a delicate problem but they have clear indication that the deformation can be corrected via software. If not, in the case that the deformations are too large, a solution has to be found. We encourage the Collaboration to follow this point and report on it at the next LSC meeting.
\item Electronics for SiPM - We note that in the prototypes the electronics were positioned outside the volume and that the power consumption was 600 mW/ch. Now the electronics has been modified reaching an excellent level of power consumption (30 mW/ch). NEXT plans to place the electronics inside the volume and shield it with copper. The feasibility of this solution has to be demonstrated.  We encourage the Collaboration to follow this point and report on it at the next LSC meeting \footnote{After following the committee recommendations it was decided to place the electronics outside the detector.}.
\item We are pleased to see that seismic calculations were done for the platform. The Collaboration plans to have them verified by an external review company. We acknowledge the attention the Collaboration is paying with respect to the safety aspects of the experiment.
\item The lead castle is ready for reviewing and therefore ready for construction.?
\item Radiopurity: we are impressed by the screening and measurement campaign put in place by the Collaboration.
\item Managerial aspect: We congratulate the Collaboration for presenting a very convincing work-breakdown structure with convincing work packages (WP).
\end{itemize}
	
{\bf
The Collaboration made extraordinary progresses both on the technical aspects and as well on the managerial parts. We congratulate the Collaboration for these major steps forward of the project, and recommend that they continue on their present energetic path.}

\end{quotation}

\paragraph{Summary-May 2012:} In May 2012 the collaboration made enormous progress in convincing the
LSCSC that the technical and scientific choices made were sound, problems were being understood, progress was 
steady and managerial aspects were correct. However, the collaboration was traversing a difficult period, due to the
lack of support of the particle physics manager (``gestor''). See section \ref{sec.diff}.

\subsubsection*{2012-November}
\begin{quotation}
LSC Scientific Committee \\
11th Meeting \\
November 15th and 16th, 2012\\
EXP-05 (NEXT)\\

{\em
The committee is very pleased to learn of the continued rapid and comprehensive development of the NEXT concept and the technical demonstrations of performance. }

In particular the committee appreciates the enormous technical progresses on
\begin{itemize}
\item The start of construction of the pressure vessel\footnote{Currently completed.}.
\item The construction of the detector platform in the underground area\footnote{Currently completed.}
\item The completion of the baseline gas system already installed at the Laboratory\footnote{Full gas system, suitable for all the stages of the NEXT experiment currently completed.}
\item The procurement of the MPPC and PMT sensors.
\item The large screening campaign brought forward to guarantee that all the part of the experiment to be installed inside the vessel are radio-pure \footnote{See \cite{Cebrian:2015jna}}.
\item The enormous effort invested on the data reconstruction algorithm that showed an improvement of a factor 3 in the background rejection  \footnote{See \cite{Martin-Albo:2015rhw}}.
\end{itemize}

	
It is particularly encouraging to see that the energy resolution may really approach the Fano factor limit \footnote{See \cite{Alvarez:2012yxw}}. {\em This is a great advance in the development of a xenon-based detector for neutrino-less double beta decay}. The demonstration of diffusion well below the predictions of the simulations is also an important step for precision tracking. The group continues to set a high standard for the documentation for each component of the system. 

The committee is pleased to see that, although there are parts of the experiment not yet in hand, with the present equipment the collaboration can realize  a simplified baseline detector with which many basic features of the experiment can be investigated and understood.

{\em The committee was concerned to learn of the very tight financial situation that the project is in}\footnote{The collaboration was traversing a difficult period, after our 2011 proposal for funding, needed to complete the construction of the experiment and to keep the team of physicists and engineers at the various national institutions involved, was rejected. See section \ref{sec.diff}}. It is very encouraged to learn that in the US, the Fermi National Accelerator Laboratory and Argonne National Laboratory will join the Lawrence Berkeley National Laboratory in requesting support for the project from DOE to provide vital parts of the final detector, namely: the energy plane, the field cage and the power supplies. The committee looks forward for a positive feedback from the DOE expected by July of next year\footnote{NEXT has received sizeable support from the US, in particular from the group of prof. D. Nygren. Currently, the US group are preparing a proposal to further contribute to the experiment.}

We are also encouraged to learn of the move to make the project an recognised CERN experiment\footnote{NEXT is now a CERN recognised experiment}. This move, if successful, not only could bring additional financial and manpower resources to the project but would also increase the international profile of the work. {\em This may help to improve the domestic attitude towards the program. }

The decision of the collaboration to proceed as quickly as possible to a technical demonstration of the strength of detector concept and to add those features that may prove essential to the lowest background later is strongly supported. The precarious state of support for the Spanish team is of great concern as the project moves towards implementation.{\em  It would be a serious loss to Spanish science if the financial constraints lead to a strong domination by its international partners in the exploitation of this experiment}.

{\bf
The Collaboration made extraordinary progresses in solving technical issues as well as expanding the U.S. participation. We congratulate the Collaboration for these major steps forward of the project, and recommend that they continue on their present energetic path. 
}

\end{quotation}

\paragraph{Summary-November 2012:} In November 2012, the LSCSC was satisfied with the progress made by 
NEXT while at the same time seriously worried by the lack of domestic support. See section \ref{sec.diff}.

