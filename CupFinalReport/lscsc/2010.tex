\subsubsection*{2010-April}
\begin{quotation}
LSC Scientific Committee\\
6th Meeting\\
April 15 and 16, 2010\\
EXP-05-2008 (NEXT)\\

The   Committee   {\bf commends   the   Collaboration   for   its   effort   to   improve   its organization, working relationship among members, and its project management. The establishment of a Steering Committee to choose the readout system technology was a positive step. The effort to involve foreign institutions is also laudable. NEXT is making excellent progress;} however the Committee has some recommendations.

The Committee recommends:
\begin{itemize}
\item that the collaboration finalize the selection of a technology choice, a plan of action, and write a schedule,
\item  that the Collaboration design a prototype to bridge from NEXT-1 to NEXT-100, with long drift distances matching those of NEXT-100,
\item that the Collaboration have a final readout system operating under 10 Atm of Xe gas,
that  the  Collaboration  write  a  detailed  plan  for  radioactive  screening  of construction materials
\item  that the Collaboration establish a presence at the Laboratory to determine their needs for counting, etc.,
\item that  the  Collaboration make a study  of  the safety  issues  with  10  Atm.  gas chambers underground, and begin discussions with the Laboratory personnel.
\item that the Collaboration write a Project Management Plan including a budget, schedule and names of persons responsible for specific tasks,
\end{itemize}
\end{quotation}

\subsubsection*{2010-October}
\begin{quotation}
LSC Scientific Committee\\
7th Meeting\\
October 7th and 8th, 2010\\
EXP-05-2008 (NEXT)\\

The Committee applauds the steps taken to further develop the strength of the collaboration, with an increasing commitment of the US groups and the enlargement to groups of the Universidad de Aveiro in Portugal and of the Universidad Antonio Nari\~no in Columbia.

The Committee was pleased by the informative presentations and the new format of the progress report. In general the project is making steady progress in its R\&D phase of the experiment. The most critical topic presented is the choice of the readout technology. The Committee observes that the Micro-Megas readout was tested at the Universities of Saragossa and Coimbra, and achieved an energy resolution of about 3\% at 10-atm pressure to be compared to the needed one, which is better than 1\%. The groups at LBNL, TAMU, IFIC, UPV, Nari\~no, Coimbra and CIEMAT are working on electroluminescence (EL) read out, which, however, has not yet demonstrated full feasibility. The Committee also observes that the Collaboration is facing the problem of non availability of radio-pure PMTs, which can take 10-15 bar pressure, by encasing them into quartz tubes. This feature allows to add a light guide function helping in the light collection. The Collaboration also presented an approach to improving light sensitivity of the SiPMs. 

{\bf
The Committee considers positively the NEXT Collaboration's reply to the recommendation made in its 6th meeting on the choice of readout technology, in particular the plan to present the Conceptual Design Report (CDR) in time for the Committee Meeting in the Spring of 2011. In particular, the Committee understands that a procedure to make the decision on the readout technology will be defined by the Collaboration in advance of that date. The decision should be based on the all available information at that time, and should take into account that experiments of similar sensitivity are already in more advanced stages of readiness.}
 
 The Committee recommended at the last meeting, and still does, that an intermediate detector be built to test the many new elements in the NEXT-100, for example the 10-atm pressure of Xe gas and the long drift distances/long drift times, as well as the readout system operation under these conditions. The Committee also recommends further study of the radon background and radio-purity measurements on construction materials to mitigate against excess background from radioactivity.

The Committee also welcomes the fact that the Collaboration is developing its Work Plan with milestones that will be integrated into the project management plan and looks forward to receiving the project management plan and the CDR before the next meeting. The Committee applauds the Collaboration for progress report with a much-improved format, but which still lacks budget and schedule information
\end{quotation}

\paragraph{Summary-2010:} The most critical issue identified by the committee in 2010 was the need to choose a technology for NEXT. As described in previous reports, such a choice was one of the first goals of CUP. At the time there were two possibilities. a) A readout based in Micromegas, pioneered by the group of Zaragoza, and b) an electroluminescent readout, using only light signals, and instrumenting the detector with SiPMs and PMTs. In 2010, the committee observed that the Micromegas technology did not meet the energy resolution requirements for the experiment, but also noticed that the EL option had not shown full feasibility. The committee recommended the construction of an intermediate prototype. Notice that none of the three NEXT prototypes (NEXT-DEMO at IFIC, NEXT-DBDM at LBNL, and NEXT-MM at Zaragoza) were yet fully operational, and the available results were based mostly in small prototypes at IFIC, LBNL and Zaragoza. 

