\subsubsection*{2013-May}
\begin{quotation}
LSC Scientific Committee \\
12th Meeting \\
May 28th and 29th, 2013\\
EXP-05 (NEXT)\\

{\em The Committee congratulates the NEXT Collaboration for the continuous and steady progress that strengthens the soundness of the NEXT concept}. In particular, NEXT has made numerous technical advancements with the NEXT-Demonstrator that has been running for 6 months\footnote{Refers to NEXT-DEMO. See \cite{Alvarez:2012yxw, Alvarez:2012zsz,Alvarez:2012hu,Alvarez:2013gxa,Lorca:2014sra,Renner:2014mha,Serra:2014zda,Ferrario:2015kta} }. The entire system is remotely controlled. The following important milestones have been met: 

\begin{itemize}
\item The system is very stable and typically runs for 8 days without errors. 
\item A very efficient calibration system for the Silicon Photomultipliers (SiPM), was set-up based on X-rays; excellent energy resolution was measured with radioactive sources of 22Na and 137Cs.
\item A very impressive demonstration was given of the reconstruction of three-dimensional (3D) tracks with the new SiPM tracking plane \footnote{See \cite{Ferrario:2015kta} }.
\item 
\end{itemize}
	 
{\em The NEXT project shows impressive progress on the mechanical structure as well as the general infrastructure}, namely:
\begin{itemize}
\item The working platform is completely installed.
\item The seismic structure and the castle are ready to be built and the tenders are ready to be issued.
\item  The pressure vessel will be ready by the end of June 2013.
\end{itemize}

{\em	
Much progress has been made on the assembly of NEXT-100}, namely:
\begin{itemize}
\item A new design for the PMT housing was completed that will make them gas tight.  
\item Two approaches are being studied and tested to guarantee that the SiPM tracking plane feed-troughs will be gas tight.
\end{itemize}

	
The NEXT Collaboration clearly demonstrated that they have the issues related to the SiPM base radio-purity under control. The problems were traced to the PCB board, the resistors and the capacitors.
Some possible Improvements to the tracking plane are being evaluated and will be followed up. The Collaboration intends to:
\begin{itemize}
\item improve the radio-purity of the adhesive of dice-board that is composed of 4 layers glued together,
\item develop a new soldering procedure for the  SiPM and the cables.
\end{itemize}

	
Overall, the NEXT team has made a deep and systematic study of all services and connections associated with the tracking plane. The new SiPM electronic system is much improved; it lowers power consumption and cost, and is considered ready for production.

A great deal of effort has been made to keep the radio-purity under control by screening every element installed in the detector. These steps have achieved a projected background level of $\mathrm{(4-9) x 10^{-4} counts \cdot keV \cdot kg \cdot yr}$. This value is totally dominated by parts that will be replaced with those with lower background, and therefore the overall background should be reduced by at least a factor three. With these values, NEXT will be very competitive with other experiments.

From the technical point of view, the construction of the NEXT-100 detector could proceed during 2014; {\em however, as of this writing there is not enough money to pay for the full detector. In addition, there are severe funding uncertainties. The committee is deeply concerned about the lack of adequate funding for this flagship project of the Laboratory}.

{\bf
	The Collaboration has made extraordinary progresses in solving technical issues; however, the lack of funding is of great concern. The Committee recommends that the Collaboration continue and even increase its efforts to recruit university collaborators in Europe and in the US. In any case it would be highly desirable for NEXT to move its upcoming prototype to LSC for operation.
}
\end{quotation}

\paragraph{Summary-May 2013:} In May 2013, progress was steady solving technical issues and producing scientific results. However, funding uncertainties due to the lack of support in FPA were holding back the project. See section \ref{sec.diff}.

\subsubsection*{2013-November}
\begin{quotation}
LSC Scientific Committee \\
113th Meeting\\
November 28-29, 2013\\

EXP-05 (NEXT)\\

{\em The committee congratulates the NEXT collaboration for the significant steps forward towards the construction of a fully-funded large-scale apparatus (named NEW)}, capable to validate the experiment concept in time for a feedback on the construction procedures of NEXT-100. It is also happy to acknowledge that crucial components of NEXT-100 setup are now funded and under construction.

The small scale prototypes developed so far have demonstrated the good performance of the electroluminescence technology in terms of energy resolution. The topological signature of electrons has also been proved, even though a two-electron track has not been observed yet. The committee invites the NEXT collaboration to verify if this can be achieved with a proper choice of the calibration source in the NEXT-DEMO prototype, exploiting in particular electron-positron pair production in the gaseous target\footnote{The collaboration followed the recommendation, and two-electron tracks were measured, see \cite{Ferrario:2015kta}}.

The committee remarks however that the present understanding of the background in the NEW project is at a very preliminary stage. It recommends therefore that the NEXT collaboration present a detailed Monte Carlo-based evaluation of the NEW background. In particular, the committee asks the collaboration to identify clearly the dominant contributions to the background in the region of interest for neutrino-less double beta decay, with a special focus on the role of 214Bi, which emits with gamma rays whose energy is very close to the 136Xe Q-value. In this context, a detailed analysis of possible 222Rn contribution is mandatory and an evaluation of the radon emanation from the tank walls should be scheduled in the near future\footnote{The collaboration followed the recommendation and a full
background model has been produced, see \cite{Martin-Albo:2015rhw}}.

{\em The committee congratulates the collaboration on securing new funding which it understands will fully support the NEW project}. The committee would like to receive at its next meeting, an estimate of the additional funding required to complete the full NEXT-100 project.
\end{quotation}

\paragraph{Summary-November 2013:} In November 2013 the committee learned that Gomez-Cadenas had obtained and Advanced Grant of the ERC. At the time, the experiment has received clear support from the
Secretary of State for Science (SEIDI) which unblocked the difficult situation of 2011 and 2012. However, the uncertain funding situation and the share complexity of the project had convinced the collaboration to stage the construction of the NEXT-100 detector in two phases. The first phase, called NEW is a 1:2 replica of NEXT-100 (half the longitudinal dimensions, a fourth of the sensors and a 10th of the mass), which is large enough to fully understand technical issues as well as to refine the background model. The collaboration has focused all its efforts in NEW, and the detector is currently being commissioned at the LSC. 
