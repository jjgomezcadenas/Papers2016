\subsubsection*{2011-May}
\begin{quotation}
LSC Scientific Committee \\
8th Meeting \\
May 26th and 27th, 2011 \\
EXP-05 (NEXT)\\

{\em The committee congratulates the NEXT collaboration on very substantial progress made towards the definition of the conceptual design for the xenon double beta decay experiment}. A major decision has been the choice of electroluminescence as the gain mechanism. The tracking readout would use Si-multi-pixel photon counters coated with TBP with devices on a 1 cm grid. The energy readout would employ an array of PMT’s on the opposite side of the detector from the tracking readout. The chamber would operate at 15 bar. The team has demonstrated that the preferred PMT’s would not survive at such pressures and the design has them housed in titanium cans with a sapphire window. The high-pressure gas would be contained in a titanium pressure vessel. The design is based on successful operation of the NEXT-1 prototypes which have demonstrated good energy resolution at 667 keV which implies, an adequate energy resolution with the usual $\sqrt{E}$~ scaling at the \Qbb energy.
 
The technology choice was made just a few weeks prior to the meeting and the collaboration has done a good job of bringing forward the ‘conceptual design’ document\footnote{Published in \cite{Alvarez:2011my}} in such a short time. The committee encourages the team to continue to flesh out this conceptual design so that it can go forward for review and final approvals. 
 
{\em The committee concurs with the need to move expeditiously to build the detector if the results are to be competitive. The committee feels that the choice of electroluminescence is likely to lead to energy and tracking measurements adequate for the project}. The committee supports the conservative choice of a simple, titanium pressure vessel built to ASME Section VIII rules.
 
The committee has the following suggestions for the team:
 \begin{enumerate}
\item The use of sapphire windows in front of the PMT’s looks feasible but this concept should be demonstrated\footnote{The concept was refined in the Technical Design Report\cite{Alvarez:2012flf}}.
\item The readout of the MPPC’s needs to be developed further because of the high multiplicity of these devices\footnote{Demonstrated in \cite{Abazajian:2012ys}}.
\item The project needs to demonstrate that large area coating of the MPPC’s is feasible and that the stability of the coatings is adequate\footnote{Demonstrated in \cite{Alvarez:2012ub}}.
\item The concept for calibrating the detector needs to be developed\footnote{Demonstrated
in \cite{Alvarez:2012haa}.}.
\item The background model needs to be developed further involving both analysis of the signals and the likely sources. The ability to discriminate alpha, beta and gamma backgrounds should be part of this analysis. From this would follow the materials purity criteria for each of the components and hence the need for radioactive screening. A request for screening facilities should be presented to the laboratory as soon as possible\footnote{Demonstrated
in the following papers: \cite{Alvarez:2012as,Alvarez:2013wxj,Dafni:2014yja,Alvarez:2014kvs,Cebrian:2015jna}}.
\item It was suggested that the conceptual design of the project would proceed separately for the different systems that make up the detector. The committee feels that it is important to have a single overall conceptual design. Once the framework is established, it is possible to have separate detailed design activities but the design criteria and interfaces must be complete at the conceptual design level to enable project review, approval and funding decisions to be made.
\item A work breakdown structure (WBS) with institutional assignments was presented orally. It is important to develop the WBS as part of the CDR. Letters of commitment from the US institutions were also presented. It is also important to get similar letters from all of partners especially in light of the technical choices that have been made.
\item The project should develop a risk analysis\footnote{Developing a full risk analysis has proven to be one of the more difficult tasks for the project, as will be discussed further in this report.}. 
\item The project team should develop a roadmap for completion and operation of the detector. Important elements of such a roadmap are stages in the development at which reviews would be carried out (and who would do these), and when critical decisions need to be made (including external decisions such as laboratory approvals and funding decisions).
\item The support systems and the pressure vessel are major engineering projects. The project must indicate the processes that will be followed to provide the necessary quality assurance that these will present adequately low risk.
\end{enumerate}

{\bf The committee congratulates the Collaboration for it progress since the last Scientific Committee meeting, and strongly encourages them to pay particular attention to the above recommendations. }

\end{quotation}

\paragraph{Summary-May 2011:} May 2011 was a very important milestone for several reasons: a) the collaboration 
{\bf had made a choice on the readout technology}. That choice involved an internal discussion in the collaboration, after the presentation of {\bf Conceptual Design Reports Proposals}, CDRP. Two such proposals were presented, one, supporting the choice of electroluminescence (EL), signed by most of the collaboration groups. The second CDRP presented micromegas as an alternative and was signed by the groups of IFAE and U. Zaragoza. The CIEMAT group did not sign any of the CDRPs.

The choice of the EL technology, proposed by D. Nygren (then at Berkeley National Laboratory, now professor at U. of Texas at Arlington), was based in the excellent preliminary results obtained with the initial operation of NEXT-DEMO and NEXT-DBDM, which indicated that a resolution of 1\% FWHM was possible. Those initial results were later supported by extensive studies which have shown a resolution in the range of 0.5--0.7 \% FWHM  \cite{Alvarez:2012yxw, Alvarez:2012zsz,Alvarez:2012hu,Alvarez:2013gxa,Lorca:2014sra,Renner:2014mha,Serra:2014zda} and a very strong topological signature  \cite{Ferrario:2015kta}. 

Studies carried out with the NEXT-MM prototype have shown that the energy resolution using a micromegas system is of the order of
3\% FWHM at \Qbb \cite{Alvarez:2013oha,Alvarez:2013kqa} even when adding a quencher (TMA). The addition of TMA, however,  suppresses the primary scintillation light and therefore the start-of-the-event ($t_0$) signal. As a consequence the event cannot be localised in the longitudinal coordinate (z) and the backgrounds increase enormously. The choice made in 2011, therefore, appears fully justified. 

{\em Notice that the committee supported the choice of EL and the plan to build the NEXT detector as fast as possible.}

\subsubsection*{2011-December}
\begin{quotation}
LSC Scientific Committee\\
9th Meeting\\
December 1st and 2nd, 2011\\
EXP-05 (NEXT)\\

{\bf Summary}

The NEXT collaboration {\em has made tremendous progress} in defining the gaseous xenon detector for double beta decay. The conceptual design has been improved in many respects and prototype detectors of moderate scale have been successfully operated proving the main concepts of the proposed design. Excellent resolution has been demonstrated for 137Cs gamma rays at 662 keV \footnote{Relevant references: \cite{Alvarez:2012yxw, Alvarez:2012zsz,Alvarez:2012hu,Alvarez:2013gxa,Lorca:2014sram}} and good tracking has also been shown \footnote{Relevant references:  \cite{Ferrario:2015kta}}. {\em The committee strongly supports the need to expedite this project} as there are now 2 operating xenon detectors in the world and the scientific impact will be greatly reduced if the NEXT project cannot be completed on their proposed timescale.

{\bf Detailed response for the collaboration.}

{\em The collaboration is to be congratulated for the excellent progress}. The committee has concerns regarding aspects of the project not presented in the written documentation. These are mainly in the area of project overview and management. The committee would like to see these addressed as soon as possible. 
There should be a (brief) discussion of the specific physics objectives. From a possible 150 kg of Xe what limits can you hope to reach, what are the implications for the backgrounds\footnote{The collaboration conducted many studies to define the final background model. The most updated results haven been recently published \cite{Martin-Albo:2015rhw}.}. Once you have specified the background constraints you have a hard number to work with in designing shielding, materials purity etc. The project should then have the usual project management structures. These should include:
An organization structure – i.e. how do decisions get made?
A WBS – you are well on the way but it probably needs more details.
A schedule – you have individual task schedules but not the overall plan. 

{\bf Milestones the committee would like to see:}
\begin{itemize}
\item When do you freeze various parts of the design. Often this is done after a design review (which can be internal or involve some external experts). You have a design that is moving very quickly (Good!) but, in effect, some aspects are frozen as you are making substantial purchases. Are all the implications of a choice understood and accepted by all parties responsible for other parts of the design?
\item What are the plans for the prototypes. Is there additional information required that impacts the design? Will the PMT housing and system be subject to a prototype? The committee strongly recommends that at least two of the R11410 PMT’s in their full enclosure be tested for long-term stability. 
\item What are the requirements for radioactivity assays and what facilities will be used to do the assays? The committee is particularly concerned that the components of the PMT and its housing and base be properly screened because there is no shielding between these and the sensitive areas of the detector. With limited facilities, these assays take a long time and hence the concern on schedule impact.
\item The schedule/WBS should be resourced to see that it can be accomplished with available manpower
\end{itemize}


There should be a budget. The financial discussion should indicate what funds are available and where the balance is expected to come from. There should be discussion of budget management. Who controls what part of the budget. How do you control the contingency (is it done centrally, by country, by institution?). What resources do you expect from the laboratory?

There is no wish to impose unnecessary burdens on the group when they are trying so hard to move this project forward. However, it is felt that with attention to the high level overview, mistakes can be avoided saving time and money. In addition, it will probably be essential to present this type of information in the requests for final funding.

{\bf	
The committee congratulates the Collaboration for it progress since the last Scientific Committee meeting, and strongly encourages them to continue to attempt to freeze the design, and experimentally verify key components and systems
}

\end{quotation}

\paragraph{Summary-December 2012:} The committee agreed with the technical choices and congratulated the collaboration by the achieved progress. The committee encourage the collaboration to proceed as fast as possible and was worried about management and  funding issues. In December 2011 the collaboration was facing serious difficulties related with funding. See section \ref{sec.diff}.


