\subsubsection*{2010-April}
\begin{quotation}
LSC Scientific Committee \\
8th Meeting \\
May 26th and 27th, 2011 \\
EXP-05 (NEXT)\\

{\em The committee congratulates the NEXT collaboration on very substantial progress made towards the definition of the conceptual design for the xenon double beta decay experiment}. A major decision has been the choice of electroluminescence as the gain mechanism. The tracking readout would use Si-multi-pixel photon counters coated with TBP with devices on a 1 cm grid. The energy readout would employ an array of PMT’s on the opposite side of the detector from the tracking readout. The chamber would operate at 15 bar. The team has demonstrated that the preferred PMT’s would not survive at such pressures and the design has them housed in titanium cans with a sapphire window. The high-pressure gas would be contained in a titanium pressure vessel. The design is based on successful operation of the NEXT-1 prototypes which have demonstrated good energy resolution at 667 keV which implies, an adequate energy resolution with the usual $\sqrt{E}$~ scaling at the \Qbb energy.
 
The technology choice was made just a few weeks prior to the meeting and the collaboration has done a good job of bringing forward the ‘conceptual design’ document\footnote{Published in \cite{Alvarez:2011my}} in such a short time. The committee encourages the team to continue to flesh out this conceptual design so that it can go forward for review and final approvals. 
 
{\em The committee concurs with the need to move expeditiously to build the detector if the results are to be competitive. The committee feels that the choice of electroluminescence is likely to lead to energy and tracking measurements adequate for the project}. The committee supports the conservative choice of a simple, titanium pressure vessel built to ASME Section VIII rules.
 
The committee has the following suggestions for the team:
 \begin{enumerate}
\item The use of sapphire windows in front of the PMT’s looks feasible but this concept should be demonstrated\footnote{The concept was refined in the Technical Design Report\cite{Alvarez:2012flf}}.
\item The readout of the MPPC’s needs to be developed further because of the high multiplicity of these devices\footnote{Demonstrated in \cite{Abazajian:2012ys}}.
\item The project needs to demonstrate that large area coating of the MPPC’s is feasible and that the stability of the coatings is adequate\footnote{Demonstrated in \cite{Alvarez:2012ub}}.
\item The concept for calibrating the detector needs to be developed\footnote{Demonstrated
in \cite{Alvarez:2012haa}.}. 
\item The background model needs to be developed further involving both analysis of the signals and the likely sources. The ability to discriminate alpha, beta and gamma backgrounds should be part of this analysis. From this would follow the materials purity criteria for each of the components and hence the need for radioactive screening. A request for screening facilities should be presented to the laboratory as soon as possible\footnote{Demonstrated
in the following papers: \cite{Alvarez:2012as,Alvarez:2013wxj,Dafni:2014yja,Alvarez:2014kvs,Cebrian:2015jna}}.
\item It was suggested that the conceptual design of the project would proceed separately for the different systems that make up the detector. The committee feels that it is important to have a single overall conceptual design. Once the framework is established, it is possible to have separate detailed design activities but the design criteria and interfaces must be complete at the conceptual design level to enable project review, approval and funding decisions to be made.
\item A work breakdown structure (WBS) with institutional assignments was presented orally. It is important to develop the WBS as part of the CDR. Letters of commitment from the US institutions were also presented. It is also important to get similar letters from all of partners especially in light of the technical choices that have been made.
\item The project should develop a risk analysis. 
\item The project team should develop a roadmap for completion and operation of the detector. Important elements of such a roadmap are stages in the development at which reviews would be carried out (and who would do these), and when critical decisions need to be made (including external decisions such as laboratory approvals and funding decisions).
\item The support systems and the pressure vessel are major engineering projects. The project must indicate the processes that will be followed to provide the necessary quality assurance that these will present adequately low risk.
\end{enumerate}

{\bf The committee congratulates the Collaboration for it progress since the last Scientific Committee meeting, and strongly encourages them to pay particular attention to the above recommendations. }

\end{quotation}

\paragraph{Summary-April 2011:} April 2011 was a very important milestone for several reasons: a) the collaboration 
{\bf had made a choice on the readout technology}. That choice involved an internal discussion in the collaboration, after the presentation of {\bf Conceptual Design Reports Proposals}, CDRP. Two such proposals were presented, one, supporting the choice of electroluminescence (EL), signed by most of the collaboration groups. The second CDRP presented micromegas as an alternative and was signed by the groups of IFAE and U. Zaragoza. The CIEMAT group did not sign any of the CDRPs.

The choice of the EL technology, proposed by D. Nygren (then at Berkeley National Laboratory, now professor at U. of Texas at Arlington), was based in the excellent preliminary results obtained with the initial operation of NEXT-DEMO and NEXT-DBDM, which indicated that a resolution of 1\% FWHM was possible. Those initial results were later supported by extensive studies which have shown a resolution in the range of 0.5--0.7 \% FWHM  \cite{Alvarez:2012yxw, Alvarez:2012zsz,Alvarez:2012hu,Alvarez:2013gxa,Lorca:2014sram,Renner:2014mha,Serra:2014zda} and a very strong topological signature  \cite{Ferrario:2015kta}. 

Studies carried out with the NEXT-MM prototype have shown that the energy resolution using a micromegas system is of the order of
3\% FWHM at \Qbb \cite{Alvarez:2013oha,Alvarez:2013kqa} even when adding a quencher (TMA). The addition of TMA, however,  suppresses the primary scintillation light and therefore the start-of-the-event ($t_0$) signal. As a consequence the event cannot be localised in the longitudinal coordinate (z) and the backgrounds increase enormously. The choice made in 2011, therefore, appears fully justified. 

{\em Notice that the committee supported the choice of EL and the plan to build the NEXT detector as fast as possible.}





