
\documentclass[a4paper,11pt,oneside]{article}
\usepackage[a4paper,vmargin={1.5cm,1.5cm},width=16cm]{geometry}
\usepackage[style=verbose-inote,doi=false,sortcites=true,block=space, backend=bibtex]{biblatex}
\usepackage[utf8]{inputenc}
\usepackage{textcomp}
\usepackage[spanish]{babel}
\usepackage{microtype}
\usepackage{lmodern}
\usepackage{graphicx}
\usepackage{fancyhdr}
\usepackage{booktabs}
\usepackage{eurosym}
\usepackage{mathptmx}
\usepackage[T1]{fontenc}
\usepackage{hyperref}
%% Added to help mimic structure.
\usepackage[many]{tcolorbox}
\usepackage{soul}
\usepackage{color}
\usepackage{lastpage}

%\documentclass[11pt,a4paper]{article}
%\usepackage[utf8]{inputenc}
%\usepackage[spanish]{babel}
%\usepackage[a4paper,height=26cm,width=17cm]{geometry}
%\usepackage{graphicx}
%\usepackage{epsfig,rotating}
%\usepackage{amssymb}
%\usepackage{mathrsfs}
%\usepackage{amsmath}
%\usepackage{amsfonts}
%\usepackage{multirow}
%\usepackage{eurosym}
%\usepackage{dcolumn}
%\usepackage{cite}

\usepackage{multirow}
%\usepackage{enumitem}

%\usepackage[hyphens]{url}
\usepackage{hyperref}


%% Hack to make math formulas bold in section titles
\makeatletter
\DeclareRobustCommand*{\bfseries}{%
  \not@math@alphabet\bfseries\mathbf
  \fontseries\bfdefault\selectfont
  \boldmath
}
\makeatother

\bibliography{biblio}


\begin{document}

% BB
\newcommand{\bb}{\ensuremath{\beta\beta}}
% BB0NU
\newcommand{\bbonu}{\ensuremath{\beta\beta0\nu}}
% BB2NU
\newcommand{\bbtnu}{\ensuremath{\beta\beta2\nu}}
% NME
\newcommand{\Monu}{\ensuremath{\Big|M_{0\nu}\Big|}}
\newcommand{\Mtnu}{\ensuremath{\Big|M_{2\nu}\Big|}}
% PHASE-SPACE FACTOR
\newcommand{\Gonu}{\ensuremath{G_{0\nu}(\Qbb, Z)}}
\newcommand{\Gtnu}{\ensuremath{G_{2\nu}(\Qbb, Z)}}

% mbb
\newcommand{\mbb}{\ensuremath{m_{\beta\beta}}}
\newcommand{\kgy}{\ensuremath{\rm kg \cdot y}}
\newcommand{\ckky}{\ensuremath{\rm counts/(keV \cdot kg \cdot yr)}}
\newcommand{\mbba}{\ensuremath{m_{\beta\beta}^a}}
\newcommand{\mbbb}{\ensuremath{m_{\beta\beta}^b}}
\newcommand{\mbbt}{\ensuremath{m_{\beta\beta}^t}}
\newcommand{\nbb}{\ensuremath{N_{\beta\beta^{0\nu}}}}

% Qbb
\newcommand{\Qbb}{\ensuremath{Q_{\beta\beta}}}

% Tonu
\newcommand{\Tonu}{\ensuremath{T_{1/2}^{0\nu}}}

% Tonu
\newcommand{\Ttnu}{\ensuremath{T_{1/2}^{2\nu}}}

% Xe-136
\newcommand{\Xe}{\ensuremath{^{136}}Xe}
\newcommand{\COT}{\ensuremath{CO_2}}
\newcommand{\CHF}{\ensuremath{CH_4}}
\newcommand{\CFF}{\ensuremath{CF_4}}

% 2S
\newcommand{\TwoS}{\ensuremath{^{2}S_{1/2}}}

\newcommand{\TwoP}{\ensuremath{^{2}P_{1/2}}}

\newcommand{\TwoD}{\ensuremath{^{2}D_{3/2}}}


% Xe-136
\newcommand{\CS}{\ensuremath{^{137}}Cs}

% Xe-136
\newcommand{\NA}{\ensuremath{^{22}}Na}


% Bi-214
\newcommand{\Bi}{\ensuremath{^{214}}Bi}

% Tl-208
\newcommand{\Tl}{\ensuremath{^{208}}Tl}

% Pb-208
\newcommand{\Pb}{\ensuremath{^{208}}Pb}
% Pb-208
\newcommand{\PBD}{\ensuremath{^{210}}Pb}

% Po-214
\newcommand{\Po}{\ensuremath{^{214}}Po}
\newcommand{\Kr}{\ensuremath{^{83}}Kr}

% bru
\newcommand{\bru}{cts/(keV$\cdot$kg$\cdot$y)}
\newcommand{\dten}{10 mm/$\sqrt{\rm m}$}
\newcommand{\dtwo}{2 mm/$\sqrt{\rm m}$}
\newcommand{\BAPP}{\ensuremath{Ba^{++}}}
\newcommand{\BAP}{\ensuremath{Ba^{+}}}

\newcommand{\HPXE}{\sc{HPXe}\rm}
\newcommand{\BATA}{\sc{BaTa}\rm}

% Saltos de carro en tablas
\newcommand{\minitab}[2][l]{\begin{tabular}{#1}#2\end{tabular}}

\newcommand{\thedraft}{0.1.1}% version for referees

\newcommand{\MO}{\ensuremath{{}^{100}{\rm Mo}}}
\newcommand{\SE}{\ensuremath{{}^{82}{\rm Se}}}
\newcommand{\ZR}{\ensuremath{{}^{96}{\rm Zr}}}
\newcommand{\KR}{\ensuremath{{}^{82}{\rm Kr}}}
\newcommand{\ND}{\ensuremath{{}^{150}{\rm Nd}}}
\newcommand{\XE}{\ensuremath{{}^{136}\rm Xe}}
\newcommand{\GE}{\ensuremath{{}^{76}\rm Ge}}
\newcommand{\GES}{\ensuremath{{}^{68}\rm Ge}}
\newcommand{\TE}{\ensuremath{{}^{128}\rm Te}}
\newcommand{\TEX}{\ensuremath{{}^{130}\rm Te}}
\newcommand{\TL}{\ensuremath{{}^{208}\rm{Tl}}}
\newcommand{\CA}{\ensuremath{{}^{48}\rm Ca}}
\newcommand{\CO}{\ensuremath{{}^{60}\rm Co}}
\newcommand{\PO}{\ensuremath{{}^{214\rm Po}}}
\newcommand{\U}{\ensuremath{{}^{235}\rm U}}
\newcommand{\CT}{\ensuremath{{}^{10}\rm C}}
\newcommand{\BE}{\ensuremath{{}^{11}\rm Be}}
\newcommand{\BO}{\ensuremath{{}^{8}\rm Be}}
\newcommand{\UDTO}{\ensuremath{{}^{238}\rm U}}
\newcommand{\CD}{\ensuremath{^{116}{\rm Cd}}}
\newcommand{\THO}{\ensuremath{{}^{232}{\rm Th}}}
\newcommand{\BI}{\ensuremath{{}^{214}}Bi}
\newcommand{\FDG}{\ensuremath{^{18}}F}


\begin{center}
{\LARGE \bf \textsf{CONSOLIDER-INGENIO 2010 PROGRAMME }} \\ \vspace{0.5cm}
{\Large \bf \textsf{FINAL SCIENTIFIC ANNUAL REPORT}} \\ \vspace{0.5cm}
{\Large \bf \textsf{I01/01/2012 al 31/12/2012 }} \\ \vspace{2cm}
\end{center}


\begin{table}[htp!]
\begin{center}
\begin{tabular}{|c|c|}
\hline
PROJECT REFERENCE NUMBER: & CDS2008-00037\\
\hline
Coordinating Researcher:  & M.C. G\'onzalez-Garc\'ia \\
Project Title: & Canfranc Underground Physics (CUP) \\
Managing Institution:  & CSIC \\
Project Start Date:  & 15/12/2008\\
Project Final Date: &  14/12/2014)\\
\hline\hline
\end{tabular}
\end{center}
\caption{Project data}
\label{default}
\end{table}%

\newpage

%%%%%%%%%%%%%%%%%%%%%%%%%%%%%%%%%%%%%%%%%%%%%%%%%%%%%%%%%
%% SECTION 1
\section{\bf \textsf{CUP: summary and main goals }}

CUP (Canfranc Underground Physics) was granted with the main goal of developing the scientific program of the
 \emph{Laboratorio Subterr\'aneo de Canfranc} (LSC), one of the ICTS (instalaci\'on cient\'ifico t\'ecnica singular) of the ministry of science (currently integrated in the MINECO). The two main activities of CUP were called
 CAFE (Canfranc Future Experiment) and NEXT (Neutrino Experiment with a Xenon TPC).

CAFE has spawn a wide range of phenomenological studies, coordinated by professor 
M.C. Gonz\'alez-Garc\'ia. These studies have involved several institutions of the CUP consortium, including the Universidad de Barcelona (UB), Instituto de F\'isica Corpuscular, (IFIC), and Instituto de F\'isica Te\'orica (IFT). 

The goals of the subproject CAFE were:
\begin{enumerate}
\item The characterisation of LSC to demonstrate its viability 
as a large underground European laboratory in the context of
the European project LAGUNA (Large Apparatus for Studying Grand
Unification and Particle Astrophysics) as well as in the context
of studies of viability for future neutrino factories which were
performed as part of the European project EURO-$\nu$.
\item Perform phenomenological studies to explore and exploit 
the physics potential LSC  as well more broad studies related
to astroparticle physics in general (neutrino physics, cosmology, etc.)
\end{enumerate}

%The CAFE activities were amply described in our previous reports, and here we will simply update the lists of publications related with the activity. 

%Two major spinoffs of the activity are two European projects directly related with CAFE goals: 
%
%\begin{enumerate}
%\item 
%A european training network, called INVISIBLES, coordinates by professor B. Gavela, from IFT. 
%%
%\item A european project of large infrastructures called LAGNA-LBNO. The spanish IP of the project is  professor J.J. Gómez-Cadenas, for IFIC, who acts also as spokesperson of NEXT.
%\end{enumerate}

CUP main goal was the construction, commissioning and operation of the NEXT detector\footnote{\url{http://next.ific.uv.es/next/}}, a high-pressure, xenon (HPXe) Time Projection Chamber (TPC), whose goal was to search for neutrinoless double beta decay  (\bbonu) events in xenon enriched at 90\% in the isotope \XE. The first phase of the experiment, called NEW, deploying 10 kg of xenon, is currently being commissioned at the LSC.The second phase of the experiment, NEXT-100, will deploy 100 kg of xenon. A third phase, deploying up to one ton of xenon, is actively being discussed, with strong international interest.
 
%
%The discovery potential of NEXT is very large. It combines four desirable features that make it an almost-ideal experiment for \bbonu\ searches, namely: 
%\begin{enumerate}
%\item Excellent energy resolution (better than 1\% FWHM in the region of interest).
%\item A topological signature (the observation of the tracks of the two electrons).
%\item A fully active, very radiopure apparatus of large mass.  
%\item The capability of extending the technology to much larger masses.
%\end{enumerate}
%
%The project has evolved very satisfactorily, from the initial Letter of Intent (LOI) in 2009 to the Technical Design Report (TDR) in 2012. A substantial number of papers, proceeding reports and conferences, documenting and demonstrating the physics case, the results of the prototypes and the technological choices have been published and are attached to this document. They can also be found in the NEXT web page: \url{http://next.ific.uv.es/next/talks.html}.
%


\section{\bf \textsf{Main advances and profits }}
CAFE has result into several advances
\begin{enumerate}
\item {\bf It has enhanced the visibility of the LSC} by participating into
European studies devoted to the long term planning for large infrastructures
in Europe such as of such as LAGUNA-LBNO and Euro-$\nu$.
\item {\bf It has contributed to the  creation of a new generation of European physicists}
with expertness in phenomenology of relevance for underground physics
by our participation in two consecutive International Training Networks
and a Network for Research and Innovation Staff Exchange, all
funded by the European Commission, as reported below. 
\item  {\bf It has  produced a large number of scientific articles, which have
advanced quantitatively the state-of-the-art of the field.} The total
number of references of published articles by the CAFE senior members
in international peer reviewed journals 
since 2008 is 70  (listed below) that  have  received a total of 
more than 3000 citations in the SPIRES data basis which is the one commonly used
in our field for reference.
\end{enumerate}

NEXT has result in several advances:
\begin{enumerate}
\item {\bf It has advanced the field of underground physics instrumentation}, with the construction of the NEXT detector series. The large prototype which has operated at IFIC since 2011 (NEXT-DEMO) was the largest HPXe electroluminescent TPC in the World, until the construction of NEW, currently being commissioned at the LSC. The detectors have shown the feasibility of the electroluminescent technology, have demonstrated the excellent energy resolution that can be achieved in xenon gas, and have demonstrated  the availability of a unique topological signature in HPXe, e.g, the capability to fully reconstruct the two electrons emitted in \bb\ decay, providing a powerful tool to discriminate against backgrounds. 
\item {\bf It has resulted in a cutting-edge experiment}, which will be searching for \bbonu\ decays during the next few years, with chances of making or contributing to a major discovery. 
\item {\bf It has bought to the field a new technology}, that of the HPXe-EL, which is currently considered one of the best candidates for the next generation of \bbonu\ experiments.
 \item {\bf It has resulted in an international collaboration}, involving groups from Portugal, United States, Russia and Colombia. Other groups (Poland, Australia), are currently considering joining the experiment.
 \item {\bf It has attracted external funding}, with contributions to the experiment from all the members of the collaboration, and in particular from the US.
  \item {\bf It has resulted in an Advanced Grant of the ERC}, granted to the spokesperson of NEXT, prof. J.J. G\'omez Cadenas, who has acted as co-PI of CUP. 
   \item {\bf It has boosted the scientific interest of the LSC}, as intended in our proposal. Currently NEXT is the most important experiment in the Canfranc Underground Laboratory and the only one with truly international projection. 
  \item {\bf It has produced a major scientific spinoff, called PETALO}, a new concept for a full-body, TOF-capable PET using xenon. Currently, a patent has been filled and two scientific papers have been published on the new concept. A new collaboration is being formed to explore the scientific potential and commercial applications of PETALO. 
   \item {\bf It has contributed in an important way to science outreach}. See for example:
   \url{http://www.jotdown.es/2012/09/david-nygren-y-alessandro-bettini-the-physics-as-fountain-of-eternal-youth/}{link1}
   \url{http://www.jotdown.es/2013/11/coversaciones-de-fisica-en-el-santa-cristina-ariella-cattai-y-concha-gonzalez-garcia/}
\end{enumerate}


\section{\bf \textsf{ Key Indicators}}

\subsection*{Reports from the LSC Scientific Committee}

In the CUP proposal, we stated:
\begin{quotation}
We propose a very simple and efficient evaluation scheme for CUP. The  activities are focused in the LSC, which has an international scientific committee, composed by high-reputation specialists in underground physics. This committee is perfectly suited to evaluate our progress in all the areas proposed.
\end{quotation}

In this section we present the reports of the LS Scientific Committee (LSCSC) since 2010, together with a discussion concerning the implementations of the committee recommendations. The committee reports are copied verbatim. {\em Emphasis} is used to highlight key aspects of the text, while {\bf bold face} is only used if the committee used it in the original report. Footnotes, often added to specific recommendations, add relevant comments or reference to bibliography. 

\subsubsection*{2010-April}
\begin{quotation}
LSC Scientific Committee\\
6th Meeting\\
April 15 and 16, 2010\\
EXP-05-2008 (NEXT)\\

The   Committee   {\bf commends   the   Collaboration   for   its   effort   to   improve   its organization, working relationship among members, and its project management. The establishment of a Steering Committee to choose the readout system technology was a positive step. The effort to involve foreign institutions is also laudable. NEXT is making excellent progress;} however the Committee has some recommendations.

The Committee recommends:
\begin{itemize}
\item that the collaboration finalize the selection of a technology choice, a plan of action, and write a schedule,
\item  that the Collaboration design a prototype to bridge from NEXT-1 to NEXT-100, with long drift distances matching those of NEXT-100,
\item that the Collaboration have a final readout system operating under 10 Atm of Xe gas,
that  the  Collaboration  write  a  detailed  plan  for  radioactive  screening  of construction materials
\item  that the Collaboration establish a presence at the Laboratory to determine their needs for counting, etc.,
\item that  the  Collaboration make a study  of  the safety  issues  with  10  Atm.  gas chambers underground, and begin discussions with the Laboratory personnel.
\item that the Collaboration write a Project Management Plan including a budget, schedule and names of persons responsible for specific tasks,
\end{itemize}
\end{quotation}

\subsubsection*{2010-October}
\begin{quotation}
LSC Scientific Committee\\
7th Meeting\\
October 7th and 8th, 2010\\
EXP-05-2008 (NEXT)\\

The Committee applauds the steps taken to further develop the strength of the collaboration, with an increasing commitment of the US groups and the enlargement to groups of the Universidad de Aveiro in Portugal and of the Universidad Antonio Nari\~no in Columbia.

The Committee was pleased by the informative presentations and the new format of the progress report. In general the project is making steady progress in its R\&D phase of the experiment. The most critical topic presented is the choice of the readout technology. The Committee observes that the Micro-Megas readout was tested at the Universities of Saragossa and Coimbra, and achieved an energy resolution of about 3\% at 10-atm pressure to be compared to the needed one, which is better than 1\%. The groups at LBNL, TAMU, IFIC, UPV, Nari\~no, Coimbra and CIEMAT are working on electroluminescence (EL) read out, which, however, has not yet demonstrated full feasibility. The Committee also observes that the Collaboration is facing the problem of non availability of radio-pure PMTs, which can take 10-15 bar pressure, by encasing them into quartz tubes. This feature allows to add a light guide function helping in the light collection. The Collaboration also presented an approach to improving light sensitivity of the SiPMs. 

{\bf
The Committee considers positively the NEXT Collaboration's reply to the recommendation made in its 6th meeting on the choice of readout technology, in particular the plan to present the Conceptual Design Report (CDR) in time for the Committee Meeting in the Spring of 2011. In particular, the Committee understands that a procedure to make the decision on the readout technology will be defined by the Collaboration in advance of that date. The decision should be based on the all available information at that time, and should take into account that experiments of similar sensitivity are already in more advanced stages of readiness.}
 
 The Committee recommended at the last meeting, and still does, that an intermediate detector be built to test the many new elements in the NEXT-100, for example the 10-atm pressure of Xe gas and the long drift distances/long drift times, as well as the readout system operation under these conditions. The Committee also recommends further study of the radon background and radio-purity measurements on construction materials to mitigate against excess background from radioactivity.

The Committee also welcomes the fact that the Collaboration is developing its Work Plan with milestones that will be integrated into the project management plan and looks forward to receiving the project management plan and the CDR before the next meeting. The Committee applauds the Collaboration for progress report with a much-improved format, but which still lacks budget and schedule information
\end{quotation}

\paragraph{Summary-2010:} The most critical issue identified by the committee in 2010 was the need to choose a technology for NEXT. As described in previous reports, such a choice was one of the first goals of CUP. At the time there were two possibilities. a) A readout based in Micromegas, pioneered by the group of Zaragoza, and b) an electroluminescent readout, using only light signals, and instrumenting the detector with SiPMs and PMTs. In 2010, the committee observed that the Micromegas technology did not meet the energy resolution requirements for the experiment, but also noticed that the EL option had not shown full feasibility. The committee recommended the construction of an intermediate prototype. Notice that none of the three NEXT prototypes (NEXT-DEMO at IFIC, NEXT-DBDM at LBNL, and NEXT-MM at Zaragoza) were yet fully operational, and the available results were based mostly in small prototypes at IFIC, LBNL and Zaragoza. 


\subsubsection*{2010-April}
\begin{quotation}
LSC Scientific Committee \\
8th Meeting \\
May 26th and 27th, 2011 \\
EXP-05 (NEXT)\\

{\em The committee congratulates the NEXT collaboration on very substantial progress made towards the definition of the conceptual design for the xenon double beta decay experiment}. A major decision has been the choice of electroluminescence as the gain mechanism. The tracking readout would use Si-multi-pixel photon counters coated with TBP with devices on a 1 cm grid. The energy readout would employ an array of PMT’s on the opposite side of the detector from the tracking readout. The chamber would operate at 15 bar. The team has demonstrated that the preferred PMT’s would not survive at such pressures and the design has them housed in titanium cans with a sapphire window. The high-pressure gas would be contained in a titanium pressure vessel. The design is based on successful operation of the NEXT-1 prototypes which have demonstrated good energy resolution at 667 keV which implies, an adequate energy resolution with the usual $\sqrt{E}$~ scaling at the \Qbb energy.
 
The technology choice was made just a few weeks prior to the meeting and the collaboration has done a good job of bringing forward the ‘conceptual design’ document\footnote{Published in \cite{Alvarez:2011my}} in such a short time. The committee encourages the team to continue to flesh out this conceptual design so that it can go forward for review and final approvals. 
 
{\em The committee concurs with the need to move expeditiously to build the detector if the results are to be competitive. The committee feels that the choice of electroluminescence is likely to lead to energy and tracking measurements adequate for the project}. The committee supports the conservative choice of a simple, titanium pressure vessel built to ASME Section VIII rules.
 
The committee has the following suggestions for the team:
 \begin{enumerate}
\item The use of sapphire windows in front of the PMT’s looks feasible but this concept should be demonstrated\footnote{The concept was refined in the Technical Design Report\cite{Alvarez:2012flf}}.
\item The readout of the MPPC’s needs to be developed further because of the high multiplicity of these devices\footnote{Demonstrated in \cite{Abazajian:2012ys}}.
\item The project needs to demonstrate that large area coating of the MPPC’s is feasible and that the stability of the coatings is adequate\footnote{Demonstrated in \cite{Alvarez:2012ub}}.
\item The concept for calibrating the detector needs to be developed\footnote{Demonstrated
in \cite{Alvarez:2012haa}.}. 
\item The background model needs to be developed further involving both analysis of the signals and the likely sources. The ability to discriminate alpha, beta and gamma backgrounds should be part of this analysis. From this would follow the materials purity criteria for each of the components and hence the need for radioactive screening. A request for screening facilities should be presented to the laboratory as soon as possible\footnote{Demonstrated
in the following papers: \cite{Alvarez:2012as,Alvarez:2013wxj,Dafni:2014yja,Alvarez:2014kvs,Cebrian:2015jna}}.
\item It was suggested that the conceptual design of the project would proceed separately for the different systems that make up the detector. The committee feels that it is important to have a single overall conceptual design. Once the framework is established, it is possible to have separate detailed design activities but the design criteria and interfaces must be complete at the conceptual design level to enable project review, approval and funding decisions to be made.
\item A work breakdown structure (WBS) with institutional assignments was presented orally. It is important to develop the WBS as part of the CDR. Letters of commitment from the US institutions were also presented. It is also important to get similar letters from all of partners especially in light of the technical choices that have been made.
\item The project should develop a risk analysis. 
\item The project team should develop a roadmap for completion and operation of the detector. Important elements of such a roadmap are stages in the development at which reviews would be carried out (and who would do these), and when critical decisions need to be made (including external decisions such as laboratory approvals and funding decisions).
\item The support systems and the pressure vessel are major engineering projects. The project must indicate the processes that will be followed to provide the necessary quality assurance that these will present adequately low risk.
\end{enumerate}

{\bf The committee congratulates the Collaboration for it progress since the last Scientific Committee meeting, and strongly encourages them to pay particular attention to the above recommendations. }

\end{quotation}

\paragraph{Summary-April 2011:} April 2011 was a very important milestone for several reasons: a) the collaboration 
{\bf had made a choice on the readout technology}. That choice involved an internal discussion in the collaboration, after the presentation of {\bf Conceptual Design Reports Proposals}, CDRP. Two such proposals were presented, one, supporting the choice of electroluminescence (EL), signed by most of the collaboration groups. The second CDRP presented micromegas as an alternative and was signed by the groups of IFAE and U. Zaragoza. The CIEMAT group did not sign any of the CDRPs.

The choice of the EL technology, proposed by D. Nygren (then at Berkeley National Laboratory, now professor at U. of Texas at Arlington), was based in the excellent preliminary results obtained with the initial operation of NEXT-DEMO and NEXT-DBDM, which indicated that a resolution of 1\% FWHM was possible. Those initial results were later supported by extensive studies which have shown a resolution in the range of 0.5--0.7 \% FWHM  \cite{Alvarez:2012yxw, Alvarez:2012zsz,Alvarez:2012hu,Alvarez:2013gxa,Lorca:2014sram,Renner:2014mha,Serra:2014zda} and a very strong topological signature  \cite{Ferrario:2015kta}. 

Studies carried out with the NEXT-MM prototype have shown that the energy resolution using a micromegas system is of the order of
3\% FWHM at \Qbb \cite{Alvarez:2013oha,Alvarez:2013kqa} even when adding a quencher (TMA). The addition of TMA, however,  suppresses the primary scintillation light and therefore the start-of-the-event ($t_0$) signal. As a consequence the event cannot be localised in the longitudinal coordinate (z) and the backgrounds increase enormously. The choice made in 2011, therefore, appears fully justified. 

{\em Notice that the committee supported the choice of EL and the plan to build the NEXT detector as fast as possible.}







%\begin{quotation}
%As for the CAFE activity, there is an obvious evaluation criteria, namely the scientific papers to be produced along the five years of the project, including the proceedings of the workshop(s) and/or schools as well as other outreach material.
%\end{quotation}
%
%The key indicators of the final results of CAFE are the following:
%
%\begin{enumerate}
%\item Involvement in the project LAGUNA-LBNO
%(\url{http://laguna.ethz.ch:8080/Plone}). This project continued the
%studied started by LAGUNA with the aim at developing large infrastructures
%in Europe. Spain participated with four associates LAGUNA-LBO, three of
%which are participants in CUP (UAM, IFIC y el propio LSC).
%\item Involvement of our scientists in the european project EURO-$\nu$.
%In particular de group at UAM made very relevant contributions to
%this project as listed in the bibliography.
%\item Participation in three funded projects of the European Commission
%for the training of young researchers in fields directly related
%with underground physics. They are: 
%\begin{itemize}
%\item 
%Title: INVISIBLES\\
%Initial year: 2012 Year final: 2016\\
%Reference: PITN-GA-2011-289442\\
%\item 
%Title: 	Elusives\\
%Initial year: 2016 Year final: 2020\\
%Reference: H2020-MSCA-ITN-2015-674896 \\
%\item  Title: 	InvisiblesPlus\\
%Initial year: 2016 Year final: 2020\\
%Reference: H2020-MSCA-RISE-2015-690575\\
%\end{itemize}
%The coordinator of the UB node of these networks is our IP, Prof.
%M.C. Gonzalez-Garcia while the coordinator of the node at IFIC
%is Prof. P. Hernandez, a member of CAFE/CUP.
%\item Publication in peer review international journal 
%of more than 70 articles in phenomenological studies of which 
%we list a selection in the
%bibliography. They have accumulated more than 3000 citation  in the
%basis SPIRES. For example our updated determinations of the neutrino 
%parameters, fundamental for any study of the expected signal at NEXT
%\cite{Gonzalez-Garcia:2014bfa,GonzalezGarcia:2012sz,GonzalezGarcia:2010er} 
%have accumulated more than 1000 citations,  
%while our contributions to studies associated to future neutrino facilities,
%such as Ref.~\cite{Bandyopadhyay:2007kx,Abazajian:2012ys,Choubey:2011zzq,Adey:2014rfv} have received more than 500 references (all data from SPIRES
%Data basis).
%\end{enumerate}
%
%

\section{\bf \textsf{ Difficulties}} 

The main difficulty the project has faced was related with the withdrawal, in 2011, of the IFAE and CIEMAT groups from the NEXT collaboration (although they did not formally leave the CUP consortium). The reasons argued for the withdrawal were related with different point of views concerning the technical choices implemented in NEXT, as well as disagreement with the leadership of the NEXT spokesperson.

A few months after the withdrawal of the IFAE and CIEMAT groups, the regular research project submitted by IFIC-UPV was marked with at C (mediocre) and rejected by the funding committee of particle physics (FPA). The external expert of that particular committee was prof. Giorgio Gratta (the spokesperson of the EXO experiment, which is the main competitor of NEXT) and the committee members included several leaders of the IFAE and CIEMAT group. The committee disregarded the reports of the LSC scientific committee and the opinion of the director of the LSC. The reports from the ANEP were requested and at least one of them was found to be extremely biased, with ad-hominem attacks and absolute lack of rigor, as acknowledged by the ANEP itself. However, the committee of particle physics did not reject that particular report.

The project submitted in 2011, requested essential additional funding, not covered by CUP. The negative result implied a major logistic drawback and a serious blow for the collaboration. In 2012, a new research project was submitted, to a different committee (general physics, FIS), given the obvious bias of the FPA committee. The project obtained a B and modest funding. In 2013, the spokesperson of NEXT was granted the first (and the only one, to date) advanced grant of the ERC in experimental particle physics in Spain. In 2014, another project was submitted to FIS, which obtained a top mark. 

The perception of the consortium and the international collaboration is that, in spite of the demonstrated excellence of the project (availed by the reports of the LSC scientific committee as well as by the excellent reports produced by external evaluators, in particular in the ERC), its capability to attract external funding, and its importance for the scientific case of the LSC, a bias still exist in FPA, mostly related with the opposition to the project of the IFAE group. Indeed, the ex-director of IFAE, Dr. Mateo Cavalli-Svorza has recently (in 2015) expressed publicly (in a seminar at the ALBA seminar) his enmity towards the spokesperson of the experiment. 

The situation is not fully solved yet, and poses a serious threat to the future of the collaboration, since external collaborators from other countries may feel that the project is not sufficiently supported in Spain. It also poses a serious threat for the future of the Canfranc laboratory, whose reputation would suffer very much should the NEXT experiment be forced to move to a different underground laboratory. 

The rejection of the research proposal in 2011 and the very scarce funding in 2012, forced the collaboration to redefine its initial plan to start NEXT-100 construction in 2012. Instead, it was decided to introduce a first-stage experiment, NEW (a detector with half of the size of NEXT-100, a fourth of the sensors and 10 kg of xenon mass). It was perceived that there was not sufficient support in Spain to attempt the construction of NEXT-100 directly. After the ERC grant the situation has improved considerably, but the collaboration decided to continue with the NEW program. Currently, NEW is starting data taking at the LSC, and the lessons learned in its construction (and certainly by operating it) are very valuable to build a better NEXT-100 detector. 


\section{\bf \textsf{ Scientific and technical activities }} 

5.-Describe the scientific and technical activities to reach the goals outlined in the project. Indicate, for each activity, the members of the equipment who have participated

%The scientific activity of CAFE has been described above. We list
%here the main lines developed in their phenomenological studies
%which are directly related to underground physics
%\begin{itemize}
%\item Precise and updated characterization of the leptonic flavour
%parameters, whose values are crucial to the prediction of expected
%event rates in neutrinoless double beta decay experiments
%\cite{Song:2015xaa,Bergstrom:2015rba,Gonzalez-Garcia:2014bfa,Gonzalez-Garcia:2013usa,GonzalezGarcia:2012sz,GonzalezGarcia:2011my,GonzalezGarcia:2010un,GonzalezGarcia:2010er}
%\item Studies of new phenomena expected from presence of additional
%sterile neutrinos and their effect at future underground experiments
%\cite{Vincent:2014rja,Bergstrom:2014fqa,Adey:2014rfv,GonzalezGarcia:2012yq,SolagurenBeascoa:2012cz,Donini:2011jh,Abazajian:2012ys}
%\item Studies directly related to future underground facilities
%in Europe 
%\cite{Bross:2013oua,Edgecock:2013lga,Blennow:2013swa,Bayes:2012ex,Agarwalla:2012zu,Hernandez:2012zz,Coloma:2012wq,Coloma:2011rq,Donini:2010xk,Coloma:2010wa,Choubey:2009ks,Bandyopadhyay:2007kx}
%\item Potential of underground experiments to detect and 
%characterize astrophysical neutrino sources 
%\cite{Bergstrom:2016cbh,Gonzalez-Garcia:2013iha,Ahlers:2011jj,Ahlers:2010fw,GonzalezGarcia:2009ya,GonzalezGarcia:2009jc}
%\item Study of theoretical frameworks for the connection  between Majorana 
%nature of the neutrino (as searched for
%in neutrinoless doublebeta decay) and the production of the
%observed matter-antimater in the Universe
%\cite{Hernandez:2015wna,Fong:2013gaa,Fong:2011yx,Fong:2010bv,Fong:2010bh,Fong:2010qh,Fong:2010zu,GonzalezGarcia:2009qd,Fong:2009iu}
%\item Construction of models which can account for all neutrino observations
%and study their specific predictions in future experiments
%(including neutrinoless doublebeta decay experiments)
%\cite{Gago:2015vma,Hernandez:2014fha,Hernandez:2013lza,Donini:2012tt,Eboli:2011ia,Gavela:2009cd}
%\item Studies of scenarios of electroweak symmetry breaking related
%for mass generation
%\cite{Corbett:2015lfa,Corbett:2014ora,Gavela:2014vra,Brivio:2014pfa,Brivio:2013pma,Corbett:2013pja,Corbett:2012ja,Corbett:2012dm,Eboli:2011ye,Eboli:2011bq,Eboli:2010qd,Alves:2009aa}
%\end{itemize}
%


\section{\bf \textsf{Collaborations between team members}}
6.-Relate the collaborations between all team members 


CAFE: As can be seen from our publication list there are regular
collaborations between the scientists in UB, IFIC and UAM. 
The members in the three groups in CAFE are all involved in the European
networks mentioned listed above. 


\section{\bf \textsf{Collaborations with other groups of research}}

7.-Relate the collaborations with other groups of research and the added value for the project. Describe, if necessary, the access to equipment or infrastructures of other groups or institutions

CAFE: We have ongoing collaborations with scientist in major institutions
and laboratories worldwide as can be deduced from our list
of publications. Just to mention a few: CERN, Fermilab (USA), 
INFN-Frascati (ITALY), Gran Sasso (Italy), Max Planck Institute (Germany),
Stony Brook Univ (USA), U. Wisconsin (USA), and, U. Sao Paulo (Brazil). 

\section{\bf \textsf{Collaborations with companies}}

8.-Relate the collaborations with companies or other socioeconomic sectors and the added value for the project, the knowledge transference or other results

\section{\bf \textsf{UE FP7 calls}}

9.-Indicate if you have attended, and how effective has been, to any of UE FP7 calls (project, training, infrastructures) and/or to other international programs in topics related to this project. Specify the program, partners, countries and topics and, where appropriate, funding received

\section{\bf \textsf{National and international Program visibility}}

10.-National and international Program visibility

CAFE: The members of CAFE been regularly reporting the results of their 
studies in  conferences and workshops. In particular they have been invited to
be review talks on neutrino physics in the most important conference
in the area. For example:
\begin{itemize}
\item Prof. M.C. Gonzalez-Garcia gave the plenary talk on Theory of Neutrino
Physics in the ICHEP Conference in Melbourne in July 2012
\item Prof. P. Hernandez gave the plenary talk on Theory of Neutrino Physics
at the HEP-EPS Conference in Vienna in July 2015 
\item Prof. M.C. Gonzalez-Garcia gave the plenary talk on Neutrino Physics
at the  Astroparticle Physics Joint  TeVPA/IDM Conference, Amsterdam, 
Netherlands in June 2014
\end{itemize}


\section{\bf \textsf{Problems and suggestions}}

11.- Problems and suggestions
%\bibliographystyle{elsarticle-num}

\end{document}

