
\documentclass[a4paper,11pt,oneside]{article}
\usepackage[a4paper,vmargin={1.5cm,1.5cm},width=16cm]{geometry}
%\usepackage[style=verbose-inote,doi=false,sortcites=true,block=space, backend=bibtex]{biblatex}
\usepackage[utf8]{inputenc}
\usepackage{textcomp}
\usepackage[spanish]{babel}
\usepackage{microtype}
\usepackage{lmodern}
\usepackage{graphicx}
\usepackage{fancyhdr}
\usepackage{booktabs}
\usepackage{eurosym}
\usepackage{mathptmx}
\usepackage[T1]{fontenc}
\usepackage{hyperref}
%% Added to help mimic structure.
%\usepackage[many]{tcolorbox}
\usepackage{tcolorbox}
\usepackage{soul}
\usepackage{color}
\usepackage{lastpage}

%\documentclass[11pt,a4paper]{article}
%\usepackage[utf8]{inputenc}
%\usepackage[spanish]{babel}
%\usepackage[a4paper,height=26cm,width=17cm]{geometry}
%\usepackage{graphicx} d
%\usepackage{epsfig,rotating}
%\usepackage{amssymb}
%\usepackage{mathrsfs}
%\usepackage{amsmath}
%\usepackage{amsfonts}
%\usepackage{multirow}
%\usepackage{eurosym}
%\usepackage{dcolumn}
%\usepackage{cite}

\usepackage{multirow}
%\usepackage{enumitem}

%\usepackage[hyphenbreaks]{breakurl}
%\usepackage[hyphens]{url}

\usepackage{hyperref}
\hypersetup{breaklinks=true}


%\usepackage[hyphens]{url}



%% Hack to make math formulas bold in section titles
\makeatletter
\DeclareRobustCommand*{\bfseries}{%
  \not@math@alphabet\bfseries\mathbf
  \fontseries\bfdefault\selectfont
  \boldmath
}
\makeatother

\begin{document}

% BB
\newcommand{\bb}{\ensuremath{\beta\beta}}
% BB0NU
\newcommand{\bbonu}{\ensuremath{\beta\beta0\nu}}
% BB2NU
\newcommand{\bbtnu}{\ensuremath{\beta\beta2\nu}}
% NME
\newcommand{\Monu}{\ensuremath{\Big|M_{0\nu}\Big|}}
\newcommand{\Mtnu}{\ensuremath{\Big|M_{2\nu}\Big|}}
% PHASE-SPACE FACTOR
\newcommand{\Gonu}{\ensuremath{G_{0\nu}(\Qbb, Z)}}
\newcommand{\Gtnu}{\ensuremath{G_{2\nu}(\Qbb, Z)}}

% mbb
\newcommand{\mbb}{\ensuremath{m_{\beta\beta}}}
\newcommand{\kgy}{\ensuremath{\rm kg \cdot y}}
\newcommand{\ckky}{\ensuremath{\rm counts/(keV \cdot kg \cdot yr)}}
\newcommand{\mbba}{\ensuremath{m_{\beta\beta}^a}}
\newcommand{\mbbb}{\ensuremath{m_{\beta\beta}^b}}
\newcommand{\mbbt}{\ensuremath{m_{\beta\beta}^t}}
\newcommand{\nbb}{\ensuremath{N_{\beta\beta^{0\nu}}}}

% Qbb
\newcommand{\Qbb}{\ensuremath{Q_{\beta\beta}}}

% Tonu
\newcommand{\Tonu}{\ensuremath{T_{1/2}^{0\nu}}}

% Tonu
\newcommand{\Ttnu}{\ensuremath{T_{1/2}^{2\nu}}}

% Xe-136
\newcommand{\Xe}{\ensuremath{^{136}}Xe}
\newcommand{\COT}{\ensuremath{CO_2}}
\newcommand{\CHF}{\ensuremath{CH_4}}
\newcommand{\CFF}{\ensuremath{CF_4}}

% 2S
\newcommand{\TwoS}{\ensuremath{^{2}S_{1/2}}}

\newcommand{\TwoP}{\ensuremath{^{2}P_{1/2}}}

\newcommand{\TwoD}{\ensuremath{^{2}D_{3/2}}}


% Xe-136
\newcommand{\CS}{\ensuremath{^{137}}Cs}

% Xe-136
\newcommand{\NA}{\ensuremath{^{22}}Na}


% Bi-214
\newcommand{\Bi}{\ensuremath{^{214}}Bi}

% Tl-208
\newcommand{\Tl}{\ensuremath{^{208}}Tl}

% Pb-208
\newcommand{\Pb}{\ensuremath{^{208}}Pb}
% Pb-208
\newcommand{\PBD}{\ensuremath{^{210}}Pb}

% Po-214
\newcommand{\Po}{\ensuremath{^{214}}Po}
\newcommand{\Kr}{\ensuremath{^{83}}Kr}

% bru
\newcommand{\bru}{cts/(keV$\cdot$kg$\cdot$y)}
\newcommand{\dten}{10 mm/$\sqrt{\rm m}$}
\newcommand{\dtwo}{2 mm/$\sqrt{\rm m}$}
\newcommand{\BAPP}{\ensuremath{Ba^{++}}}
\newcommand{\BAP}{\ensuremath{Ba^{+}}}

\newcommand{\HPXE}{\sc{HPXe}\rm}
\newcommand{\BATA}{\sc{BaTa}\rm}

% Saltos de carro en tablas
\newcommand{\minitab}[2][l]{\begin{tabular}{#1}#2\end{tabular}}

\newcommand{\thedraft}{0.1.1}% version for referees

\newcommand{\MO}{\ensuremath{{}^{100}{\rm Mo}}}
\newcommand{\SE}{\ensuremath{{}^{82}{\rm Se}}}
\newcommand{\ZR}{\ensuremath{{}^{96}{\rm Zr}}}
\newcommand{\KR}{\ensuremath{{}^{82}{\rm Kr}}}
\newcommand{\ND}{\ensuremath{{}^{150}{\rm Nd}}}
\newcommand{\XE}{\ensuremath{{}^{136}\rm Xe}}
\newcommand{\GE}{\ensuremath{{}^{76}\rm Ge}}
\newcommand{\GES}{\ensuremath{{}^{68}\rm Ge}}
\newcommand{\TE}{\ensuremath{{}^{128}\rm Te}}
\newcommand{\TEX}{\ensuremath{{}^{130}\rm Te}}
\newcommand{\TL}{\ensuremath{{}^{208}\rm{Tl}}}
\newcommand{\CA}{\ensuremath{{}^{48}\rm Ca}}
\newcommand{\CO}{\ensuremath{{}^{60}\rm Co}}
\newcommand{\PO}{\ensuremath{{}^{214\rm Po}}}
\newcommand{\U}{\ensuremath{{}^{235}\rm U}}
\newcommand{\CT}{\ensuremath{{}^{10}\rm C}}
\newcommand{\BE}{\ensuremath{{}^{11}\rm Be}}
\newcommand{\BO}{\ensuremath{{}^{8}\rm Be}}
\newcommand{\UDTO}{\ensuremath{{}^{238}\rm U}}
\newcommand{\CD}{\ensuremath{^{116}{\rm Cd}}}
\newcommand{\THO}{\ensuremath{{}^{232}{\rm Th}}}
\newcommand{\BI}{\ensuremath{{}^{214}}Bi}
\newcommand{\FDG}{\ensuremath{^{18}}F}


\begin{center}
{\LARGE \bf \textsf{CONSOLIDER-INGENIO 2010 PROGRAMME }} \\ \vspace{0.5cm}
{\Large \bf \textsf{FINAL SCIENTIFIC ANNUAL REPORT}} \\ \vspace{0.5cm}
{\Large \bf \textsf{I01/01/2012 al 31/12/2012 }} \\ \vspace{2cm}
\end{center}


\begin{table}[htp!]
\begin{center}
\begin{tabular}{|c|c|}
\hline
PROJECT REFERENCE NUMBER: & CDS2008-00037\\
\hline
Coordinating Researcher:  & M.C. G\'onzalez-Garc\'ia \\
Project Title: & Canfranc Underground Physics (CUP) \\
Managing Institution:  & CSIC \\
Project Start Date:  & 15/12/2008\\
Project Final Date: &  4/12/2015)\\
\hline\hline
\end{tabular}
\end{center}
\caption{Project data}
\label{default}
\end{table}%

\newpage

%%%%%%%%%%%%%%%%%%%%%%%%%%%%%%%%%%%%%%%%%%%%%%%%%%%%%%%%%
%% SECTION 1
\section{\bf \textsf{CUP: summary and main goals }}
\label{sec.advances}
CUP (Canfranc Underground Physics) was granted with the main goal of developing the scientific program of the
 \emph{Laboratorio Subterr\'aneo de Canfranc} (LSC), one of the ICTS (instalaci\'on cient\'ifico t\'ecnica singular) of the ministry of science (currently integrated in the MINECO). The two main activities of CUP were called
 CAFE (Canfranc Future Experiment) and NEXT (Neutrino Experiment with a Xenon TPC).

CAFE has spawn a wide range of phenomenological studies, coordinated by professor 
M.C. Gonz\'alez-Garc\'ia. These studies have involved several institutions of the CUP consortium, including the Universidad de Barcelona (UB), Instituto de F\'isica Corpuscular, (IFIC), and Instituto de F\'isica Te\'orica (IFT). 

The goals of the subproject CAFE were:
\begin{enumerate}
\item The characterisation of LSC to demonstrate its viability 
as a large underground European laboratory in the context of
the European project LAGUNA (Large Apparatus for Studying Grand
Unification and Particle Astrophysics) as well as in the context
of studies of viability for future neutrino factories which were
performed as part of the European project EURO-$\nu$.
\item Perform phenomenological studies to explore and exploit 
the physics potential LSC  as well more broad studies related
to astroparticle physics in general (neutrino physics, cosmology, etc.)
\end{enumerate}

%The CAFE activities were amply described in our previous reports, and here we will simply update the lists of publications related with the activity. 

%Two major spinoffs of the activity are two European projects directly related with CAFE goals: 
%
%\begin{enumerate}
%\item 
%A european training network, called INVISIBLES, coordinates by professor B. Gavela, from IFT. 
%%
%\item A european project of large infrastructures called LAGNA-LBNO. The spanish IP of the project is  professor J.J. Gómez-Cadenas, for IFIC, who acts also as spokesperson of NEXT.
%\end{enumerate}

CUP main goal was the construction, commissioning and operation of the NEXT detector\footnote{\url{http://next.ific.uv.es/next/}}, a high-pressure, xenon (HPXe) Time Projection Chamber (TPC), whose goal was to search for neutrinoless double beta decay  (\bbonu) events in xenon enriched at 90\% in the isotope \XE. The first phase of the experiment, called NEW, deploying 10 kg of xenon, is currently being commissioned at the LSC.The second phase of the experiment, NEXT-100, will deploy 100 kg of xenon. A third phase, deploying up to one ton of xenon, is actively being discussed, with strong international interest.
 
%
%The discovery potential of NEXT is very large. It combines four desirable features that make it an almost-ideal experiment for \bbonu\ searches, namely: 
%\begin{enumerate}
%\item Excellent energy resolution (better than 1\% FWHM in the region of interest).
%\item A topological signature (the observation of the tracks of the two electrons).
%\item A fully active, very radiopure apparatus of large mass.  
%\item The capability of extending the technology to much larger masses.
%\end{enumerate}
%
%The project has evolved very satisfactorily, from the initial Letter of Intent (LOI) in 2009 to the Technical Design Report (TDR) in 2012. A substantial number of papers, proceeding reports and conferences, documenting and demonstrating the physics case, the results of the prototypes and the technological choices have been published and are attached to this document. They can also be found in the NEXT web page: \url{http://next.ific.uv.es/next/talks.html}.
%



\section{\bf \textsf{Main advances and profits }}
\label{sec.advances}
CAFE has result into several advances
\begin{enumerate}
\item {\bf It has enhanced the visibility of the LSC} by participating into
European studies devoted to the long term planning for large infrastructures
in Europe such as of such as LAGUNA-LBNO and Euro-$\nu$.
\item {\bf It has contributed to the  creation of a new generation of European physicists}
with expertness in phenomenology of relevance for underground physics
by our participation in two consecutive International Training Networks
and a Network for Research and Innovation Staff Exchange, all
funded by the European Commission, as reported below. 
\item  {\bf It has  produced a large number of scientific articles, which have
advanced quantitatively the state-of-the-art of the field.} The total
number of references of published articles by the CAFE senior members
in international peer reviewed journals 
since 2008 is 70  (listed below) that  have  received a total of 
more than 3000 citations in the SPIRES data basis which is the one commonly used
in our field for reference.
\end{enumerate}

NEXT has result in several advances:
\begin{enumerate}
\item {\bf It has advanced the field of underground physics instrumentation}, with the construction of the NEXT detector series. The large prototype which has operated at IFIC since 2011 (NEXT-DEMO) was the largest HPXe electroluminescent TPC in the World, until the construction of NEW, currently being commissioned at the LSC. The detectors have shown the feasibility of the electroluminescent technology, have demonstrated the excellent energy resolution that can be achieved in xenon gas, and have demonstrated  the availability of a unique topological signature in HPXe, e.g, the capability to fully reconstruct the two electrons emitted in \bb\ decay, providing a powerful tool to discriminate against backgrounds. 
\item {\bf It has resulted in a cutting-edge experiment}, which will be searching for \bbonu\ decays during the next few years, with chances of making or contributing to a major discovery. 
\item {\bf It has bought to the field a new technology}, that of the HPXe-EL, which is currently considered one of the best candidates for the next generation of \bbonu\ experiments.
 \item {\bf It has resulted in an international collaboration}, involving groups from Portugal, United States, Russia and Colombia. Other groups (Poland, Australia), are currently considering joining the experiment.
 \item {\bf It has attracted external funding}, with contributions to the experiment from all the members of the collaboration, and in particular from the US.
  \item {\bf It has resulted in an Advanced Grant of the ERC}, granted to the spokesperson of NEXT, prof. J.J. G\'omez Cadenas, who has acted as co-PI of CUP. 
   \item {\bf It has boosted the scientific interest of the LSC}, as intended in our proposal. Currently NEXT is the most important experiment in the Canfranc Underground Laboratory and the only one with truly international projection. 
  \item {\bf It has produced a major scientific spinoff, called PETALO}, a new concept for a full-body, TOF-capable PET using xenon. Currently, a patent has been filled and two scientific papers have been published on the new concept. A new collaboration is being formed to explore the scientific potential and commercial applications of PETALO. 
   \item {\bf It has contributed in an important way to science outreach}. See for example:
   \url{http://www.jotdown.es/2012/09/david-nygren-y-alessandro-bettini-the-physics-as-fountain-of-eternal-youth/}{link1}
   \url{http://www.jotdown.es/2013/11/coversaciones-de-fisica-en-el-santa-cristina-ariella-cattai-y-concha-gonzalez-garcia/}
\end{enumerate}


\section{\bf \textsf{ Key Indicators}}
\label{sec.indicators}

\subsection*{NEXT: Reports from the LSC Scientific Committee}

In the CUP proposal, we stated:
\begin{quotation}
We propose a very simple and efficient evaluation scheme for CUP. The  activities are focused in the LSC, which has an international scientific committee, composed by high-reputation specialists in underground physics. This committee is perfectly suited to evaluate our progress in all the areas proposed.
\end{quotation}

The reports of the LSC Scientific Committee (LSCSC) have been consistently encouraging and enthusiastic towards NEXT. They represent a continuous and methodic follow up of a very difficult and complex project, and they
also keep a detailed historical track of the steady progress of the experiment. At the same time, the reports reflect the concerns of the LSCSC regarding a perceived lack of domestic support to the project, in particular during 2011 and 2012.
It is important to remark that the LSCSC is an independent body, formed almost exclusively by experts of international reputation. This adds tremendous weight to the opinions, recommendations and commendations expressed in their reports and permits an objective evaluation of the degree of success of the project. 

{\em
In appendix A, we offer a detailed discussion of the LSC reports. 
}

\subsection*{Other key indicators for NEXT. Publications, international projection and grants}

\begin{itemize}
\item {\bf Publications}: the project has generated a large number of publications in refereed journals and conferences, as well as posters and communications. A full list of publications can be found online\footnote{\url{http://next.ific.uv.es/next/talks.html}}.
\item{\bf International projection}: NEXT is a CERN recognised experiment and is included among the most important
\bbonu\ experiments in Europe. See for example the recent presentation of A. Giuliani in the APPEC meeting \footnote{\url{http://app2016.in2p3.fr/programme.html}}.
\item{\bf Grants:} The spokesperson of NEXT has obtained and Advanced Grant of the ERC for the NEXT project. 
\end{itemize}

\subsection*{CAFE: key indicators}

%\begin{quotation}
%As for the CAFE activity, there is an obvious evaluation criteria, namely the scientific papers to be produced along the five years of the project, including the proceedings of the workshop(s) and/or schools as well as other outreach material.
%\end{quotation}
%
The key indicators of the final results of CAFE are the following:

\begin{enumerate}
\item Involvement in the project LAGUNA-LBNO
(\url{http://laguna.ethz.ch:8080/Plone}). This project continued the
studied started by LAGUNA with the aim at developing large infrastructures
in Europe. Spain participated with four associates LAGUNA-LBO, three of
which are participants in CUP (UAM, IFIC y el propio LSC).
\item Involvement of our scientists in the european project EURO-$\nu$.
In particular de group at UAM made very relevant contributions to
this project as listed in the bibliography.
\item Participation in three funded projects of the European Commission
for the training of young researchers in fields directly related
with underground physics. They are: 
\begin{itemize}
\item 
Title: INVISIBLES\\
Initial year: 2012 Year final: 2016\\
Reference: PITN-GA-2011-289442\\
\item 
Title: 	Elusives\\
Initial year: 2016 Year final: 2020\\
Reference: H2020-MSCA-ITN-2015-674896 \\
\item  Title: 	InvisiblesPlus\\
Initial year: 2016 Year final: 2020\\
Reference: H2020-MSCA-RISE-2015-690575\\
\end{itemize}
The coordinator of the UB node of these networks is our IP, Prof.
M.C. Gonzalez-Garcia while the coordinator of the node at IFIC
is Prof. P. Hernandez, a member of CAFE/CUP.
\item Publication in peer review international journal 
of more than 70 articles in phenomenological studies of which 
we list a selection in the
bibliography. They have accumulated more than 3000 citation  in the
basis SPIRES. For example our updated determinations of the neutrino 
parameters, fundamental for any study of the expected signal at NEXT
\cite{Gonzalez-Garcia:2014bfa,GonzalezGarcia:2012sz,GonzalezGarcia:2010er} 
have accumulated more than 1000 citations,  
while our contributions to studies associated to future neutrino facilities,
such as Ref.~\cite{Bandyopadhyay:2007kx,Abazajian:2012ys,Choubey:2011zzq,Adey:2014rfv} have received more than 500 references (all data from SPIRES
Data basis).
\end{enumerate}

%


\section{\bf \textsf{ Difficulties}} 
\label{sec.diff}
The project went through a difficult period which extended from 2011 to late 2013. The difficulties started with the decision, in early 2011 to adopt the electroluminescence (EL) technology as the baseline for NEXT. Such a decision came after a long internal debate in the collaboration and was firmly rooted in scientific arguments. It has indeed proven to be correct, as discussed with more detail in Appendix B. The decision was supported by the LSC Scientific Committee, who wrote, in its May-2011 report.

\begin{quotation}
The committee congratulates the NEXT collaboration on very substantial progress made towards the definition of the conceptual design for the xenon double beta decay experiment. A major decision has been the choice of electroluminescence as the gain mechanism [...]

The committee concurs with the need to move expeditiously to build the detector if the results are to be competitive. The committee feels that the choice of electroluminescence is likely to lead to energy and tracking measurements adequate for the project.
\end{quotation}
 
The choice of EL was not only sound from the scientific point of view and supported by the LSCSC, but it was backed by the overwhelming majority of the collaboration. The alternative to EL was the proposal to use a gain-based TPC, with a Micromegas readout. 
%The main reason why this proposal was considered not competitive with EL was energy resolution, which had been demonstrated to be considerably worse than that of EL (3--4 \% FWHM to be compared with 0.5--0.7 \% FWHM, both at \Qbb, see Appendix B for details). 
%
The decision mechanism was based in the elaboration of Conceptual Design Report Proposals (CDRP). 
The Micromegas (MM) CDRP was signed by the group of Zaragoza and by the group of IFAE. The EL CDRP was signed by the rest of the collaboration (IFIC, UPV, US, IFT-UAM, UG, LBNL, TAMU and UAN) with the exception of CIEMAT which chose not to sign any of the two CDRPs. The two proposals were intensely debated inside the collaboration and the EL CDRP was endorsed by overwhelming majority of the board members. 

After the decision of adopting EL, the IFAE group opted out of the NEXT collaboration, sending a letter to the board which stated their lack of trust in the technology, the schedule and the management of the collaboration. For the soundness of the technology we refer to Appendix B. Regarding the aggressive schedule that the collaboration was attempting, we quote the report of the LSCSC (December 2011):

\begin{quotation}
The NEXT collaboration {\em has made tremendous progress} in defining the gaseous xenon detector for double beta decay. The conceptual design has been improved in many respects and prototype detectors of moderate scale have been successfully operated proving the main concepts of the proposed design. Excellent resolution has been demonstrated for 137Cs gamma rays at 662 keV \footnote{Relevant references: \cite{Alvarez:2012yxw, Alvarez:2012zsz,Alvarez:2012hu,Alvarez:2013gxa,Lorca:2014sra}} and good tracking has also been shown \footnote{Relevant references:  \cite{Ferrario:2015kta}}. {\em The committee strongly supports the need to expedite this project} as there are now 2 operating xenon detectors in the world and the scientific impact will be greatly reduced if the NEXT project cannot be completed on their proposed timescale.
\end{quotation}

Regarding managerial aspects, we quote again the LSCSC (May 2012, see Appendix A):

\begin{quotation}
{\em The NEXT Collaboration presented enormous progresses both on technical and on managerial domains} [...]

Managerial aspect: We congratulate the Collaboration for presenting a very convincing work-breakdown structure with convincing work packages (WP).
\end{quotation}

The withdrawal of the IFAE group was followed by the withdrawal of the CIEMAT group, after they decline to sign the NEXT Technical Design Report.
%\footnote{NB: with hindsight, it still appears that the withdrawal of IFAE was unavoidable, given their position, extremely critical of the NEXT management. Perhaps, however, the CIEMAT group could have been kept in NEXT by negotiating some temporary solutions that would have allowed them to delay the signature of the TDR for a given period of time.}.
%
Notice, on the other hand, that neither IFAE, nor CIEMAT withdrew formally from CUP. However, their activity in the consortium stopped. 

In 2011, the IFIC, UPV and UAM presented a join funding proposal to the open call for science projects, within the area of particle physics (FPA, for short). 
Notice that it was recognised since the beginning of CUP, that the CUP consolider grant alone could not supply  enough  funds for the construction of the NEXT- 100 detector and the associated infrastructure,
together with the salaries that were needed to create a hitherto non-existing group of experts at the leading institutions. Without additional funds from FPA, the construction of the needed infrastructures and the large underground detector was not realistic.

The FPA committee in 2011 included among the panel of experts, several leaders of IFAE and CIEMAT groups. It could be argue, therefore, that objectivity was not fully granted. 
%In fact, F. Sanchez presented the IFAE arguments to withdraw from NEXT during his public exposition {\em of a different research proposal}\footnote{The transparencies
%are available at \url{https://www.fpa.csic.es/docs/2011Exp/29823-C02-02\_Sanchez\_Nieto.pdf}}. Remarkably, as it can be followed in such presentation, Sanchez described the R\&D conducted at IFAE for NEXT, {\em which had focused in the EL option, not in the MM option}. Indeed, IFAE (and Sanchez) had focused their research in an EL detector, identical to the one finally approved, with the only difference that they proposed the use of Avalanche Photodiodes (APDs) for the tracking plane, while IFIC, LBNL, TAMU and others proposed the use of silicon fotomultipliers (SiPMs). The choice of one type of sensor or the other could be considered a relatively minor technological issue (and in fact, the evolution of technology has demonstrated without doubts that the choice of SiPMs, nowadays omnipresent in particle physics and medical imaging, while APDs have practically disappeared from the market was correct). Therefore it is hard to understand the decision to sign the MM proposal (the IFAE had worked since the beginning of the project in the EL option), or the criticism of a technical choice that was, in fact supported by their work, except for the minor detail of the choice of sensor. 
%
The join IFIC-UPV-UAM proposal submitted to FPA in 2011 obtained only a ``C'' (mediocre) and was rejected without funding. Regarding the possible bias of the FPA panel of experts and the referees designed by the ANEP coordinator, we offer the following remarks:

\begin{itemize}
\item There is no reference anywhere in the inform of the panel of experts to the very positive reports produced by the LSCSC.
\item The coordinator of the ANEP for FPA was the head of the experimental division at CIEMAT. The coordinator of ANEP has the role to request independent evaluators for a given project. However, at least one of the reports was extremely biased and included numerous subjective statements and attacks {\em at hominem}\footnote{We quote literally the opening statement of such report: {\em El IP tiene un historial lleno de luces y de sombras, y sus logros cient\'ificos no est\'an tan claros como \'el expresa. \'Ultimamente dedica m\'as tiempo a la escritura de novelas u otros libros que a ocuparse de los experimentos y de los consolider en los que tiene muchas responsabilidades.} The translation to English reads: {\em The PI has a trajectory full of lights and shadows and his accomplishments are not so obvious as he states. Lately he devotes more time to writing novels and other books than to care about the experiments en projects of which he is responsible}.}. But the report was not withdrawn by the coordinator and no additional reports were requested.
\item The panel of experts included also leading scientists from IFAE. Without discussing the obvious scientific merit of those scientists, one can raise reasonable doubts about a potential bias related with the situation. 
%In particular, Dr. Cavalli-Svorza has made clear {\em publicly} (the last time in a seminar at the ALBA installation, in 2015) his clear adversity towards the spokesperson of NEXT.
%\item The external expert was Dr. Giorgio Gratta, the spokesperson of EXO, which is a US-based experiment proposing to use liquid xenon, rather than an HPXe detector. Again, in spite of the prestige and clear scientific merits of Prof. Gratta, it is also a fact that EXO and NEXT are direct competitors, and one could have imagined other external referees with  less potential bias.  
\item The external expert was Prof. Giorgio Gratta from Stanford University,
an internationally recognised expert on Xe detectors. However, prof. Gratta
happens to be  the spokesperson  of EXO, a US-based experiment proposing to use liquid xenon to pursue a physics program very similar to that of NEXT.  So EXO and NEXT can be seen as direct competitors, and one could have imagined other external referees with  less potential bias.  
\end{itemize}
 
The project submitted in 2011 requested essential additional funding, not covered by CUP. The negative result implied a major logistic drawback and a serious blow to the morale of the collaboration. It also implied loosing international credibility, since from the point of view of international agencies (in particular DOE in the USA) it was unclear whether NEXT was being supported by the Spanish scientific authorities or not. It can be argued that the prospects of international funding and of expanding the NEXT collaboration suffered from the negative FPA decision in 2011. 

In 2012, a new research project was submitted, to a different committee (general physics, FIS), given the perceived bias of the FPA coordinator and his panel of experts. This time, the proposal  obtained a B (``Good'') and modest funding for 2 years. In 2013, the spokesperson of NEXT was granted the first (and to this day the only one in the area of FPA) advanced grant of the ERC. The process of selecting a project for an AdG grant implies the pair review of numerous international experts (in 2013 there were 8 independent referees evaluating NEXT) and requires marks above 90\% to be selected. In 2014, another project was submitted to FIS for 4 years. The project obtained an A (``excellent'') and obtained sizeable funding, in spite of the limited budget available. 

The problem may not be fully solved yet, since the NEXT project is still funded by an area other than FPA, a totally illogical situation, given the well established merits of the experiment and its relevance for the future of the LSC. NEXT leading groups will apply again for funding in 2018.  

%Clearly, there is an inconsistency which points towards a bias. In 2015, the gestor of FPA has changed. However, the perception of the collaboration is that the situation is not fully solved yet. While the ERC funds and the funds granted in 2014 has given the Spanish groups enough solvency to carry forward the project up to 2019, the uncertainties regarding   funding once the AdG and the current project expire remain. The collaboration faces the real possibility that insufficient funding in 2019 hinder our competitiveness precisely in a moment where Europe will be, very likely, considering what are the candidates for a ton-scale experiment. NEXT has the real chance of becoming one of the leading \bbonu\ experiments in Europe in the NEXT decade, projecting consequently the relevance of the LSC, {\em but only if there is firm, clear and sustained support to the project from the side of the Spanish scientific authorities.} 
 

%The rejection of the research proposal in 2011 and the very scarce funding in 2012, forced the collaboration to redefine its initial plan to start NEXT-100 construction in 2012. Instead, it was decided to introduce a first-stage experiment, NEW (a detector with half of the size of NEXT-100, a fourth of the sensors and 10 kg of xenon mass). It was perceived that there was not sufficient support in Spain to attempt the construction of NEXT-100 directly. After the ERC grant the situation has improved considerably, but the collaboration decided to continue with the NEW program. Currently, NEW is starting data taking at the LSC, and the lessons learned in its construction (and certainly by operating it) are very valuable to build a better NEXT-100 detector. 
%


\section{\bf \textsf{ Scientific and technical activities }} 
\label{sec.science}
\subsection*{NEXT: Scientific and technical progress}

The NEXT experiment has evolved, largely thanks to the support of the Consolider CUP, from a visionary idea ({\em to setup an international experiment at the LSC able to be in the forefront of the Underground Physics field, with a high discovery potential}) to an established reality. After an initial period of R\&D (2010 and the first half of 2011), the project has evolved in a consistent way, first with the construction, commissioning and operation of NEXT-DEMO (2011--2013), which has produced a large scientific and technical output, then with NEW which started in earnest in 2014 and will carry its scientific program through 2018. The current plan is to present an update TDR in November of 2016, and start the construction of NEXT-100 in mid/late 2017 or 2018, depending on the available human resources and funding. 
 
 NEXT-DEMO and NEXT-DBDM were essential to fully demonstrate the performance of the HPXe-EL technology, including:
 \begin{enumerate}
\item An energy resolution near the Fano-Factor ($\sim$ 0.5 \% FWHM at \Qbb)\cite{Alvarez:2012yxw,Alvarez:2012zsz}.
\item First operation in the World of a tracking plane made of SiPMs \cite{Alvarez:2013gxa}.
\item Demonstration of the unique topological signature available in HPXe\cite{Ferrario:2015kta}.
\end{enumerate}

During the NEW phase of the project, the collaboration has addressed both scientific and technical challenges. In the scientific front:

\begin{enumerate}
\item Understanding of the radioactive budget \cite{Cebrian:2015jna, Alvarez:2014kvs, Dafni:2014yja}.
\item Definition of the detector sensitivity \cite{Martin-Albo:2015rhw}.
\item Demonstration of the unique topological signature available in HPXe\cite{Ferrario:2015kta}.
\end{enumerate}

And in the technical front:

\begin{enumerate}
\item Construction and certification of a large underground detector (NEW) for underground operation.
\item Construction and certification of large infrastructures (gas system, high-voltage system) for for underground operation.
\item Development of slow controls.
\item Risk analysis.
\item System integration.
\end{enumerate}

With the acquired experience, the collaboration is looking forward to an steady operation of NEW followed by the
timely construction, commissioning and operation of the NEXT-100 detector. The NEXT program is one of the major alternatives in Europe for a future, ton-scale, \bbonu\ experiment. 


\subsection*{CAFE: Scientific activity}

The scientific activity of CAFE has been described above. We list
here the main lines developed in their phenomenological studies
which are directly related to underground physics
\begin{itemize}
\item Precise and updated characterization of the leptonic flavour
parameters, whose values are crucial to the prediction of expected
event rates in neutrinoless double beta decay experiments
\cite{Song:2015xaa,Bergstrom:2015rba,Gonzalez-Garcia:2014bfa,Gonzalez-Garcia:2013usa,GonzalezGarcia:2012sz,GonzalezGarcia:2011my,GonzalezGarcia:2010un,GonzalezGarcia:2010er}
\item Studies of new phenomena expected from presence of additional
sterile neutrinos and their effect at future underground experiments
\cite{Vincent:2014rja,Bergstrom:2014fqa,Adey:2014rfv,GonzalezGarcia:2012yq,SolagurenBeascoa:2012cz,Donini:2011jh,Abazajian:2012ys}
\item Studies directly related to future underground facilities
in Europe 
\cite{Bross:2013oua,Edgecock:2013lga,Blennow:2013swa,Bayes:2012ex,Agarwalla:2012zu,Hernandez:2012zz,Coloma:2012wq,Coloma:2011rq,Donini:2010xk,Coloma:2010wa,Choubey:2009ks,Bandyopadhyay:2007kx}
\item Potential of underground experiments to detect and 
characterize astrophysical neutrino sources 
\cite{Bergstrom:2016cbh,Gonzalez-Garcia:2013iha,Ahlers:2011jj,Ahlers:2010fw,GonzalezGarcia:2009ya,GonzalezGarcia:2009jc}
\item Study of theoretical frameworks for the connection  between Majorana 
nature of the neutrino (as searched for
in neutrinoless doublebeta decay) and the production of the
observed matter-antimater in the Universe
\cite{Hernandez:2015wna,Fong:2013gaa,Fong:2011yx,Fong:2010bv,Fong:2010bh,Fong:2010qh,Fong:2010zu,GonzalezGarcia:2009qd,Fong:2009iu}
\item Construction of models which can account for all neutrino observations
and study their specific predictions in future experiments
(including neutrinoless doublebeta decay experiments)
\cite{Gago:2015vma,Hernandez:2014fha,Hernandez:2013lza,Donini:2012tt,Eboli:2011ia,Gavela:2009cd}
\item Studies of scenarios of electroweak symmetry breaking related
for mass generation
\cite{Corbett:2015lfa,Corbett:2014ora,Gavela:2014vra,Brivio:2014pfa,Brivio:2013pma,Corbett:2013pja,Corbett:2012ja,Corbett:2012dm,Eboli:2011ye,Eboli:2011bq,Eboli:2010qd,Alves:2009aa}
\end{itemize}
%


\section{\bf \textsf{Collaborations between team members}}
\label{sec.coll}

\subsection*{NEXT}

The scientific and technical activities of NEXT are carried out in the context of an international collaboration which includes the members of the CUP consortium but also other international members, such as the groups from Portugal, Colombia and the US. 

The leadership of the collaboration rests largely in Spanish groups, and in particular in the leading groups in the CUP consortium. Specifically:

\begin{itemize}
\item The spokesperson (J.J. Gomez-Cadenas) is a member of the IFIC group.
\item The hardware project manager (J.F. Toledo) is a member of the UPV.
\item The software project manager (J.A. Hernando) is a member of the U. Santiago (US).
\item The technical coordinator (F. Monrabal) is a member of U. of Texas at Arlington and a former Ph.D. student of NEXT.
\item The analysis convener (M. Sorel) is a member of IFIC.
\item The software coordinator (P. Novella) is a member of the IFIC. 
\item The radiopurity coordinator (S. Cebrian) is a member of U. Zaragoza.
\item The integration manager (J. Torrent) is a member of U. Zaragoza.
\item The DAQ coordinator and the Slow Control coordinators are member of the UPV.
\item The project leaders of detector construction are members of IFIC and UPV.
\end{itemize}


\subsection*{CAFE: Scientific activity}

As can be seen from our publication list there are regular
collaborations between the scientists in UB, IFIC and UAM. 
The members in the three groups in CAFE are all involved in the European
networks mentioned listed above. 


\section{\bf \textsf{Collaborations with other groups of research}}
\label{sec.others}


\subsection*{NEXT}

The international part of the NEXT collaboration has played a very important role in the dynamics of the experiment. In particular, the impact of the US groups cannot be overemphasised. Prof. D. Nygren was the inventor, in the mid seventies, of the TPC technology which constitutes the backbone of both the NEXT and EXO experiment (together with many other experiments in underground physics, nuclear physics and high energy physics). Nygren is a recipient of numerous prices and awards, a long-time distinguished scientist at Berkeley (now holding a chair at U. of Texas at Arlington) and one of the most influential scientist of his generation. The late prof. J. White was a World expert in high-pressure xenon chambers, and his expertise was essential to build the NEXT-DEMO detector.

The Spanish group at IFIC, UPV and elsewhere, benefited enormously from the experience, know-how and ideas of this two leading scientist and their teams. Furthermore, NEXT has been collaborating with scientists from Fermilab national laboratory. NEXT has been presented to the Nuclear Science Committee (NSAC) and has deserved excellent review
\footnote{\url{http://science.energy.gov/np/nsac/
}}

The US groups have contributed also importantly to the funding of the experiment, with man power and contributions in cash and equipment, currently at the level of $\sim$300 k\euro. 

The Portuguese groups have also contributed with their know-how (the groups of Coimbra and Aveiro are recognised experts in the field) and have lead several important studies in connection with energy resolution and gas mixtures. The Russian group from Dubna was essential for the procuration of the 100 kg of enriched xenon owned by the LSC and necessary for the experiment (the current cost of enriched xenon is five time the cost payed in 2011). Last, but not least, the group from A. Nari\~no (Colombia) is contributing in several aspects of software and DAQ. 

{\em The success of NEW and a clear message of support from the Spanish authorities would be instrumental in enlarging the NEXT collaboration}. The construction and full commissioning of the NEXT-100 detector could benefit enormously from a stronger international collaboration, and the costs could be shared among more partners. 

\subsection*{CAFE}

We have ongoing collaborations with scientist in major institutions
and laboratories worldwide as can be deduced from our list
of publications. Just to mention a few: CERN, Fermilab (USA), 
INFN-Frascati (ITALY), Gran Sasso (Italy), Max Planck Institute (Germany),
Stony Brook Univ (USA), U. Wisconsin (USA), and, U. Sao Paulo (Brazil). 

\section{\bf \textsf{Collaborations with companies}}

The NEXT project has relations with a large number of companies. Among these:
\begin{itemize}
\item Hamamatsu Photonics and SENSL, for sensor developments (radiopure PMTs, radiopure SiPMs). We have an on-going collaboration with both companies, in particular to develop ultra-pure SiPMs.
\item Linde, Air Liquide and Swagelock, concerning gas system developments. 
\item ACYM and other pressure vessel manufacturing companies. 
\item SCIENTIFICA and AVS, which are science-related companies with whom we have also collaboration agreements. 
\end{itemize}


\section{\bf \textsf{UE FP7 calls}}
\label{sec.fp7}

\begin{itemize}
\item 
Title: INVISIBLES\\
Initial year: 2012 Year final: 2016\\
Reference: PITN-GA-2011-289442\\
\item 
Title: 	Elusives\\
Initial year: 2016 Year final: 2020\\
Reference: H2020-MSCA-ITN-2015-674896 \\
\item  Title: 	InvisiblesPlus\\
Initial year: 2016 Year final: 2020\\
Reference: H2020-MSCA-RISE-2015-690575\\
\item Title: NEXT \\
Initial year: 2014 Year final: 2019\\
Reference: FP7-Advanced Grant--339787.
\end{itemize}

\section{\bf \textsf{National and international Program visibility}}

\label{sec.visibility}

\subsection*{CAFE}
The members of CAFE been regularly reporting the results of their 
studies in  conferences and workshops. In particular they have been invited to
be review talks on neutrino physics in the most important conference
in the area. For example:
\begin{itemize}
\item Prof. M.C. Gonzalez-Garcia gave the plenary talk on Theory of Neutrino
Physics in the ICHEP Conference in Melbourne in July 2012
\item Prof. P. Hernandez gave the plenary talk on Theory of Neutrino Physics
at the HEP-EPS Conference in Vienna in July 2015 
\item Prof. M.C. Gonzalez-Garcia gave the plenary talk on Neutrino Physics
at the  Astroparticle Physics Joint  TeVPA/IDM Conference, Amsterdam, 
Netherlands in June 2014
\end{itemize}

\subsection*{NEXT}
NEXT presentes regularly its status and results in international conferences, see
\url{http://next.ific.uv.es/next/talks.html}. Prof. Gomez-Cadenas has given numerous international talks and seminars as well as
taught in many international schools. For example, in 2015:

\begin{itemize}
\item Gave a plenary talk on Experimental Neutrino Physics
at the HEP-EPS Conference in Vienna in July 2015.
\item Was convener of the \bbonu\ discussion group at the Aspen Physics Center, August, 2015.
\item Gave an invited plenary talk at the Colloquium Spectroscopicum Internationale (CSI), Septembre 2015.
\end{itemize}

Other talks and seminars:

\begin{itemize}
\item \textit{Ettore Majorana through the looking glass (searching for neutrinoless double beta decay)}, Harvard Monday Colloquium, 22 October 2012; Fermilab Colloquium, 24 October, 2012; University of Wisconsin Madison seminar, 26 October 2012.
%% 2
\item \textit{NEXT, high-pressure xenon gas experiments for ultimate sensitivity to Majorana neutrinos}, invited talk at the 14th International Workshop on Radiation Imaging Detectors (iWoRID 2012), Figueira da Foz, Coimbra (Portugal), 1-5 July, 2012. 
%% 3
\item \textit{Xenon for DM and DBD searches}, invited talk at IDPASC Dark Matter Workshop, \'Evora (Portugal), 2011.
%% 4
\item \textit{Status of the NEXT experiment}, at DBD'11: International Workshop on Double Beta Decay and Neutrinos, Osaka (Japan), 2011.
%% 5
\item \textit{How to probe anti-neutrino = neutrino and the absolute neutrino mass scale}, International Neutrino Summer School, Geneva (Switzerland), 2011.
%% 6
\item \textit{Sense and sensitivity in \bbonu\ experiments} at XIV International Workshop on Neutrino Telescopes, Venice (Italy), 2010.
%% 7
\item {\it Ettore Majorana meets his shadow (searching for neutrino less double beta decay)}, Wednesday colloquium, Weizmann institute, 24 November 2010.
\end{itemize}

NEXT is a recognised CERN experiment. 


\section{\bf \textsf{Problems and suggestions}}

\label{sec.problems}

The Consolider CUP, together with the AdG from the ERC and the support of the SEIDI has made possible to launch the NEXT experiment. NEXT is the flagship experiment at the LSC, as clearly recognised by the LSCSC (see appendix A), and its success will be a tremendous boost for the credibility and visibility of our laboratory. NEXT has a large discovery potential and is considered one of the most relevant technologies for the future of t\bbonu\ searches. Very likely there will be a ton-scale HPXe experiment searching for \bbonu\ events in the next decade. Having launched the NEXT program, Spain is in a perfect position to host such a large-scale experiment, involving a large international collaboration at the LSC, leading the field. 

However, the project still faces difficulties in Spain. A large experiment like NEXT requires a stable funding scheme like the CONSOLIDER program which is, alas, extinct. CUP made possible the creation of groups of young scientists and engineers at the consortium's institutions. In particular the IFIC group includes about 5 post-docs and 5 engineers who play leading roles in the development of the experiment. None of these people have permanent positions. Their salaries are payed either by a SEIDI grant or by the ERC grant. In the medium term this situation is not sustainable. If the IFIC group dismembers due to lack of funds or to lack of long-term prospects for the physicists and engineers working there, the know-how and capability to lead the field will be lost and the experimental push at the LSC will stale. Another opportunity for Spanish science will be squandered. 

NEXT is one of our few cases of an international experiment being led by Spanish physicists and conducted in Spain. NEXT has also attracted considerable external funding and prestige, as is the case also with CAFE. The two projects together participate or have participated in 6 EC projects (LAGUNA-LBO, Euro-$\nu$, INVISIBLES, INVISIBLES-PLUS, ELUSIVES and the AdG). 
%However, NEXT has faced the difficulties already discussed which may not be totally solved. 
In addition NEXT is already producing technological spinoff, with the proposal of PETALO, a PET based in liquid xenon which has generated already a patent and two scientific publications\footnote{See \cite{Petalo2015}, \cite{Gomez-Cadenas:2016mkq}} and is attracting considerable national and international attention. 



In conclusion, we believe that both CAFE and NEXT have fulfilled the promise of excellence associated to the CONSOLIDER program, generating scientific and technological spinoffs as well as international funding. We hope that this effort may be continued successfully in the forthcoming future. We suggest that a long--plan term for the LSC and its experimental effort is needed. 
\newpage  

\section{\bf \textsf{Appendix A: Reports of the LSC Scientific Committee }}
In this section we present the reports of the LSC Scientific Committee (LSCSC) since 2010, together with a discussion concerning the implementations of the committee recommendations. The committee reports are copied verbatim. {\em Emphasis} is used to highlight key aspects of the text, while {\bf bold face} is only used if the committee used it in the original report. Footnotes, often added to specific recommendations, add relevant comments or reference to bibliography. 

The LSCSC is composed by international experts of very high reputation. Among the panel members, the LSCSC includes Prof. F. Avignone (U. of South Caroline, one of the pioneers of \bbonu\ experiments), Prof. D. Sinclair (Carleton University, a World expert in High Pressure Xenon Chambers), prof. Ariella Cattai (CERN, a World expert in instrumentation for High Energy Physics and Prof. Cristiano Galbiatti (Princenton, a World expert in Dark Matter experiments).  

\subsubsection*{2010-April}
\begin{quotation}
LSC Scientific Committee\\
6th Meeting\\
April 15 and 16, 2010\\
EXP-05-2008 (NEXT)\\

The   Committee   {\bf commends   the   Collaboration   for   its   effort   to   improve   its organization, working relationship among members, and its project management. The establishment of a Steering Committee to choose the readout system technology was a positive step. The effort to involve foreign institutions is also laudable. NEXT is making excellent progress;} however the Committee has some recommendations.

The Committee recommends:
\begin{itemize}
\item that the collaboration finalize the selection of a technology choice, a plan of action, and write a schedule,
\item  that the Collaboration design a prototype to bridge from NEXT-1 to NEXT-100, with long drift distances matching those of NEXT-100,
\item that the Collaboration have a final readout system operating under 10 Atm of Xe gas,
that  the  Collaboration  write  a  detailed  plan  for  radioactive  screening  of construction materials
\item  that the Collaboration establish a presence at the Laboratory to determine their needs for counting, etc.,
\item that  the  Collaboration make a study  of  the safety  issues  with  10  Atm.  gas chambers underground, and begin discussions with the Laboratory personnel.
\item that the Collaboration write a Project Management Plan including a budget, schedule and names of persons responsible for specific tasks,
\end{itemize}
\end{quotation}

\subsubsection*{2010-October}
\begin{quotation}
LSC Scientific Committee\\
7th Meeting\\
October 7th and 8th, 2010\\
EXP-05-2008 (NEXT)\\

The Committee applauds the steps taken to further develop the strength of the collaboration, with an increasing commitment of the US groups and the enlargement to groups of the Universidad de Aveiro in Portugal and of the Universidad Antonio Nari\~no in Columbia.

The Committee was pleased by the informative presentations and the new format of the progress report. In general the project is making steady progress in its R\&D phase of the experiment. The most critical topic presented is the choice of the readout technology. The Committee observes that the Micro-Megas readout was tested at the Universities of Saragossa and Coimbra, and achieved an energy resolution of about 3\% at 10-atm pressure to be compared to the needed one, which is better than 1\%. The groups at LBNL, TAMU, IFIC, UPV, Nari\~no, Coimbra and CIEMAT are working on electroluminescence (EL) read out, which, however, has not yet demonstrated full feasibility. The Committee also observes that the Collaboration is facing the problem of non availability of radio-pure PMTs, which can take 10-15 bar pressure, by encasing them into quartz tubes. This feature allows to add a light guide function helping in the light collection. The Collaboration also presented an approach to improving light sensitivity of the SiPMs. 

{\bf
The Committee considers positively the NEXT Collaboration's reply to the recommendation made in its 6th meeting on the choice of readout technology, in particular the plan to present the Conceptual Design Report (CDR) in time for the Committee Meeting in the Spring of 2011. In particular, the Committee understands that a procedure to make the decision on the readout technology will be defined by the Collaboration in advance of that date. The decision should be based on the all available information at that time, and should take into account that experiments of similar sensitivity are already in more advanced stages of readiness.}
 
 The Committee recommended at the last meeting, and still does, that an intermediate detector be built to test the many new elements in the NEXT-100, for example the 10-atm pressure of Xe gas and the long drift distances/long drift times, as well as the readout system operation under these conditions. The Committee also recommends further study of the radon background and radio-purity measurements on construction materials to mitigate against excess background from radioactivity.

The Committee also welcomes the fact that the Collaboration is developing its Work Plan with milestones that will be integrated into the project management plan and looks forward to receiving the project management plan and the CDR before the next meeting. The Committee applauds the Collaboration for progress report with a much-improved format, but which still lacks budget and schedule information
\end{quotation}

\paragraph{Summary-2010:} The most critical issue identified by the committee in 2010 was the need to choose a technology for NEXT. As described in previous reports, such a choice was one of the first goals of CUP. At the time there were two possibilities. a) A readout based in Micromegas, pioneered by the group of Zaragoza, and b) an electroluminescent readout, using only light signals, and instrumenting the detector with SiPMs and PMTs. In 2010, the committee observed that the Micromegas technology did not meet the energy resolution requirements for the experiment, but also noticed that the EL option had not shown full feasibility. The committee recommended the construction of an intermediate prototype. Notice that none of the three NEXT prototypes (NEXT-DEMO at IFIC, NEXT-DBDM at LBNL, and NEXT-MM at Zaragoza) were yet fully operational, and the available results were based mostly in small prototypes at IFIC, LBNL and Zaragoza. 


\subsubsection*{2010-April}
\begin{quotation}
LSC Scientific Committee \\
8th Meeting \\
May 26th and 27th, 2011 \\
EXP-05 (NEXT)\\

{\em The committee congratulates the NEXT collaboration on very substantial progress made towards the definition of the conceptual design for the xenon double beta decay experiment}. A major decision has been the choice of electroluminescence as the gain mechanism. The tracking readout would use Si-multi-pixel photon counters coated with TBP with devices on a 1 cm grid. The energy readout would employ an array of PMT’s on the opposite side of the detector from the tracking readout. The chamber would operate at 15 bar. The team has demonstrated that the preferred PMT’s would not survive at such pressures and the design has them housed in titanium cans with a sapphire window. The high-pressure gas would be contained in a titanium pressure vessel. The design is based on successful operation of the NEXT-1 prototypes which have demonstrated good energy resolution at 667 keV which implies, an adequate energy resolution with the usual $\sqrt{E}$~ scaling at the \Qbb energy.
 
The technology choice was made just a few weeks prior to the meeting and the collaboration has done a good job of bringing forward the ‘conceptual design’ document\footnote{Published in \cite{Alvarez:2011my}} in such a short time. The committee encourages the team to continue to flesh out this conceptual design so that it can go forward for review and final approvals. 
 
{\em The committee concurs with the need to move expeditiously to build the detector if the results are to be competitive. The committee feels that the choice of electroluminescence is likely to lead to energy and tracking measurements adequate for the project}. The committee supports the conservative choice of a simple, titanium pressure vessel built to ASME Section VIII rules.
 
The committee has the following suggestions for the team:
 \begin{enumerate}
\item The use of sapphire windows in front of the PMT’s looks feasible but this concept should be demonstrated\footnote{The concept was refined in the Technical Design Report\cite{Alvarez:2012flf}}.
\item The readout of the MPPC’s needs to be developed further because of the high multiplicity of these devices\footnote{Demonstrated in \cite{Abazajian:2012ys}}.
\item The project needs to demonstrate that large area coating of the MPPC’s is feasible and that the stability of the coatings is adequate\footnote{Demonstrated in \cite{Alvarez:2012ub}}.
\item The concept for calibrating the detector needs to be developed\footnote{Demonstrated
in \cite{Alvarez:2012haa}.}. 
\item The background model needs to be developed further involving both analysis of the signals and the likely sources. The ability to discriminate alpha, beta and gamma backgrounds should be part of this analysis. From this would follow the materials purity criteria for each of the components and hence the need for radioactive screening. A request for screening facilities should be presented to the laboratory as soon as possible\footnote{Demonstrated
in the following papers: \cite{Alvarez:2012as,Alvarez:2013wxj,Dafni:2014yja,Alvarez:2014kvs,Cebrian:2015jna}}.
\item It was suggested that the conceptual design of the project would proceed separately for the different systems that make up the detector. The committee feels that it is important to have a single overall conceptual design. Once the framework is established, it is possible to have separate detailed design activities but the design criteria and interfaces must be complete at the conceptual design level to enable project review, approval and funding decisions to be made.
\item A work breakdown structure (WBS) with institutional assignments was presented orally. It is important to develop the WBS as part of the CDR. Letters of commitment from the US institutions were also presented. It is also important to get similar letters from all of partners especially in light of the technical choices that have been made.
\item The project should develop a risk analysis. 
\item The project team should develop a roadmap for completion and operation of the detector. Important elements of such a roadmap are stages in the development at which reviews would be carried out (and who would do these), and when critical decisions need to be made (including external decisions such as laboratory approvals and funding decisions).
\item The support systems and the pressure vessel are major engineering projects. The project must indicate the processes that will be followed to provide the necessary quality assurance that these will present adequately low risk.
\end{enumerate}

{\bf The committee congratulates the Collaboration for it progress since the last Scientific Committee meeting, and strongly encourages them to pay particular attention to the above recommendations. }

\end{quotation}

\paragraph{Summary-April 2011:} April 2011 was a very important milestone for several reasons: a) the collaboration 
{\bf had made a choice on the readout technology}. That choice involved an internal discussion in the collaboration, after the presentation of {\bf Conceptual Design Reports Proposals}, CDRP. Two such proposals were presented, one, supporting the choice of electroluminescence (EL), signed by most of the collaboration groups. The second CDRP presented micromegas as an alternative and was signed by the groups of IFAE and U. Zaragoza. The CIEMAT group did not sign any of the CDRPs.

The choice of the EL technology, proposed by D. Nygren (then at Berkeley National Laboratory, now professor at U. of Texas at Arlington), was based in the excellent preliminary results obtained with the initial operation of NEXT-DEMO and NEXT-DBDM, which indicated that a resolution of 1\% FWHM was possible. Those initial results were later supported by extensive studies which have shown a resolution in the range of 0.5--0.7 \% FWHM  \cite{Alvarez:2012yxw, Alvarez:2012zsz,Alvarez:2012hu,Alvarez:2013gxa,Lorca:2014sram,Renner:2014mha,Serra:2014zda} and a very strong topological signature  \cite{Ferrario:2015kta}. 

Studies carried out with the NEXT-MM prototype have shown that the energy resolution using a micromegas system is of the order of
3\% FWHM at \Qbb \cite{Alvarez:2013oha,Alvarez:2013kqa} even when adding a quencher (TMA). The addition of TMA, however,  suppresses the primary scintillation light and therefore the start-of-the-event ($t_0$) signal. As a consequence the event cannot be localised in the longitudinal coordinate (z) and the backgrounds increase enormously. The choice made in 2011, therefore, appears fully justified. 

{\em Notice that the committee supported the choice of EL and the plan to build the NEXT detector as fast as possible.}





\subsubsection*{2012-May}
\begin{quotation}
LSC Scientific Committee \\
10th Meeting\\
May 9th and 10th , 2012\\
EXP-05 (NEXT)\\

{\em The NEXT Collaboration presented enormous progresses both on technical and on managerial domains}. We appreciated, in particular, the points listed below:
\begin{itemize}
\item The experiment presented an excellent quality assurance plan for qualifying the Sapphire windows, a very delicate part of the experiment. The effort to guarantee the soundness of the windows is very convincing. In addition NEXT made large and constructive effort to work together with manufacturers to assure the quality of the windows when they were at the level of raw material before purchasing them.
\item The NEXT spokesman presented a very convincing and systematic study of the homogeneity of the TPB coating. The quality of the coating depends on the exposure time and consequently on the thickness, and the spinning velocity. Homogeneity was studied using glass plates with the same dimensions (30??30??3 mm3) of the final one. They were coated with different TPB thicknesses (0.6, 0.2, 0.1 and 0.05 mg/cm2). Since the study was to reproduce the final set-up, we are confident that the positive results obtained by the Collaboration reflect those that will be obtained in final production \footnote{See  \cite{Alvarez:2012ub}}.
\item The long-term stability of the TPB depositions is an important issue for NEXT. They realized that poor storage of TBP could result in very poor quality of the deposition layer and therefore jeopardize the detector. They have a good and safe plan to avoid this potential problem.
\item NEXT conducted very systematic aging studies on 14 SiPM after a storage time of 9 months in a controlled environment. They successfully demonstrated that the relative current variation of the 14 SiPMs is less than 1\%, which is well within the experimental uncertainties. 
\item The Collaboration developed a very successful protocol for coating the SiPMs with TPB used as WLS. In particular the precautions adopted for obtaining clean and uniform coatings allowed them to obtain optimal fluorescence 
efficiency \footnote{See  \cite{Alvarez:2012ub}}. 
\item We acknowledge the impressive work devoted to estimating the radioactivity of all materials that constitute the energy plane system \footnote{See, in particular \cite{Cebrian:2015jna}}.
\item The Next-DEMO prototype was successfully and systematically studies with a Na22 source for many months. The results obtained with the prototype reinforce the soundness of the experiments \footnote{ See  \cite{Alvarez:2012yxw, Alvarez:2012zsz,Alvarez:2012hu,Alvarez:2013gxa,Lorca:2014sra,Renner:2014mha,Serra:2014zda, Ferrario:2015kta}}.
\item The NEXT Collaboration demonstrated that Teflon coating of all surfaces enormously augmented the light collection. They can therefore trigger on only S1 signal! Unfortunately, the life-time of electrons is very small, which may be due to some water contamination that occurred during the coating process. We encourage the Collaboration to follow this point and report on it at the next LSC meeting.
\end{itemize}

We acknowledge the efforts and progress that the Collaboration did progressing toward NEXT-100, namely:
\begin{itemize}
\item The pressure vessel is steel. NEXT measured the activation of the material and it is well below the expected limit. We are very pleased to hear that the Collaboration has already procured this high quality material. 
\item We acknowledge that the order for the procurement of all the PMTs for the energy plane has been completed.
\item We appreciate the enormous effort made by the Collaboration to understand the activation of each component in the tracking plane and the radioactive screening that they employ for material selection.
\item The Collaboration has created a very efficient QA test to check and screen the dice boards that carry the SiPMs.
\item Amplification region - Work is steadily and surely going on! NEXT will measure the deflection of the meshes for the large extraction grids. This is quite a delicate problem but they have clear indication that the deformation can be corrected via software. If not, in the case that the deformations are too large, a solution has to be found. We encourage the Collaboration to follow this point and report on it at the next LSC meeting.
\item Electronics for SiPM - We note that in the prototypes the electronics were positioned outside the volume and that the power consumption was 600 mW/ch. Now the electronics has been modified reaching an excellent level of power consumption (30 mW/ch). NEXT plans to place the electronics inside the volume and shield it with copper. The feasibility of this solution has to be demonstrated.  We encourage the Collaboration to follow this point and report on it at the next LSC meeting \footnote{After following the committee recommendations it was decided to place the electronics outside the detector.}.
\item We are pleased to see that seismic calculations were done for the platform. The Collaboration plans to have them verified by an external review company. We acknowledge the attention the Collaboration is paying with respect to the safety aspects of the experiment.
\item The lead castle is ready for reviewing and therefore ready for construction.?
\item Radiopurity: we are impressed by the screening and measurement campaign put in place by the Collaboration.
\item Managerial aspect: We congratulate the Collaboration for presenting a very convincing work-breakdown structure with convincing work packages (WP).
\end{itemize}
	
{\bf
The Collaboration made extraordinary progresses both on the technical aspects and as well on the managerial parts. We congratulate the Collaboration for these major steps forward of the project, and recommend that they continue on their present energetic path.}

\end{quotation}

\paragraph{Summary-May 2012:} In May 2012 the collaboration made enormous progress in convincing the
LSCSC that the technical and scientific choices made were sound, problems were being understood, progress was 
steady and managerial aspects were correct. However, the collaboration was traversing a difficult period, due to the
lack of support of the particle physics manager (``gestor''). See section \ref{sec.diff}.

\subsubsection*{2012-November}
\begin{quotation}
LSC Scientific Committee \\
11th Meeting \\
November 15th and 16th, 2012\\
EXP-05 (NEXT)\\

{\em
The committee is very pleased to learn of the continued rapid and comprehensive development of the NEXT concept and the technical demonstrations of performance. }

In particular the committee appreciates the enormous technical progresses on
\begin{itemize}
\item The start of construction of the pressure vessel\footnote{Currently completed.}.
\item The construction of the detector platform in the underground area\footnote{Currently completed.}
\item The completion of the baseline gas system already installed at the Laboratory\footnote{Full gas system, suitable for all the stages of the NEXT experiment currently completed.}
\item The procurement of the MPPC and PMT sensors.
\item The large screening campaign brought forward to guarantee that all the part of the experiment to be installed inside the vessel are radio-pure \footnote{See \cite{Cebrian:2015jna}}.
\item The enormous effort invested on the data reconstruction algorithm that showed an improvement of a factor 3 in the background rejection  \footnote{See \cite{Martin-Albo:2015rhw}}.
\end{itemize}

	
It is particularly encouraging to see that the energy resolution may really approach the Fano factor limit \footnote{See \cite{Alvarez:2012yxw}}. {\em This is a great advance in the development of a xenon-based detector for neutrino-less double beta decay}. The demonstration of diffusion well below the predictions of the simulations is also an important step for precision tracking. The group continues to set a high standard for the documentation for each component of the system. 

The committee is pleased to see that, although there are parts of the experiment not yet in hand, with the present equipment the collaboration can realize  a simplified baseline detector with which many basic features of the experiment can be investigated and understood.

{\em The committee was concerned to learn of the very tight financial situation that the project is in}\footnote{The collaboration was traversing a difficult period, after our 2011 proposal for funding, needed to complete the construction of the experiment and to keep the team of physicists and engineers at the various national institutions involved, was rejected. See section \ref{sec.diff}}. It is very encouraged to learn that in the US, the Fermi National Accelerator Laboratory and Argonne National Laboratory will join the Lawrence Berkeley National Laboratory in requesting support for the project from DOE to provide vital parts of the final detector, namely: the energy plane, the field cage and the power supplies. The committee looks forward for a positive feedback from the DOE expected by July of next year\footnote{NEXT has received sizeable support from the US, in particular from the group of prof. D. Nygren. Currently, the US group are preparing a proposal to further contribute to the experiment.}

We are also encouraged to learn of the move to make the project an recognised CERN experiment\footnote{NEXT is now a CERN recognised experiment}. This move, if successful, not only could bring additional financial and manpower resources to the project but would also increase the international profile of the work. {\em This may help to improve the domestic attitude towards the program. }

The decision of the collaboration to proceed as quickly as possible to a technical demonstration of the strength of detector concept and to add those features that may prove essential to the lowest background later is strongly supported. The precarious state of support for the Spanish team is of great concern as the project moves towards implementation.{\em  It would be a serious loss to Spanish science if the financial constraints lead to a strong domination by its international partners in the exploitation of this experiment}.

{\bf
The Collaboration made extraordinary progresses in solving technical issues as well as expanding the U.S. participation. We congratulate the Collaboration for these major steps forward of the project, and recommend that they continue on their present energetic path. 
}

\end{quotation}

\paragraph{Summary-November 2012:} In November 2012, the LSCSC was satisfied with the progress made by 
NEXT while at the same time seriously worried by the lack of domestic support. See section \ref{sec.diff}.


\subsubsection*{2013-May}
\begin{quotation}
LSC Scientific Committee \\
12th Meeting \\
May 28th and 29th, 2013\\
EXP-05 (NEXT)\\

{\em The Committee congratulates the NEXT Collaboration for the continuous and steady progress that strengthens the soundness of the NEXT concept}. In particular, NEXT has made numerous technical advancements with the NEXT-Demonstrator that has been running for 6 months\footnote{Refers to NEXT-DEMO. See \cite{Alvarez:2012yxw, Alvarez:2012zsz,Alvarez:2012hu,Alvarez:2013gxa,Lorca:2014sra,Renner:2014mha,Serra:2014zda,Ferrario:2015kta} }. The entire system is remotely controlled. The following important milestones have been met: 

\begin{itemize}
\item The system is very stable and typically runs for 8 days without errors. 
\item A very efficient calibration system for the Silicon Photomultipliers (SiPM), was set-up based on X-rays; excellent energy resolution was measured with radioactive sources of 22Na and 137Cs.
\item A very impressive demonstration was given of the reconstruction of three-dimensional (3D) tracks with the new SiPM tracking plane \footnote{See \cite{Ferrario:2015kta} }.
\item 
\end{itemize}
	 
{\em The NEXT project shows impressive progress on the mechanical structure as well as the general infrastructure}, namely:
\begin{itemize}
\item The working platform is completely installed.
\item The seismic structure and the castle are ready to be built and the tenders are ready to be issued.
\item  The pressure vessel will be ready by the end of June 2013.
\end{itemize}

{\em	
Much progress has been made on the assembly of NEXT-100}, namely:
\begin{itemize}
\item A new design for the PMT housing was completed that will make them gas tight.  
\item Two approaches are being studied and tested to guarantee that the SiPM tracking plane feed-troughs will be gas tight.
\end{itemize}

	
The NEXT Collaboration clearly demonstrated that they have the issues related to the SiPM base radio-purity under control. The problems were traced to the PCB board, the resistors and the capacitors.
Some possible Improvements to the tracking plane are being evaluated and will be followed up. The Collaboration intends to:
\begin{itemize}
\item improve the radio-purity of the adhesive of dice-board that is composed of 4 layers glued together,
\item develop a new soldering procedure for the  SiPM and the cables.
\end{itemize}

	
Overall, the NEXT team has made a deep and systematic study of all services and connections associated with the tracking plane. The new SiPM electronic system is much improved; it lowers power consumption and cost, and is considered ready for production.

A great deal of effort has been made to keep the radio-purity under control by screening every element installed in the detector. These steps have achieved a projected background level of $\mathrm{(4-9) x 10^{-4} counts \cdot keV \cdot kg \cdot yr}$. This value is totally dominated by parts that will be replaced with those with lower background, and therefore the overall background should be reduced by at least a factor three. With these values, NEXT will be very competitive with other experiments.

From the technical point of view, the construction of the NEXT-100 detector could proceed during 2014; {\em however, as of this writing there is not enough money to pay for the full detector. In addition, there are severe funding uncertainties. The committee is deeply concerned about the lack of adequate funding for this flagship project of the Laboratory}.

{\bf
	The Collaboration has made extraordinary progresses in solving technical issues; however, the lack of funding is of great concern. The Committee recommends that the Collaboration continue and even increase its efforts to recruit university collaborators in Europe and in the US. In any case it would be highly desirable for NEXT to move its upcoming prototype to LSC for operation.
}
\end{quotation}

\paragraph{Summary-May 2013:} In May 2013, progress was steady solving technical issues and producing scientific results. However, funding uncertainties due to the lack of support in FPA were holding back the project. See section \ref{sec.diff}.

\subsubsection*{2013-November}
\begin{quotation}
LSC Scientific Committee \\
113th Meeting\\
November 28-29, 2013\\

EXP-05 (NEXT)\\

{\em The committee congratulates the NEXT collaboration for the significant steps forward towards the construction of a fully-funded large-scale apparatus (named NEW)}, capable to validate the experiment concept in time for a feedback on the construction procedures of NEXT-100. It is also happy to acknowledge that crucial components of NEXT-100 setup are now funded and under construction.

The small scale prototypes developed so far have demonstrated the good performance of the electroluminescence technology in terms of energy resolution. The topological signature of electrons has also been proved, even though a two-electron track has not been observed yet. The committee invites the NEXT collaboration to verify if this can be achieved with a proper choice of the calibration source in the NEXT-DEMO prototype, exploiting in particular electron-positron pair production in the gaseous target\footnote{The collaboration followed the recommendation, and two-electron tracks were measured, see \cite{Ferrario:2015kta}}.

The committee remarks however that the present understanding of the background in the NEW project is at a very preliminary stage. It recommends therefore that the NEXT collaboration present a detailed Monte Carlo-based evaluation of the NEW background. In particular, the committee asks the collaboration to identify clearly the dominant contributions to the background in the region of interest for neutrino-less double beta decay, with a special focus on the role of 214Bi, which emits with gamma rays whose energy is very close to the 136Xe Q-value. In this context, a detailed analysis of possible 222Rn contribution is mandatory and an evaluation of the radon emanation from the tank walls should be scheduled in the near future\footnote{The collaboration followed the recommendation and a full
background model has been produced, see \cite{Martin-Albo:2015rhw}}.

{\em The committee congratulates the collaboration on securing new funding which it understands will fully support the NEW project}. The committee would like to receive at its next meeting, an estimate of the additional funding required to complete the full NEXT-100 project.
\end{quotation}

\paragraph{Summary-November 2013:} In November 2013 the committee learned that Gomez-Cadenas had obtained and Advanced Grant of the ERC. At the time, the experiment has received clear support from the
Secretary of State for Science (SEIDI) which unblocked the difficult situation of 2011 and 2012. However, the uncertain funding situation and the share complexity of the project had convinced the collaboration to stage the construction of the NEXT-100 detector in two phases. The first phase, called NEW is a 1:2 replica of NEXT-100 (half the longitudinal dimensions, a fourth of the sensors and a 10th of the mass), which is large enough to fully understand technical issues as well as to refine the background model. The collaboration has focused all its efforts in NEW, and the detector is currently being commissioned at the LSC. 

\subsubsection*{2014-May}
\begin{quotation}
LSC Scientific Committee \\
14th Meeting\\
May 29-30, 2014\\

EXP-05 (NEXT)\\

{\em The committee wants to congratulate the NEXT Collaboration for the continuous and steady progresses that strengthen the soundness of the NEXT concep}t. Among the excellent work produced, in many cases with support from the LSC staff, the points listed below were particularly appreciated:

{\bf Infrastructure}:
\begin{itemize}
\item The completion of the castle structure.
\item The procurement of all the lead needed to construct the shield (it comes from OPERA as a loan for the NEXT lifetime)\footnote{This loan has to be read as an additional external support to NEXT and the LSC, and its credit rests mainly in the personal prestige of the then-director, Alessandro Bettini.}. 
\item The plan to clean the 60 t of lead in an industrial way on the platform.
\item The progresses in construction and understanding of the mechanics of the NEW detector.
\item The detailed design and construction of the pressure vessel and the cleaning protocol adopted.
\item The new solutions adopted for the mechanics of the Energy Plane.
\item The use of an appropriate optical gel that couples effectively the sapphire window to the quartz window of the PMT; the response has increased by a factor 2.
\item The high vacuum quality tested, using RGA.
\item Operation with argon up to 20 bar pressure with a leak rate as low as 0.01 g/y (expected to diminish with Xe).
\item The new solutions adopted for Tracking Plane.
\item The successful New Kapton pig-tail Design.
\item The extensive investigations pursued to optimize the feed-through.
\item The completion of the simulation of the electric field cage (COMSOL Multi physics) with which it is possible to study in details different regions of the cage.
\item The large campaign of radio-purity measurements; all components are carefully measured to assure that any contaminated element is introduced in the detector. Actually some components were eliminated from the production chain.
\item The progress in understanding the functioning and the performance of the detector.
\item The careful analysis of the background induced by different radio-elements\footnote{see 
\cite{Martin-Albo:2015rhw}}.
\item The study performed with point-deposited energy from X-ray: drift velocity studies, calibration of PMT, study of energy resolution \footnote{See \cite{Lorca:2014sra}}.
\item The successful detection of the electron tracks at the double escape peak generated by a Thorium source  \footnote{See \cite{Ferrario:2015kta}}. 
\item The extensive studies of TMA as quenching gas – More investigations are needed at this point; in particular to prove the compatibility of TMA with the materials that constitute the detector\footnote{See \cite{Alvarez:2013oha,Alvarez:2013kqa}}.
\end{itemize}
	

Together with many achievements, some concerns emerged during the review.
	 
The collaboration has identified the need to control radon in the detector environment. It should present a study of the sensitivity of the neutrino-less decay measurement in NEXT-100, to the level of external radon as a metric of the actual reduction required. They should then consider technical implementations such as a radon reduced air or nitrogen environment for the detector to achieve the required radon reduction\footnote{Currently a radon reduction abatement system has been purchased by the LSC and will be used to provide radon-free air to the NEXT experimental area}. 
The radon emanation from the pressure vessel and detector materials has not been presented. The committee recommends that this be investigated prior to the next meeting as this can be a significant source of background\footnote{Radon emanation measurements are under way in Cracow, Poland.}.
	 
The committee welcomes the aggressive schedule leading to installation of the NEW detector at LSC at the beginning of next year. However, the group needs to ensure that adequate time is allowed for testing of the new technologies being employed. The committee was particularly concerned about the high voltage structure and the sensitivity of the new anode structure to possible breakdown. The new anode structure in particular is expected to be very sensitive to electrical discharge and steps are likely required to ensure that breakdown does not occur, even at a low rate.

The committee was very pleased to learn that the collaboration has all funds required to complete the NEW phase of the project. This will provide an excellent proof of concept for the xenon gas detector. However, the committee, the project, and the laboratory all concur that the full potential of the project will only come from the construction of the full NEXT-100 detector. The committee is pleased to learn that there are opportunities to apply for further support to move ahead with the 100 kg detector. The committee strongly supports such a development and re-iterates its endorsement of the physics potential of the 100 kg gas detector as being a major contribution to the global search for neutrino-less double beta decay.


\end{quotation}

\paragraph{Summary-May 2014:} In May 2014, progress was steady solving technical issues and producing scientific results. The remarks of the committee were very relevant, and led to major improvements in the detector planning and procedures, in particular concerning the EL structure and the campaign to understand the effect of radon. In 2014, the leading groups of CUP/NEXT presented a join proposal for funding, which was evaluated with top ranks and granted. However the funding level was about half of the requested amount (due as limited budget, as explicitly stated in the resolution).  See section \ref{sec.diff}.

\subsubsection*{2014-November}
\begin{quotation}
LSC Scientific Committee \\
15th Meeting \\
November 20,21, 2014\\

EXP-05 (NEXT)\\

{\em The committee congratulates the NEXT Collaboration for the continuous achievements and the impressive progresses on the construction of the first detector demonstrator and the understanding of its basic performances.} Among the excellent work produced, the points listed below were particularly appreciated:

{\bf Infrastructure}:
 \begin{itemize}
\item The completion and installation of the platform and castle structures
\item The completion of the gas system.
\end{itemize}
	 
 
{\bf The progresses of the NEW detector }
  \begin{itemize}
\item The completion of the pressure vessel at the factory 
\item The production of the Energy Plane and the assembly test of the sapphire window (including the use of brass screws)
\item The strategy adopted for cleaning, assembly and test the vessel and the energy plane.
\item The performances of the new SiPMs produced by SensL (better radio-purity, gain, temperature dependence and dark current wrt the Hamamatsu one).
\item The solutions adopted for Tracking Plane
\item The potting solution adopted for the feed-through 
\item The decision of constructing a full mock-up of the Tracking Plane.
\item The quartz anode plane (featuring the Dark Side approach) is ordered; a series of groves is foreseen on the mechanical structure in order to limit the effect of the discharges.
\item The clear experimental reconstruction in the NEXT-DEMO prototype of two-electron tracks corresponding to pair production with a 232Th source, requested by the Committee one year ago and now successfully achieved\footnote{ See \cite{Ferrario:2015kta}}.
\item The progress in understanding the functioning and the performances of the detector and the excellent results obtain on radio-purity measurements and background suppression. In particular, the Committee has appreciated the detailed evaluation of the contributions to the final NEXT-100 background in the Region of Interest (ROI) from the various detector elements, based on an intensive campaign of specific radioactivity measurements. 
\item The very convincing studies of Radon contamination that conclude that the rate can be suppressed by 3 orders of magnitude provided that a decontaminated atmosphere surrounds the detector. The Committee would like to stress that for the first time the Radon issue has been approached with a fully quantitative analysis, establishing precise correlations (and therefore well-defined targets) between the 222Rn activity level outside and inside the detector and the background in the ROI for NEXT-100.
\end{itemize}

{\em The scientific committee acknowledges the large progresses made by the NEXT experiment: the new design and hardware solutions presented, the tests and the simulation performed.  The reviewers did not identify any substantial problem and therefore encourage the NEXT group to proceed at full speed to the construction of the NEW detector}.  The team should work with the laboratory to ensure that any safety requirements are understood and planned for. The team should also understand that laboratory approval will be required prior to use of the enriched isotope. This approval will require a clear scientific justification as well as a demonstration of technical readiness.

\end{quotation}

\paragraph{Summary-November 2014:} The November 2014 meeting was characterized by steady progress, and a sense of optimism related with better funding prospects. 
\subsubsection*{2015-April}
\begin{quotation}

LSC Scientific Committee \\
16th Meeting\\
April 20-22, 2015\\
EXP-05  NEXT  \\

{\em The committee wants to congratulate the NEXT Collaboration for the continuous progresses and the impressive achievements on the construction of the first detector demonstrator}. Among the excellent work produced, the points listed below were particularly appreciated: the pressure vessel and inner copper shield have been installed and cleaned; the gas purification and recovery system is well advanced; the read out electronics for the energy plane is installed and the complete assembly of this system is progressing well; and progress is being made towards completion of the tracking plane.

Despite this excellent technical progress, there are two areas of concern. 

A first risk analysis has recently been provided to the laboratory. A dialogue between the LSC and the collaboration has started with the aim of bringing this to a level acceptable to the laboratory but the experiment might expedite this process by getting input from a consultant with experience in producing such analyses for underground physics experiments.

A second concern is that it has been realized that the pressure vessel for the experiment has not been certified as required under EU law. The experiment will investigate how this certification might be obtained\footnote{The PV has been successfully certified.}.

In order to help prevent this type of problem in the future, it is recommended that the LSC consider producing a document listing all approvals required for equipment to be located in the laboratory. While production of such a document is labour intensive, the committee suggests that it may be possible to obtain such a document from another European underground laboratory at least to use as a starting point.

\end{quotation}
\paragraph{Summary-May 2015:} The May 2015 witnessed steady progress. Several issues with safety certification were identified, highlighting that the safety and risk control procedures of the LSC were not yet fully spelled out. During 2015 and the first half of 2016, much work has been invested in defining such procedures and carrying out the needed risks analysis. 

subsubsection*{2016-November}
\begin{quotation}
LSC Scientific Committee\\
17th Meeting\\
November 5-4, 2015\\
EXP-05 NEXT
{\em 
The committee congratulates the NEXT Collaboration for the continuous progresses and the impressive achievements on the detector, electronics and Infrastructures.}

Among the excellent work produced, the points listed below were particularly appreciated:
\begin{itemize}
\item The aggressive New schedule and installation campaigns that will allow  to have, by the end of 2015
\begin{itemize}
\item Integration and cabling of the front-end electronics, commissioning of DAQ, and consequent tests of the tracking plane sensors\footnote{Achieved.}.
\item An operative gas system\footnote{Achieved.}.
\end{itemize}
\item The aggressive road-map to winter 2016 that will permit:
\begin{itemize}
\item The installation of the gas system and commissioning of the slow control\footnote{Achieved.}.
\item The installation of the field cage\footnote{Scheduled for April, 2016.}.
\item Commissioning of the detector by March 2016 and initial results by the end of the year \footnote{The commissioning of the detector started in March, as schedule and initial results are expected by May, 2016.}
\end{itemize}
\item The effort of presenting an updated TDR of NEXT-100 by November-2016 meeting. If so, the construction of major parts of NEXT-100 could start in 2017.  In this framework it was appreciated that no changes are foreseen for the FEE and DAQ.
\item The effort invested in improving the infrastructures that will be fully commissioned by end of the year (in order to reduce the background, the collaboration plan to construct a Cu shielding hub).
\item Progresses of the NEW detector: 
\begin{itemize}
\item NEW vessel: it has been certified for operation with pressure and it is installed on its platform enclosed by a fully-functional lead castle.
\item Energy Plane (EP): is completed and was successfully tested up to 15 bar!   Only one out of the 12 PMT broke and will be investigated in the incoming future\footnote{PMT replaced, energy plane fully operational.}.
\item Commissioning of the detector by March 2016 and initial results by the end of the year \footnote{The commissioning of the detector started in March, as schedule and initial results are expected by May, 2016.}
\item Tracking plane (TP):  all 33 Dice Boards are tested and ready to go with all the necessary services (just 1 SiPM out of 2000 failed). Noticeable that all feed-through were tested and are qualified\footnote{Tracking plane already installed in detector.}
\end{itemize}
\item FE electronic: all ready and tested.
\item Read-out and DAQ electronic: all ready and tested.
\item Gas system: all parts are in hand, will be assembled, will undergo internal and external refereeing and will be certified by Cryvoc by February 2016. A risk analysis will be performed afterwards.
\item  Gas system safety and control protocol: A very detailed and complete  protocol was presented; it will  allow monitoring  all relevant parameters  that can jeopardize the good functioning of the system and will take immediate safety  actions.
\end{itemize}


Together with many achievements, some concerns emerged during the review. The reviewers solicit clarifications, or follow-up, on the items listed below:
\begin{itemize}
\item NEW requires the construction of a sealed containment around the NEW detector shield fed with radon reduced air. Details of this system need to be worked out with the Lab management\footnote{The problem is solved with the new radon abatement system, that allows pumping radon-clean air into the NEW experimental area.}.
\item The gas Slow Control system needs to be finalized and commissioned\footnote{Slow Control finalized, commissioning in progress. }
\item The HVFT just arrived from Cryvac; it is not up to specifications and has to be returned to the company. A second piece will arrive next week but the break-down tests have still to be performed and the connector has still to be qualified\footnote{Two new HVFT have been manufactured. Testing under way. }
\item Computing and infrastructure: the radio link of 32Mb/sec is insufficient to send data to IFIC and should be improved. This item might take time since it depends on the Jaca region and not only on LSC. A back-up solution needs to be put in place.
\end{itemize}

{\bf
The scientific committee acknowledges the large efforts and huge progresses made by the NEXT Collaboration in the assembly of NEXT-NEW a flagship project of the laboratory.}

The schedule for bringing this project into operation is aggressive and to be successful, it needs to be carefully coordinated with the laboratory. The committee urges the experiment to provide a detailed plan to the laboratory, including milestones at which the safety analyses will be available for the lab’s review. Adequate time for such reviews must be included in the plan\footnote{Two such reviews have been carried out during 2016.}.

\end{quotation}
\paragraph{Summary-November 2015:} The November 2015 was also characterized by steady progress. The recommendations of the committee have been followed, in particular in the establishment of a safety protocol, risk analysis and slow control, of which have been major tasks whose importance was no fully perceived in the initial stages of the project. In that sense, the NEW stage is proving extremely successful, since it allows us to work out the safety, controls and infrastructures for the full experiment. 



\section{\bf \textsf{Appendix B: The EL choice}}

In this appendix we summarise the main arguments to choose the EL technology for NEXT. They are:

\begin{itemize}
\item {\bf Energy resolution}. This is a crucial feature of any \bbonu\ experiment, and one of the main advantages
of NEXT over its competition (the other two experiments, EXO and KamLAND-Zen have respectively 4\% FWHM and 10\% FWHM resolution at \Qbb, while the EL option of NEXT yields $\sim$0.5 \% FWHM). The electroluminescent amplification of the signal, being lineal permits to achieve resolutions near the Fano factor {\em in gas xenon}. Any system that operates with non-lineal gain (e.g, with avalanche) as is the case of Micromegas, needs to face gain fluctuations, which spoil the resolution. The NEXT collaboration has measured the resolution that can be attained both in a EL and in a MM system. Indeed, NEXT-DBDM found a resolution near the Fano factor \cite{Alvarez:2012yxw},
($\sim$ 0.5 \% at \Qbb), while NEXT-DEMO has measured a resolution of $\sim$ 0.7 \% at \Qbb \cite{Alvarez:2012zsz, Alvarez:2013gxa, Lorca:2014sra}, all in pure xenon. Instead the NEXT-MM prototype, operating with a mixture of Xenon and TMA measures a resolution that extrapolates to 3.2 \% FWHM at \Qbb \cite{Gonzalez-Diaz:2015oba}.
\item {\bf Operation with a quencher}. It is very difficult to operate a gain system at high pressure with pure xenon, and indeed, the NEXT-MM prototype has operated with a mixture of xenon and trimethylamine (TMA). Indeed, as
demonstrated in \cite{Gonzalez-Diaz:2015oba} reduces the diffusion and introduces a beneficial Penning effect. On the other hand, it totally suppresses the primary scintillation light in xenon, thus killing the start-of-the-event signal 
($t_0$). Without $t_0$~the background of the experiment grows by a large factor, as understood since a long time
\footnote{See: R. Luescher et al. Phys. Lett. B434 (1998) 407;
J.C. Vuilleumier et al. Phys. Rev. D48 (1993) 1009.}. 
\end{itemize}

%\section{\bf \textsf{Appendix C: Support documentation}}

\subsection*{Resignation letter (IFAE)}


\begin{quotation}
Dear Gloria and Juanjo (cc. IB board),

In the last days, we have been carefully evaluating the recently approved NEXT CDR, particularly regarding IFAE participation. We have balanced pros and cons and we have concluded that the IFAE group will not sign this CDR for a 100 kg detector. There are many reasons for this but they can be condensed in four points:
\begin{enumerate}
\item the proposed technology has not been demonstrated to be a viable one, neither from the performance nor from the construction points of view.
\item We consider the proposed schedule as too challenging in the scope of the previous point and considering the experimental progress towards this design since the LoI 2 years ago. Therefore, we have serious doubts that the experiment will be taking data in 2014 as a competitive alternative to EXO or any other of the main competitors in the DB field on the 100 kg scale.
\item In our opinion, the technology election procedure has not been appropriate, given the lack of a consensus within the collaboration or the evaluation from and independent, external review panel.
\item The lack of information and scientific discussion within the collaboration, particularly during the critical past months, has led us to lose confidence in the capabilities of the collaboration’s management bodies to coordinate and integrate the efforts of the NEXT groups towards the important challenges the collaboration will face in the near future.
Given all this, we hereby waive our membership in the NEXT Collaboration, wishing you sincerely all the best with this challenging experiment.
\end{enumerate}
 
Javier Rico and Federico Sanchez
\end{quotation}

\subsection*{Report of ANEP referee to the 2011 panel (extract)}


\begin{quotation}
\item El IP tiene un historial lleno de luces y de sombras, y sus logros cient\'ificos no est\'an tan claros como \'el expresa\footnote{A subjective ``gut feeling'' statement, not backed by fats}. \'Ultimamente dedica m\'as tiempo a la escritura de novelas u otros libros que a ocuparse de los experimentos y de los consolider en los que tiene muchas responsabilidades\footnote{Ad hominem}.
\end{quotation}

\section{\bf \textsf{Appendix C: NEXT schedule}}

A very important consideration in internationally competitive experiments is their schedule. In fact, this point 
has been recently discussed, in the context of doble beta decay experiments, by the European Consortium 
in Astroparticle Physics (see talk by A. Giuliani and summary talk by F. Linde in the APPECC Town Meeting 
2016 \footnote{\url{
http://app2016.in2p3.fr/programme.html}}). In summary, the committee considered that running and projected experiments, including NEW-NEXT,  
are in a comparable position to compete for scaling up to the future ton experiment. The evaluation of the next two 
years performance of the available technologies would be crucial to make the right choice. An international 
panel of experts will be appointed by APPEC to work out recommendations.

In more details, EXO has completed his physics run with an exposure of about 100 kg $\cdot$ year, and its sensitivity
is dominated by backgrounds. This is also the case of KamLAND-Zen (both experiments are based in
xenon and are the direct competition to NEXT). An HPXe offers better resolution and better control of
backgrounds. NEXT could improve the results of both EXO and KamLAND-Zen, and explore a region
not yet covered by the competition in only two years of run\cite{Martin-Albo:2015rhw}.
Xenon has the great advantage (compared with other technologies like germanium diodes or bolometers)
to be cheap and easy scalable. In fact, it appears clear that the choice of a technology for the
ton-scale will take a few years. It is also clear that a successful operation of NEW will already situate the
NEXT technology as one of the major alternatives for the ton scale.

%A consistent criticism to the NEXT project has been the schedule. Indeed, the initial ambition of having a fully operational 100 kg detector in 2014, has proven to be too optimistic. This is not unheard of in ambitious scientific projects (including the LHC, which started with about 8 years delay) but is important to understand the reasons and to assess the implications.

%The main reason why the initial schedule was unrealistic is that the project proposed the construction of a complex and technologically revolutionary machine, as is the NEXT detector, to be operated in a new underground laboratory. In 2010, nobody in the World had built a large HPXe-EL detector. There were only a few experts (among then the later and much missed prof. J. White) who had the knowledge and technical proficiencty to attack the problem. In Spain, the know-how was absent and the labs lacked all the specialised equipment and instrumentation. Indeed, the construction, commissioning and operation of NEXT-DEMO in only two years (the detector was fully operational in 2012) was considered a major feat (as for instance in the reports of the referees during the AdG evaluations of 2013, or the reports of the LSCSC). Scaling the technology from DEMO to NEXT-100, required solving additional problems and understanding radiopurity issues. It could had been attempted in a single step (as was the opinion of the US groups and also of the LSCSC) but the lack of support of FPA in 2011 made totally inviable that option. The collaboration decided, instead, stage the experiment, building first the NEW detector, which is currently being commissioned at the LSC.
%
%With hindsight we believe that the staging decision was correct. 
%and that the skepticism regarding the accelerated schedule initially intended was justified (leaving the collaboration, rather tan contributing to improve it and denying support to the effort, as was the FPA reaction is an altogether different matter). 
%The NEW stage has revealed not only technical issues best addressed by an intermediate detector, but also the need to develop carefully the major infrastructures (gas system, shielding, emergency recovery system, slow controls) needed by the NEXT-100 experiment. It is important to realise that the LSC laboratory is completely new and lacked, only a few years ago, the infrastructures and facilities available at other undergrounds labs such as the LNGS. Instead, the febrile activity associated to NEXT has been instrumental for the development of such infrastructures, that range from lead shielding to a state-of-the-art Radon abatement system. 
%%All this requires time. 
%Notice that the commissioning of NEW is also commissioning the infrastructures and services needed for NEXT-100. 
%
%Another major element is safety. It was recognised very early by the LSCSC that a deep risk assessment was necessary to operate a large high pressure system underground, but the LSC did not have a safety engineer until about one year ago. Safety studies could not have been conducted three years ago at the LSC due to the lack of personnel and proper protocol. Indeed, the quality of the laboratory has been improving along the development of the NEXT project. 
%
%The current schedule calls for NEW operation in 2016 and 2017. NEXT-100 can be built in one year, given the fact that the major infrastructures, slow controls and safety studies are in place and considerable expertise have been gained building NEW. However, the construction of the larger detector requires the full support from the Spanish system of science (NEXT needs sufficient support both in instrumentation investment and in personnel in 2019, when the ERC and current SEIDI grant runs out).  
%
%An argument often heard of is that NEW arrives late to the international competition. This is not the case as can be clearly seen in the recent presentation by A. Giuliani in the APPECC Town Meeting 2016\footnote{\url{
%http://app2016.in2p3.fr/programme.html}}. In fact, EXO has completed his physics run with an exposure of about 100 kg $\cdot$ year, and its sensitivity is dominated by backgrounds. This is also the case of KamLAND-Zen (both experiments are based in xenon and are the direct competition to NEXT). An HPXe offers better resolution and better control of backgrounds. NEXT could improve the results of both EXO and KamLAND-Zen, and explore a region not yet covered by the competition in only two years of run\cite{Martin-Albo:2015rhw}. 
%
%Xenon has the great advantage (compared with other technologies like
%germanium diodes or bolometers) to be cheap and easy scalable. In fact, it appears clear that the choice of a technology for the ton-scale will take a few years. It is also clear that a successful operation of NEW will already situate the NEXT technology as one of the major alternatives for the ton scale. 
% Science is a high-risk, high-reward business. We do believe that a discovery is still possible. 


%\bibliographystyle{elsarticle-num}

\bibliographystyle{plain}
\bibliography{biblio}

\end{document}

